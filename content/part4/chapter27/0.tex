\noindent
\textbf{内容概要}

\begin{itemize}
\item
什么是多线程编程

\item
如何启动多个线程

\item
如何取消线程

\item
如何从线程检索结果

\item
什么是死锁和条件竞争,以及如何使用互斥量来避免

\item
原子类型和原子操作

\item
什么是条件变量

\item
信号量、门闩和栅栏

\item
future和promise的线程间通信

\item
什么是线程池

\item
什么是可恢复的函数或协程
\end{itemize}

本章的所有代码示例都可以在\url{https://github.com/Professional-CPP/edition-6}获得。

多线程编程对于具有多个处理器单元的计算机系统非常重要。它允许编写一个程序,以便并行使用所有这些处理器单元。系统拥有多个处理器单元有多种方式,系统有多个独立的中央处理单元(CPU)的处理器芯片,或系统有一个内部包含多个独立CPU的单一处理器芯片,这些CPU也称为内核。这些类型的处理器称为多核处理器(系统也可以同时拥有这两种类型的组合),具有多个处理器单元的系统已经存在很长时间了;然而,在消费者系统中使用得并不多。今天,所有CPU供应商都在销售多核处理器,这些处理器广泛应用于从服务器到消费者计算机,再到智能手机的各个方面。由于多核处理器的普及,了解如何编写多线程应用程序变得非常重要。专业C++开发者需要知道如何编写正确的多线程代码,以便充分利用所有可用的处理器单元。

多线程编程是一个复杂的话题。本章介绍了使用标准线程库进行多线程编程,但由于篇幅限制,无法涵盖所有细节。有关多线程程序开发的完整书籍已经有很多了。如果对细节感兴趣,请参阅附录B中的多线程部分。

还有一些第三方C++库试图使多线程编程更加平台独立,例如pthreads和boost::thread库。然而,由于这些库不属于C++标准,因此不在本书中讨论。






