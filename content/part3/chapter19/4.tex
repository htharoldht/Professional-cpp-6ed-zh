
C++标准库提供了std::function和move\_only\_function。两者都是多态函数包装器,是能够包装任何可调用对象(如函数、函数对象或Lambda表达式)的函数对象;后者将在本章后面讨论。

\mySubsubsection{19.4.1}{std::function}

std::function 定义在 <functional> 头文件中。std::function 的实例可以用作函数指针,或者用作实现回调的函数参数,并且可以存储、复制、移动,当然也可以执行。函数模板的模板参数与其他模板参数看起来有点不同:

\begin{cpp}
std::function<R(ArgTypes...)>
\end{cpp}

R 是函数的返回类型,ArgTypes 是函数参数类型的逗号分隔列表。

以下示例演示了如何使用 std::function 实现 函数指针,创建了一个名为 f1 的函数指针,指向 func() 函数。定义 f1 后,可以使用它来调用 func():

\begin{cpp}
void func(int num, string_view str) { println("func({}, {})", num, str); }

int main()
{
    function<void(int, string_view)> f1 { func };
    f1(1, "test");
}
\end{cpp}

function<R(ArgTypes...)> 可以存储一个参数列表与 ArgTypes 完全匹配且返回类型为 R 的函数。也可以存储其他具有参数列表的函数,该列表允许使用 ArgTypes 参数集调用,并返回可以转换为 R 类型的类型。上一个示例中的 func() 函数可以接受其第一个参数为 const int\&,而 main() 中的代码不需要更改:

\begin{cpp}
void func(const int& num, string_view str) { println("func({}, {})", num, str); }
\end{cpp}

通过类模板参数推导,可以简化 f1 的创建:

\begin{cpp}
function f1 { func };
\end{cpp}

前面的示例中,可以使用 auto 关键字,这样可以省去指定 f1 类型的需要。以下 f1 的定义工作方式相同,并且更短,但编译器推导出的 f1 类型是函数指针,即 void (*f1)(int, string\_view),而不是 std::function:

\begin{cpp}
auto f1 { func };
\end{cpp}

std::function 类型表现为函数指针,所以可以将其传递给接受回调的函数。本章前面定义的原始 findMatches() 实现,定义了两个函数指针的类型别名。这些类型别名可以重写为使用 std::function,而示例中的其他内容保持不变:

\begin{cpp}
// A type alias for a function accepting two integer values,
// returning true if both values are matching, false otherwise.
using Matcher = function<bool(int, int)>;

// A type alias for a function to handle a match. The first
// parameter is the position of the match,
// the second and third are the values that matched.
using MatchHandler = function<void(size_t, int, int)>;
\end{cpp}

推荐 findMatches() 实现使用的是函数模板,而不是函数指针或 std::function。

这些示例中,std::function 并不是非常有用;但当需要将回调存储为类的数据成员时,std::function 真的非常有用。这是本章末尾的一个练习。

\mySubsubsection{19.4.2}{std::move\_only\_function}

\CXXTwentythreeLogo{-40}{15}

std::function 要求存储在其中的可调用对象是可复制的。为了缓解这个问题,C++23 引入了std::move\_only\_function 包装器,也在 <functional> 中定义,可用于包装只移动的可调用对象。此外,move\_only\_function 函数对象允许显式地将其函数调用操作符标记为 const 和/或 noexcept。这在 std::function 中是不可能实现的,因为函数调用操作符总是 const。

以下代码演示了 move\_only\_function。假设 BigData 类存储了大量数据,BigDataProcessor 函数对象处理 BigData 实例中的数据。为了避免复制,这个函数对象存储了一个 unique\_ptr 指向 BigData 实例,通过其构造函数获得。为了演示,函数调用的操作符标记为 const,并且简单地打印出一些文本。main() 函数首先创建一个 BigData 实例的 unique\_ptr,创建const processor,最后在 processor 上调用函数调用操作符。此示例中使用 function 的话是行不通的,而BigDataProcessor 是一个只移动类型。

\begin{cpp}
class BigData {};
class BigDataProcessor
{
    public:
        explicit BigDataProcessor(unique_ptr<BigData> data)
            : m_data { move(data) } { }
        void operator()() const { println("Processing BigData data..."); }
    private:
     unique_ptr<BigData> m_data;
};

int main()
{
    auto data { make_unique<BigData>() };
    const move_only_function<void() const> processor {
        BigDataProcessor { move(data) } };
    processor();
}
\end{cpp}





























