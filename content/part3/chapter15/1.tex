
第1章中解释了C++中的操作符,如+、<、*和<{}<,可作用于内置类型,如int和double,允许执行算术、逻辑和其他操作。还有如->和*这样的操作符,对指针解引用。C++中的操作符概念很广泛,甚至包括[](数组索引)、()(函数调用)、类型转换以及内存分配和回收操作符。操作符重载可以改变语言操作符的行为。然而,这种能力也有自己的规则、限制和选择。

\mySubsubsection{15.1.1.}{为什么要重载操作符?}

学习如何重载操作符之前,可能想知道为什么你想要这样做。因不同的操作符而异,但一般的指导原则是让类表现得像内置类型。类越接近内置类型,客户端使用起来就越容易。例如,写一个类来表示分数,能够定义+、-、*和/在应用于该类的对象时就会很有用。

重载操作符的另一个原因是,获得对程序行为更大的控制权。重载类内存分配和回收操作符,以精确指定每个新对象的内存分配和回收。

需要强调的是,操作符重载并不一定会让开发工作更容易,其主要目的是让类的用户使用起来更容易。

\mySubsubsection{15.1.2.}{操作符重载的限制}

以下是在重载操作符时不能做的事情:

\begin{itemize}
\item
不能添加新的操作符符号,只能重新定义语言中已有的操作符的含义。本章后面“可重载操作符总结”一节中的表格,列出了可以重载的操作符。

\item
有些操作符不能重载,比如.和.*(对象中的成员访问)、::(范围解析操作符)和?:(条件操作符)。

\item
元数描述了与操作符相关联的参数或操作数的数量。只能更改函数调用、new和delete操作符的元数。自C++23起,还可以更改下标操作符(数组索引)[]的元数。对于所有其他操作符,不能更改元数。一元操作符,如++,只作用于一个操作数。二元操作符,如/,作用于两个操作数。

\item
不能改变操作符的优先级和结合性。优先级用于决定哪些操作符需要在其他操作符之前执行,而结合性可以是左到右或右到左,并指定具有相同优先级的操作符的执行顺序。这个约束在大多数程序中不应该是关注焦点,很有有改变求值顺序的需求,但在某些领域中,这是需要考虑的。例如,正在编写一个表示数学向量的类,并希望重载\^{}操作符以便能够将向量提升到一定的幂,那么请记住\^{}与许多其他操作符(如+)相比具有较低的优先级。例如,x和y是数学向量,编写x\^{}3+y将计算$x^{3+y}$,而不是$(x^3)+y$。

\item
不能为内置类型重新定义操作符。操作符必须是类中的成员函数,或者至少全局重载操作符函数的数必须是用户定义类型。所以不能做些荒谬的事情,比如将+对int的定义改为减法(尽管可以在自己的类中这样做)。这个规则的例外是内存分配和回收操作符,可以替换程序中所有内存分配的全局操作符。
\end{itemize}

有些操作符已经有两个不同的含义。operator-可以用作二元操作符(如在x=y-z;中)或一元操作符(如在x=-y;中)。operator*可以用作乘法或对指针解引用,<{}<操作符是流插入操作符或左移操作符,具体取决于上下文。对于具有双重含义的操作符,可以两种含义都重载。

\mySubsubsection{15.1.3.}{选择操作符重载}

重载操作符时,需要编写一个名为operatorX的全局函数或成员函数,其中X是某个操作符的符号,operator和X之间可以有可选的空格。第9章为SpreadsheetCell对象声明operator+:

\begin{cpp}
SpreadsheetCell operator+(const SpreadsheetCell& lhs, const SpreadsheetCell& rhs);
\end{cpp}

以下各节描述了在编写每个重载操作符时,涉及到的选择。

\mySamllsection{成员函数或全局函数}

需要决定操作符应该是类的成员函数,还是全局函数。后者可以是类的友元函数,但应作为最后的手段——应尽可能限制向类添加友元函数,它们可以直接访问private数据成员,从而绕过数据隐藏原则。

如何选择成员函数和全局函数?首先,需要了解这两种选择的区别。当操作符是类的成员函数时,操作符表达式的左侧必须始终是该类的对象。如果编写一个全局函数,左侧可以是不同类型的对象。

有三种不同类型的操作符:

\begin{itemize}
\item
必须作为成员函数的操作符。在类的外部没有意义,C++语言要求某些操作符必须是类的成员函数。operator=与类紧密相关,不能存在于其他地方。本节“可重载操作符的总结”中的表格列出了必须作为成员函数的操作符,大多数操作符没有这个要求。

\item
必须作为全局函数的操作符。需要允许操作符左侧的变量与类不同的类型时,必须使操作符成为全局函数。这条规则特别适用于<{}<和>{>}插入和提取流操作符,其中左侧是iostream对象,而不是类对象。这也适用于二元+和-这样的交换操作符,允许左侧的变量不是类的对象。如果希望为二元操作符的左操作数进行隐式转换,则需要全局函数。第9章已经介绍了这个问题。

\item
可以作为成员函数或全局函数的操作符。C++社区中,关于是否最好将操作符重载为成员函数或全局函数存在一些分歧。我建议,除非必须按照前面的描述将其作为全局函数,否则使每个操作符成为成员函数。

这个规则的主要优点是,成员函数可以是虚函数,而全局函数显然不能。在继承树中编写重载操作符时,应使它们成为成员函数。
\end{itemize}

将重载操作符创建为成员函数时,如果不改变对象,应将其标记为const。

将重载操作符创建为全局函数时,需要将其放在包含操作符类的同一命名空间中。

\mySamllsection{选择参数类型}

选择参数类型时,会受到一定限制。对于大多数操作符,不能更改参数的数量。operator/ 是一个全局函数,必须始终有两个参数;如果它是一个成员函数,则必须有一个参数。如果与此标准不符,编译器将报错。操作符函数与普通函数不同,后者可以通过任意数量的参数进行重载。此外,可以为相应的类型编写操作符,但选择会受到操作符类型的限制。例如,想为类 T 实现 addition,不能有一个接受两个字符串的 operator+!当尝试确定是按值,还是按引用传递参数,以及是否使它们成为 const 时,真正需要做出选择的时刻出现了。

按值与引用的选择很简单:除非函数始终复制传递的对象,否则应该通过引用传递每个非原始参数类型,请参见第9章。

const 的决定也很简单:除非需要修改它,否则请将每个参数标记为 const。“可重载操作符摘要”部分中的表格显示了每个操作符的示例原型,其中参数都标记为 const 和引用。

\mySamllsection{选择返回类型}

C++ 不基于返回类型确定重载解析,重载操作符时,可以指定想要的返回类型,但并不意味着应该这样做。这种灵活性会实现一些令人困惑的代码,例如:比较操作符返回指针,算术操作符返回 bool。相反,应该使重载的操作符返回与内置类型的操作符相同的类型。如果重载了比较操作符,请返回一个 bool。如果重载了算术操作符,请返回一个表示结果的对象。有时,返回类型一开始并不明显。例如,第8章提到的operator= 应返回对其调用对象的引用,以支持链式赋值。其他操作符也有类似的复杂返回类型,这些都在后面的表格中进行了总结。

引用和 const 的选择同样适用于返回类型,但对于返回值,选择更为困难。值或引用的一般规则是:如果能返回引用,则返回引用;否则,返回值。如何知道何时可以返回引用?这个选择仅适用于返回对象的操作符:对于返回 bool 的比较操作符、没有返回类型的转换操作符,以及可能返回类型的函数调用操作符,那么这个选择就无关紧要了。如果操作符构造了一个新对象,则必须按值返回新对象。如果没有构造新对象,可以直接返回对调用该操作符的对象或其参数的引用。

可以作为左值(例如,在赋值表达式的左侧)修改的返回值必须是非 const;否则,应该是 const。有很多的操作符都要求返回左值,包括所有赋值操作符(operator=、operator+=、operator-= 等)。

\mySamllsection{选择行为}

可以在重载操作符中提供想要的实现。例如,编写一个启动 Scrabble 游戏的 operator+,通常应将实现限制为客户端期望的行为。编写 operator+ 以执行加法,或者类似加法的行为,如字符串连接。本章将解释应该如何实现重载操作符。特殊情况下,可以不按照建议来做;但一般来说,应该遵循标准模式。

\mySubsubsection{15.1.4.}{不应重载的操作符}

尽管允许重载,但有些操作符不应重载。特别是地址操作符(operator\&),重载它并没有太大用处,反而可能导致混淆,可能会以潜在不可预期的方式改变基本语言行为(获取变量的地址)。整个标准库中,虽然广泛使用操作符重载,但从未重载过地址操作符。

此外,应避免重载二元布尔操作符 operator\&\& 和 ||,因为这样会失去 C++ 的短路求值规则,因为所有操作数都需要在传递给重载操作符函数之前进行计算。如果类需要逻辑操作符,请提供 operator\& 和 |。

最后,不应重载逗号操作符(operator,)。是的,没看错:C++ 中确实有一个逗号操作符,也称为序列点操作符,用于在单个语句中分隔两个表达式,同时保证它们从左到右进行计算。以下代码段演示了逗号操作符的用法:

\begin{cpp}
int x { 1 };
println("{}", (++x, 2 * x)); // Increments x to 2, doubles it, and prints 4.
\end{cpp}

通常,无需重载逗号操作符。

\mySubsubsection{15.1.5.}{可重载操作符的总结}

下表列出了可以重载的操作符,指明了它们应该是类的成员函数,还是全局函数,总结了何时应该(或不应该)重载,并提供了示例,展示了正确的参数和返回值类型。不能重载的操作符,如 ., .*, ::, 和 ?: 不在此列表中。

想要重载操作符时,这个表格是一个有用的参考。可能会忘记应该使用哪种返回类型,以及该函数是否应该是成员函数。

表中T 是编写重载操作符的类的名称,E 是不同的类型。给出的示例并不详尽;通常对于给定的操作符,T 和 E 之间可能有其他组合:

% Please add the following required packages to your document preamble:
% \usepackage[normalem]{ulem}
% \useunder{\uline}{\ul}{}
% \usepackage{longtable}
% Note: It may be necessary to compile the document several times to get a multi-page table to line up properly
\begin{longtable}{|l|l|l|l|l|}
\hline
\textbf{操作符} &
\textbf{\begin{tabular}[c]{@{}l@{}}名称类别\end{tabular}} &
\textbf{\begin{tabular}[c]{@{}l@{}}成员函数\\全局函数\end{tabular}} &
\textbf{\begin{tabular}[c]{@{}l@{}}何时重载\end{tabular}} &
\textbf{\begin{tabular}[c]{@{}l@{}}示例\end{tabular}} \\ \hline
\endfirsthead
%
\endhead
%
\begin{tabular}[c]{@{}l@{}}operator+\\ operator-\\ operator*\\ operator/\\ operator\%\end{tabular} &
\begin{tabular}[c]{@{}l@{}}二元算术\\操作符\end{tabular} &
\begin{tabular}[c]{@{}l@{}}推荐为\\全局函数\end{tabular} &
\begin{tabular}[c]{@{}l@{}}想为类提供\\这些操作,\\就可以重载\\二元算术\\操作符。
\end{tabular} &
\begin{tabular}[c]{@{}l@{}}T operator+(\\const T\&, const T\&);\\ T operator+(\\const T\&, const E\&);\end{tabular} \\ \hline
\begin{tabular}[c]{@{}l@{}}operator-\\ operator+\\ operator˜\end{tabular} &
\begin{tabular}[c]{@{}l@{}}一元算术\\和位操作符\end{tabular} &
\begin{tabular}[c]{@{}l@{}}推荐为\\成员函数\end{tabular} &
\begin{tabular}[c]{@{}l@{}}想为类提供\\这些操作,\\就可以重载\\一元算术和位\\操作符。
\end{tabular} &
T operator-() const; \\ \hline
\begin{tabular}[c]{@{}l@{}}operator++\\ operator-{}-\end{tabular} &
\begin{tabular}[c]{@{}l@{}}前置增量\\和前置减量\\操作符\end{tabular} &
\begin{tabular}[c]{@{}l@{}}推荐为\\成员函数\end{tabular} &
\begin{tabular}[c]{@{}l@{}}重载有\\算术参数\\的+=和-=\\(int, \\long,\\…)\end{tabular} &
T\& operator++(); \\ \hline
\begin{tabular}[c]{@{}l@{}}operator++\\ operator-{}-\end{tabular} &
\begin{tabular}[c]{@{}l@{}}后置增量\\和后置减量\\操作符\end{tabular} &
\begin{tabular}[c]{@{}l@{}}推荐为\\成员函数\end{tabular} &
\begin{tabular}[c]{@{}l@{}}重载有\\算术参数\\的+=和-=时\\(int, \\long,\\…)\end{tabular} &
T operator++(int); \\ \hline
operator= &
\begin{tabular}[c]{@{}l@{}}赋值操作符\end{tabular} &
\begin{tabular}[c]{@{}l@{}}必须为\\成员函数\end{tabular} &
\begin{tabular}[c]{@{}l@{}}当类具有动态\\分配的资源或\\引用成员时\end{tabular} &
\begin{tabular}[c]{@{}l@{}}T\& operator=(const T\&);\end{tabular} \\ \hline
\begin{tabular}[c]{@{}l@{}}operator+=\\ operator-=\\ operator*=\\ operator/=\\ operator\%=\end{tabular} &
\begin{tabular}[c]{@{}l@{}}简写/复合\\算术赋值\\操作符\end{tabular} &
\begin{tabular}[c]{@{}l@{}}推荐为\\成员函数\end{tabular} &
\begin{tabular}[c]{@{}l@{}}重载二元算\\术操作符时,\\类不可变\end{tabular} &
\begin{tabular}[c]{@{}l@{}}T\& operator+=(const T\&);\\ T\& operator+=(const E\&);\end{tabular} \\ \hline
\begin{tabular}[c]{@{}l@{}}operator\textless{}\textless\\ operator\textgreater{}\textgreater\\ operator\&\\ operator|\\ operator\textasciicircum{}\end{tabular} &
\begin{tabular}[c]{@{}l@{}}二元位操作符\end{tabular} &
\begin{tabular}[c]{@{}l@{}}推荐为\\全局函数\end{tabular} &
\begin{tabular}[c]{@{}l@{}}只要想,\\就可以重载\end{tabular} &
\begin{tabular}[c]{@{}l@{}}T operator\textless{}\textless{}(\\const T\&, const T\&);\\ T operator\textless{}\textless{}(\\const T\&, const E\&);\end{tabular} \\ \hline
\begin{tabular}[c]{@{}l@{}}operator\textless{}\textless{}=\\ operator\textgreater{}\textgreater{}=\\ operator\&=\\ operator|=\\ operator\textasciicircum{}=\end{tabular} &
\begin{tabular}[c]{@{}l@{}}简写/复合\\位赋值操作符\end{tabular} &
\begin{tabular}[c]{@{}l@{}}推荐为\\成员函数\end{tabular} &
\begin{tabular}[c]{@{}l@{}}重载二元算\\术操作符时,\\类不可变\end{tabular} &
\begin{tabular}[c]{@{}l@{}}T\& operator\textless{}\textless{}=(const T\&);\\ T\& operator\textless{}\textless{}=(const E\&);\end{tabular} \\ \hline
operator\textless{}=\textgreater{} &
\begin{tabular}[c]{@{}l@{}}三向比较\\操作符\end{tabular} &
\begin{tabular}[c]{@{}l@{}}推荐为\\成员函数\end{tabular} &
\begin{tabular}[c]{@{}l@{}}当为类提\\供比较支持\\时;如果\\可能的话,\\应该使用\\=default\\作为默认值\end{tabular} &
\begin{tabular}[c]{@{}l@{}}auto\\ operator\textless{}=\textgreater{}(\\const T\&) const = default;\\ partial\_ordering\\ operator\textless{}=\textgreater{}(\\const E\&) const;\end{tabular} \\ \hline
operator== &
\begin{tabular}[c]{@{}l@{}}二元相等\\操作符\end{tabular} &
\begin{tabular}[c]{@{}l@{}}C++20后:\\推荐为\\成员函数\\C++ 20前:\\推荐为\\全局函数\end{tabular} &
\begin{tabular}[c]{@{}l@{}}当为类提\\供比较支\\持,并且\\不能默认\\使用三向\\比较操作\\符时\end{tabular} &
\begin{tabular}[c]{@{}l@{}}bool operator==(\\const T\&) const;\\ bool operator==(\\const E\&) const;\\ bool operator==(\\const T\&, const T\&);\\ bool operator==(\\const T\&, const E\&);\end{tabular} \\ \hline
operator!= &
\begin{tabular}[c]{@{}l@{}}二元不等\\操作符\end{tabular} &
\begin{tabular}[c]{@{}l@{}}C++20后:\\推荐为\\成员函数\\C++ 20前:\\推荐为\\全局函数\end{tabular} &
\begin{tabular}[c]{@{}l@{}}C++20后:\\不需要,\\因为编译\\器在支持\\==时自动\\提供!=;\\ C++ 20前:\\当为类提\\供比较支\\持时\end{tabular} &
\begin{tabular}[c]{@{}l@{}}bool operator!=(\\const T\&) const;\\ bool operator!=(\\const E\&) const;\\ bool operator!=(\\const T\&, const T\&);\\ bool operator!=(\\const T\&, const E\&);\end{tabular} \\ \hline
\begin{tabular}[c]{@{}l@{}}operator\textless\\ operator\textgreater\\ operator\textless{}=\\ operator\textgreater{}=\end{tabular} &
\begin{tabular}[c]{@{}l@{}}二元比较\\操作符\end{tabular} &
\begin{tabular}[c]{@{}l@{}}推荐为\\全局函数\end{tabular} &
\begin{tabular}[c]{@{}l@{}}想提供这\\些操作时;\\不需要<=>\\时\end{tabular} &
\begin{tabular}[c]{@{}l@{}}bool operator\textless{}(\\const T\&, const T\&);\\ bool operator\textless{}(\\const T\&, const E\&);\end{tabular} \\ \hline
\begin{tabular}[c]{@{}l@{}}operator\textless{}\textless\\ operator\textgreater{}\textgreater{}\end{tabular} &
\begin{tabular}[c]{@{}l@{}}I/O流\\操作符\\(插入和提取)\end{tabular} &
\begin{tabular}[c]{@{}l@{}}必须为\\全局函数\end{tabular} &
\begin{tabular}[c]{@{}l@{}}想要提供\\这些操作时\end{tabular} &
\begin{tabular}[c]{@{}l@{}}ostream\&\\ operator\textless{}\textless{}(\\ostream\&, const T\&);\\ istream\&\\ operator\textgreater{}\textgreater{}(\\istream\&, T\&);\end{tabular} \\ \hline
operator! &
\begin{tabular}[c]{@{}l@{}}布尔操作符\end{tabular} &
\begin{tabular}[c]{@{}l@{}}推荐为\\成员函数\end{tabular} &
\begin{tabular}[c]{@{}l@{}}很少重载;\\请使用bool\\或void*转换\end{tabular} &
\begin{tabular}[c]{@{}l@{}}bool operator!() const;\end{tabular} \\ \hline
\begin{tabular}[c]{@{}l@{}}operator\&\&\\ operator||\end{tabular} &
\begin{tabular}[c]{@{}l@{}}二元布尔\\操作符\end{tabular} &
\begin{tabular}[c]{@{}l@{}}推荐为\\全局函数\end{tabular} &
\begin{tabular}[c]{@{}l@{}}很少重载,\\会失去短\\路特性;\\最好重载\\\&和|,\\它们永远\\不会短路\end{tabular} &
\begin{tabular}[c]{@{}l@{}}bool operator\&\&(\\const T\&, const T\&);\end{tabular} \\ \hline
operator{[}{]} &
\begin{tabular}[c]{@{}l@{}}下标\\(数组索引)\\操作符\end{tabular} &
\begin{tabular}[c]{@{}l@{}}必须为\\成员函数\end{tabular} &
\begin{tabular}[c]{@{}l@{}}想支持下\\标时\end{tabular} &
\begin{tabular}[c]{@{}l@{}}E\& operator{[}{]} (size\_t);\\ const E\& operator{[}{]} (\\size\_t) const;\end{tabular} \\ \hline
operator() &
\begin{tabular}[c]{@{}l@{}}函数操作符\end{tabular} &
\begin{tabular}[c]{@{}l@{}}必须为\\成员函数\end{tabular} &
\begin{tabular}[c]{@{}l@{}}想让对象\\表现得像\\函数指针时\end{tabular} &
\begin{tabular}[c]{@{}l@{}}返回类型和参数可以不同\end{tabular} \\ \hline
\begin{tabular}[c]{@{}l@{}}operator \\type() \end{tabular}&
\begin{tabular}[c]{@{}l@{}}转换或强制\\转换操作符\\(类型独立\\的操作符)\end{tabular} &
\begin{tabular}[c]{@{}l@{}}必须为\\成员函数\end{tabular} &
\begin{tabular}[c]{@{}l@{}}想进行类\\型转换时\end{tabular} &
\begin{tabular}[c]{@{}l@{}}operator double() const;\end{tabular} \\ \hline
\begin{tabular}[c]{@{}l@{}}operator \\""\_x \end{tabular}&
\begin{tabular}[c]{@{}l@{}}自定义\\字面量\\操作符\end{tabular} &
\begin{tabular}[c]{@{}l@{}}必须为\\全局函数\end{tabular} &
\begin{tabular}[c]{@{}l@{}}希望支持\\自定义字\\面量时\end{tabular} &
\begin{tabular}[c]{@{}l@{}}T operator""\_i(\\long double d);\end{tabular} \\ \hline
\begin{tabular}[c]{@{}l@{}}operator \\new\\ operator \\new{[}{]}\end{tabular} &
\begin{tabular}[c]{@{}l@{}}内存分配\end{tabular} &
\begin{tabular}[c]{@{}l@{}}建议为\\成员函数\end{tabular} &
\begin{tabular}[c]{@{}l@{}}想要控制\\类的内存\\分配时\\(很少用)\end{tabular} &
\begin{tabular}[c]{@{}l@{}}void* operator new(\\size\_t size);\\ void* operator new{[}{]} (\\size\_t size);\end{tabular} \\ \hline
\begin{tabular}[c]{@{}l@{}}operator \\delete\\ operator \\delete{[}{]}\end{tabular} &
\begin{tabular}[c]{@{}l@{}}内存释放\end{tabular} &
\begin{tabular}[c]{@{}l@{}}建议为\\成员函数\end{tabular} &
\begin{tabular}[c]{@{}l@{}}想要控制\\类的内存\\分配时\\(很少用)\end{tabular} &
\begin{tabular}[c]{@{}l@{}}void operator delete(\\void* ptr) noexcept;\\ void operator delete{[}{]}(\\void* ptr) noexcept;\end{tabular} \\ \hline
\begin{tabular}[c]{@{}l@{}}operator*\\ operator-\textgreater{}\end{tabular} &
\begin{tabular}[c]{@{}l@{}}解引用\\操作符\end{tabular} &
\begin{tabular}[c]{@{}l@{}}operator*:\\建议为\\成员函数\\ operator-\textgreater{}:\\必须为\\成员函数\end{tabular} &
\begin{tabular}[c]{@{}l@{}}对智能指\\针有用\end{tabular} &
\begin{tabular}[c]{@{}l@{}}E\& operator*() const;\\ E* operator-\textgreater{}() const;\end{tabular} \\ \hline
operator\& &
\begin{tabular}[c]{@{}l@{}}取地址\\操作符\end{tabular} &
N/A &
从不 &
N/A \\ \hline
operator-\textgreater{}* &
\begin{tabular}[c]{@{}l@{}}解引用\\指向成员\\的指针\end{tabular} &
N/A &
从不 &
N/A \\ \hline
operator, &
\begin{tabular}[c]{@{}l@{}}逗号操作符\end{tabular} &
N/A &
从不 &
N/A \\ \hline
\end{longtable}

\mySubsubsection{15.1.6.}{右值引用}

第9章讨论了移动语义和右值引用,通过定义移动赋值操作符来展示这些概念,编译器在源对象是一个临时对象且在赋值后会销毁,或者使用 std::move() 明确从某个对象移动时,会使用这些移动赋值操作符。前一个表中的普通赋值操作符有以下原型:

\begin{cpp}
T& operator=(const T&);
\end{cpp}

移动赋值操作符有相同原型,但它使用右值引用,修改了参数,使其不能作为 const 传递。有关详细信息,请参阅第9章。

\begin{cpp}
T& operator=(T&&) noexcept;
\end{cpp}

前一个表没有包含使用右值引用的示例原型,对于大多数操作符来说,使用左值引用的版本和使用右值引用的版本都可以,但是否合理则取决于类的实现细节。operator= 是第9章的示例,另一个示例是 operator+,以避免不必要的内存分配。标准库中的 std::string 类实现了使用右值引用的 operator+,如下所示(简化过):

\begin{cpp}
string operator+(string&& lhs, string&& rhs);
\end{cpp}

这个操作符的实现重用了其中一个参数的内存,作为右值引用传递,所以两者都是临时对象,当 operator+ 完成时销毁。根据两个操作数的大小和容量, operator+ 的实现具有以下效果:

\begin{cpp}
return move(lhs.append(rhs));
\end{cpp}

或者

\begin{cpp}
return move(rhs.insert(0, lhs));
\end{cpp}

实际上,string 定义了几个 operator+ 重载,接受两个字符串作为参数,以及不同的左值和右值引用组合。以下是列表(简化过):

\begin{cpp}
string operator+(const string& lhs, const string& rhs); // No memory reuse.
string operator+(string&& lhs, const string& rhs); // Can reuse memory of lhs.
string operator+(const string& lhs, string&& rhs); // Can reuse memory of rhs.
string operator+(string&& lhs, string&& rhs); // Can reuse memory of lhs or rhs.
\end{cpp}

重用一个右值引用参数的内存的实现方式与第9章中的移动赋值操作符相同。

\mySubsubsection{15.1.7.}{优先级和结合性}

包含多个操作符的语句中,操作符的优先级用于决定,哪些操作符需要先于其他操作符执行。例如,* 和 / 总是在 + 和 - 之前执行。

结合性可以是左到右或右到左,指定了具有相同优先级的操作符的执行顺序。

下表列出了所有可用 C++ 操作符的优先级和结合性,包括那些不能重载的操作符,以及本书中尚未提及的操作符。具有较低优先级编号的操作符,将在具有较高优先级编号的操作符之前执行。表中,T 表示类型,而 x、y 和 z 表示对象:

% Please add the following required packages to your document preamble:
% \usepackage{longtable}
% Note: It may be necessary to compile the document several times to get a multi-page table to line up properly
\begin{longtable}{|l|l|l|}
\hline
\textbf{优先级} & \textbf{操作符}                                   & \textbf{结合性} \\ \hline
\endfirsthead
%
\endhead
%
1                   & ::                                                  & 从左到右          \\ \hline
2                   & x++ \quad x-- \quad x() \quad x{[}{]} \quad T() \quad T\{\} . \quad -\textgreater{}     & 从左到右          \\ \hline
3 &
\begin{tabular}[c]{@{}l@{}}++x \quad --x \quad +x \quad -x \quad ! \quad \textasciitilde \quad *x \quad \quad \&x \quad (T)\\ sizeof \quad co\_await \quad new \quad delete \quad new{[}{]} \quad delete{[}{]}\end{tabular} &
从右到左 \\ \hline
4                   & .* \quad -\textgreater{}*                                 & 从左到右          \\ \hline
5                   & x*y \quad x/y \quad x\%y                                        & 从左到右          \\ \hline
6                   & x+y \quad x-y                                             & 从左到右          \\ \hline
7                   & \textless{}\textless \quad \textgreater{}\textgreater{}   & 从左到右          \\ \hline
8                   & \textless{}=\textgreater{}                          & 从左到右          \\ \hline
9                   & \textless{}\textless{}= \quad \textgreater{}\textgreater{}= & 从左到右          \\ \hline
10                  & == \quad !=                                               & 从左到右          \\ \hline
11                  & x\&y                                                & 从左到右          \\ \hline
12                  & \textasciicircum{}                                  & 从左到右          \\ \hline
13                  & |                                                   & 从左到右          \\ \hline
14                  & \&\&                                                & 从左到右          \\ \hline
15                  & ||                                                  & 从左到右          \\ \hline
16 &
\begin{tabular}[c]{@{}l@{}}x?y:z \quad throw \quad co\_yield\\ = \quad += \quad -= \quad *= \quad /= \quad \%= \quad \textless{}\textless{}= \quad \textgreater{}\textgreater{}= \quad \&= \quad \textasciicircum{}= \quad |=\end{tabular} &
从右到左 \\ \hline
17                  & ,                                                   & 从左到右          \\ \hline
\end{longtable}

\mySubsubsection{15.1.8.}{关系操作符}

<utility>头文件中的 std::rel\_ops 命名空间中定义了一组关系操作符的函数模板:

\begin{cpp}
template<class T> bool operator!=(const T& a, const T& b);// Needs operator==
template<class T> bool operator>(const T& a, const T& b); // Needs operator<
template<class T> bool operator<=(const T& a, const T& b);// Needs operator<
template<class T> bool operator>=(const T& a, const T& b);// Needs operator<
\end{cpp}

这些函数模板使用 == 和 < 操作符定义了 !=, >, <=, 和 >= 操作符,适用于任何类。可以为类实现 operator== 和 < 后,就可以获得其他关系操作符了。

然而,这种技术存在很多问题。第一个问题是,这些操作符可能会在关系操作中的所有类创建。

第二个问题是,像 std::greater<T> 这样的工具模板(第19章)与自动生成的这些关系操作符不兼容。

另一个问题是,无法隐式转换。

最后,由于 C++20 的三向比较操作符,以及 C++20 已经弃用了 std::rel\_ops 命名空间,因此没有理由再使用 rel\_ops 了。

\begin{myWarning}{WARNING}
永远不要使用 std::rel\_ops, C++20 开始已经将其弃用!要为一个类添加对所有六个比较操作符的支持,只需显式默认或实现 operator<=>,并可能为该类实现 operator==。详情请参见第9章。
\end{myWarning}

\mySubsubsection{15.1.9.}{替代表示法}

C++ 支持以下替代表示法供选择的操作符使用。这些主要在旧代码中使用,当时使用的字符集不包括某些字符,如 $\sim$、| 和 \^{}。

% Please add the following required packages to your document preamble:
% \usepackage{longtable}
% Note: It may be necessary to compile the document several times to get a multi-page table to line up properly
\begin{longtable}{|l|l|l|l|l|}
\hline
\textbf{操作符} & \textbf{替代符号} && \textbf{操作符} & \textbf{替代符号} \\ \hline
\endfirsthead
%
\endhead
%
\&\&   & and     && !=                  & not\_eq \\ \hline
\&=    & and\_eq && ||                  & or      \\ \hline
\&     & bitand  && |=                  & or\_eq  \\ \hline
|      & bitor   && \textasciicircum{}  & xor     \\ \hline
$\sim$ & compl   && \textasciicircum{}= & xor\_eq \\ \hline
!     & not     &&                     &         \\ \hline
\end{longtable}



