
C++ 中可以重载函数调用操作符,写为 operator()。如果为类实现了一个 operator(),就可以像使用函数指针一样使用该类的对象。具有函数调用操作符的对象称为函数对象,或简称为函子。下面是一个简单的类,有一个重载的函数调用操作符和一个具有相同行为的类成员函数的例子:

\begin{cpp}
class Squarer
{
    public:
        int operator()(int value) const; // Overloaded function call operator.
        int doSquare(int value) const; // Normal member function.
};
// Implementation of overloaded function call operator.
int Squarer::operator()(int value) const { return doSquare(value); }
// Implementation of normal member function.
int Squarer::doSquare(int value) const { return value * value; }
\end{cpp}

下面是使用函数调用操作符的代码示例,与调用类的正常成员函数进行对比:

\begin{cpp}
int x { 3 };
Squarer square;
int xSquared { square(x) }; // Call the function call operator.
int xSquaredAgain { square.doSquare(xSquared) };// Call the normal member function.
println("{} squared is {}, and squared again is {}.", x, xSquared, xSquaredAgain);
\end{cpp}

输出为:

\begin{shell}
3 squared is 9, and squared again is 81.
\end{shell}

起初,函数调用操作符看起来有点奇怪。为什么会为一个类写一个特殊的成员函数,使类的对象看起来像函数指针?为什么不只是写一个全局函数或类的标准成员函数?

函数对象相对于类标准成员函数的优势很简单:这些对象有时可以伪装成函数指针,可以将函数对象作为回调函数传递给其他函数,这将在第19章中详细介绍。

函数对象相对于全局函数的优势更为复杂。有两个主要优势:

\begin{itemize}
\item
重复调用其函数调用操作符时,对象可以在其数据成员之间保留信息。例如,一个函数对象可用来保持每次调用函数调用操作符时,收集数字的总和。

\item
可以通过设置数据成员来定制函数对象的行为,可以写一个函数对象来比较函数调用操作符的参数与数据成员。这个数据成员也许可配置,这样对象就可以定制化比较了。
\end{itemize}

当然,可以用全局或静态变量来实现上述优点,但函数对象提供了一种更清晰的方式来做这些事情。而且,应该避免使用全局或静态变量,因为它在多线程应用程序中可能会引起问题。函数对象的真正的优势将在第20章中通过标准库展示。

遵循正常的成员函数重载规则,可以为类写尽可能多的 operator()。可以为 Squarer 类添加一个接受双精度的重载:

\begin{cpp}
int operator()(int value) const;
double operator()(double value) const;
\end{cpp}

双精度重载的实现如下所示:

\begin{cpp}
double Squarer::operator()(double value) const { return value * value; }
\end{cpp}

\mySubsubsection{15.6.1}{静态函数调用操作符}

\CXXTwentythreeLogo{-40}{15}

从 C++23 开始,函数调用操作符可以标记为静态,如果其实现不需要访问 this,或者不需要访问非静态数据成员和非静态成员函数。这与本章介绍过的下标号操作符可以标记为静态的方式相似,允许编译器更好地优化代码。

以下是一个例子,一个简化的 Squarer 函数对象,具有static、constexpr 和 noexcept 函数调用操作符:

\begin{cpp}
class Squarer
{
    public:
        static constexpr int operator()(int value) noexcept
        {
            return value * value;
        }
};
\end{cpp}

这个函数对象可以这样使用:

\begin{cpp}
int x { 3 };
int xSquared { Squarer::operator()(x) };
int xSquaredAgain { Squarer{}(xSquared) };
println("{} squared is {}, and squared again is {}.", x, xSquared, xSquaredAgain);
\end{cpp}

静态函数调用操作符的另一个好处是,可以轻松地获取其地址,例如 \&Squarer::operator(),可以像使用函数指针一样使用它们。这可以使用在第20章介绍的标准库算法中,从而提高运行性能。这些算法许多都接受一个可调用对象,如一个函数对象,以定制其行为。如果函数对象有一个静态函数调用操作符,可将该函数调用操作符的地址传递给这些算法,可以让编译器生成比非静态函数调用操作符更高效的代码。因为有了静态函数调用操作符,编译器就可以不管 this 指针了。






















