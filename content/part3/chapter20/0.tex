\noindent
\textbf{内容概要}

\begin{itemize}
\item
标准库算法

\item
标准库提供的算法

\item
并行执行算法以提高性能

\item
约束算法
\end{itemize}

本章的所有代码示例都可以在\url{https://github.com/Professional-CPP/edition-6}获得

第 18 章中,标准库提供了一系列令人印象深刻的通用数据结构。然而,标准库还包含了一系列通用算法,这些算法可以应用于容器的元素(尽管存在一些例外)。使用这些算法,可以对容器中的元素进行查找、排序和处理,并执行一系列其他操作。算法的优雅之处在于,不仅独立于底层元素的类型,也独立于它们操作的容器的类型。算法仅使用迭代器或范围接口,这些接口在第 17 章中介绍过。

标准库附带了一个大型的无约束算法集,所有这些算法仅与迭代器一起工作。这些算法没有与它们相关的概念形式的约束;详见第 12 章。标准库还包含了一个大型的约束算法集,有时称为基于范围的算法。这些算法能够与迭代器和范围一起工作,并且有适当的约束,当算法错误使用时,编译器可以产生更可读的错误消息。本章首先关注无约束算法,这些算法在现有和遗留代码库中使用最多,需要了解它们是如何工作的。当知道了它们是如何工作的,再看约束算法如何使事情变得简单。

第 16 章给出了所有标准库算法的高层次概述,但没有编码细节。结合第 19 章的知识,现在是时候看看这些算法如何在实践中使用,并发挥其真正的力量。
