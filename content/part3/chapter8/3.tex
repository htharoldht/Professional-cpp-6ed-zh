
对象生命周期涉及三个行为:创建、销毁和赋值。了解对象如何以及何时创建、销毁和赋值,以及如何自定义这些行为非常重要。

\mySubsubsection{8.3.1.}{创建}

对象在声明它们(如果在栈上)时创建,或者当使用智能指针、new或new[]显式分配空间时创建。当一个对象创建时,其所有的嵌入对象也会创建。以下是一个例子:

\begin{cpp}
import std;

class MyClass
{
    private:
    std::string m_name;
};

int main()
{
    MyClass obj;
}
\end{cpp}

string对象在main()函数中创建MyClass对象时创建,并在其包含对象销毁时销毁。

声明变量时给出初始值通常很有帮助:

\begin{cpp}
int x { 0 };
\end{cpp}

也可以为对象提供初始值,通过声明并编写一个特殊的成员函数来实现这个功能,这个函数称为构造函数,可以在其中为对象执行初始化工作。每当创建一个对象时,都会执行造函数。

\begin{myNotic}{NOTE}
C++开发者有时将构造函数称为“ctor”(发音为“seetor”)。
\end{myNotic}

\mySamllsection{定义构造函数}

从语法上讲,构造函数通过与类名相同的成员函数名来指定。构造函数没有返回类型,可以有参数,也可以没有参数。一个可以不带参数调用的构造函数称为默认构造函数,可能是一个没有参数的构造函数,或者一个所有参数都有默认值的构造函数。某些情况下,可能需要提供一个默认构造函数,如果没有提供,将收获编译错误。默认构造函数会在本章后面进行讨论。

以下是对SpreadsheetCell类添加构造函数的首次尝试:

\begin{cpp}
export class SpreadsheetCell
{
    public:
        SpreadsheetCell(double initialValue);
        // Remainder of the class definition omitted for brevity
};
\end{cpp}

必须为普通成员函数提供实现一样,也必须为构造函数提供实现:

\begin{cpp}
SpreadsheetCell::SpreadsheetCell(double initialValue)
{
    setValue(initialValue);
}
\end{cpp}

SpreadsheetCell构造函数是SpreadsheetCell类的一个成员,所以C++要求在构造函数名前使用正常的SpreadsheetCell::作用域解析。构造函数名本身也是SpreadsheetCell,所以代码最终呈现出奇怪的SpreadsheetCell::SpreadsheetCell,而实现只调用了setValue()。

\mySamllsection{使用构造函数}

使用构造函数可以创建对象并为其初始化值,可以使用构造函数来进行栈上和堆上的分配。

\mySamllsection{栈上对象的构造函数}

在栈上分配一个SpreadsheetCell对象时,可以像这样使用构造函数:

\begin{cpp}
SpreadsheetCell myCell(5), anotherCell(4);
println("cell 1: {}", myCell.getValue());
println("cell 2: {}", anotherCell.getValue());
\end{cpp}

或者,可以使用统一初始化语法:

\begin{cpp}
SpreadsheetCell myCell { 5 }, anotherCell { 4 };
\end{cpp}

不需要显式调用SpreadsheetCell构造函数。不要这样做:

\begin{cpp}
SpreadsheetCell myCell.SpreadsheetCell(5); // WILL NOT COMPILE!
\end{cpp}

同样,也不能显式调用构造函数:

\begin{cpp}
SpreadsheetCell myCell;
myCell.SpreadsheetCell(5); // WILL NOT COMPILE!
\end{cpp}

\mySamllsection{堆区对象的构造函数}

动态分配SpreadsheetCell对象时,可以像这样使用构造函数:

\begin{cpp}
auto smartCellp { make_unique<SpreadsheetCell>(4) };
// ... do something with the cell, no need to delete the smart pointer

// Or with raw pointers, without smart pointers (not recommended)
SpreadsheetCell* myCellp { new SpreadsheetCell { 5 } };
// Or
// SpreadsheetCell* myCellp{ new SpreadsheetCell(5) };
SpreadsheetCell* anotherCellp { nullptr };
anotherCellp = new SpreadsheetCell { 4 };
// ... do something with the cells
delete myCellp; myCellp = nullptr;
delete anotherCellp; anotherCellp = nullptr;
\end{cpp}

可以声明一个SpreadsheetCell对象的指针而不立即调用构造函数,这与栈上的对象不同,栈上的对象在声明时调用构造函数。

始终初始化指针,要么是正确的指针,要么是nullptr。

\mySamllsection{提供多个构造函数}

可以在一个类中提供多个构造函数。所有构造函数都有相同的名称(类的名称),但不同的构造函数必须接受不同数量或不同类型的参数。如果有多个具有相同名称的函数,编译器会选择参数类型与调用站点匹配的一个。这称为重载,在第9章中有详细讨论。

SpreadsheetCell类中,提供两个构造函数是有帮助的:一个接受初始double值,另一个接受初始string值。以下是类的新定义:

\begin{cpp}
export class SpreadsheetCell
{
    public:
        SpreadsheetCell(double initialValue);
        SpreadsheetCell(std::string_view initialValue);
        // Remainder of the class definition omitted for brevity
};
\end{cpp}

这是第二个构造函数的实现:

\begin{cpp}
SpreadsheetCell::SpreadsheetCell(string_view initialValue)
{
    setString(initialValue);
}
\end{cpp}

以下是使用两个不同构造函数的代码:

\begin{cpp}
SpreadsheetCell aThirdCell { "test" }; // Uses string-arg ctor
SpreadsheetCell aFourthCell { 4.4 }; // Uses double-arg ctor
auto aFifthCellp { make_unique<SpreadsheetCell>("5.5") }; // string-arg ctor
println("aThirdCell: {}", aThirdCell.getValue());
println("aFourthCell: {}", aFourthCell.getValue());
println("aFifthCellp: {}", aFifthCellp->getValue());
\end{cpp}

当有多个构造函数时,可能会试图在构造函数初始化器中实现一个构造函数委托另一个构造函数。例如,想从string构造函数调用double构造函数:

\begin{cpp}
SpreadsheetCell::SpreadsheetCell(string_view initialValue)
{
    SpreadsheetCell(stringToDouble(initialValue));
}
\end{cpp}

这看起来很有道理,可以在其他成员函数中调用正常的类成员函数。代码将编译、链接并运行,但不会做期望的事情!显式调用SpreadsheetCell构造函数实际上创建了一个新的临时无名SpreadsheetCell对象,并没有调用应该初始化对象的构造函数。

然而,C++确实支持委托构造函数。可以在构造函数初始化器中调用同一类的其他构造函数,本章稍后会介绍构造函数初始化器。

\mySamllsection{默认构造函数}

默认构造函数是一个不接受参数的构造函数,也称为无参数构造函数。

\mySamllsection{何时需要默认构造函数}

考虑对象数组。创建对象数组执行两个任务:为所有对象分配连续的内存空间,并在每个对象上调用默认构造函数。C++没有提供语法直接告诉数组创建代码调用不同的构造函数,如果没有为SpreadsheetCell类定义默认构造函数,以下代码将无法编译:

\begin{cpp}
SpreadsheetCell cells[3]; // FAILS compilation without default constructor
SpreadsheetCell* myCellp { new SpreadsheetCell[10] }; // Also FAILS
\end{cpp}

可以通过使用这些初始化器来规避这个限制:

\begin{cpp}
SpreadsheetCell cells[3] { SpreadsheetCell { 0 }, SpreadsheetCell { 23 },
    SpreadsheetCell { 41 } };
\end{cpp}

如果打算创建SpreadsheetCell类的对象数组,通常需要确保类有一个默认构造函数。如果没有定义自己的构造函数,编译器会自动创建一个默认构造函数。这个编译器生成的构造函数回在后面的章节中讨论。

\mySamllsection{如何编写默认构造函数}

这里是带有默认构造函数的SpreadsheetCell类定义的一部分:

\begin{cpp}
export class SpreadsheetCell
{
    public:
        SpreadsheetCell();
        // Remainder of the class definition omitted for brevity
};
\end{cpp}

这是默认构造函数的第一次实现:

\begin{cpp}
SpreadsheetCell::SpreadsheetCell()
{
    m_value = 0;
}
\end{cpp}

如果在类定义中使用了m\_value的内联成员初始化器,则这个默认构造函数中的单个语句可以省略。

\begin{cpp}
SpreadsheetCell::SpreadsheetCell()
{}
\end{cpp}

栈上使用默认构造函数的方式:

\begin{cpp}
SpreadsheetCell myCell;
myCell.setValue(6);
println("cell 1: {}", myCell.getValue());
\end{cpp}

上述代码创建了一个新的SpreadsheetCell对象myCell,设置其值,并打印出其值。与栈上对象的其它构造函数不同,不需要用函数调用语法调用默认构造函数。根据其他构造函数的语法,可能会像这样调用默认构造函数:

\begin{cpp}
SpreadsheetCell myCell(); // WRONG, but will compile.
myCell.setValue(6); // However, this line will not compile.
println("cell 1: {}", myCell.getValue());
\end{cpp}

尝试调用默认构造函数的行可以编译,但后面的代码无法编译。这个问题通常称为“C++最令人烦恼的解析”,编译器认为第一行实际上是具有零个参数并返回SpreadsheetCell对象的函数声明。到达第二行时,认为是在尝试使用一个函数名作为一个对象!

当然,可以使用统一初始化语法,而不是函数调用风格的括号:

\begin{cpp}
SpreadsheetCell myCell { }; // Calls the default constructor.
\end{cpp}

\begin{myWarning}{WARNING}
使用默认构造函数在堆栈上创建对象时,要么使用花括号统一初始化语法,要么省略括号。
\end{myWarning}

对于堆区对象分配,可以使用以下默认构造函数:

\begin{cpp}
auto smartCellp { make_unique<SpreadsheetCell>() };
// Or with a raw pointer (not recommended)
SpreadsheetCell* myCellp { new SpreadsheetCell { } };
// Or
// SpreadsheetCell* myCellp { new SpreadsheetCell };
// Or
// SpreadsheetCell* myCellp { new SpreadsheetCell() };
// ... use myCellp
delete myCellp; myCellp = nullptr;
\end{cpp}

\mySamllsection{编译器生成的默认构造函数}

本章中第一个SpreadsheetCell类定义如下:

\begin{cpp}
export class SpreadsheetCell
{
    public:
        void setValue(double value);
        double getValue() const;
    private:
        double m_value;
};
\end{cpp}

这个定义没有声明默认构造函数,但接下来的代码仍然可以正常工作:

\begin{cpp}
SpreadsheetCell myCell;
myCell.setValue(6);
\end{cpp}

以下定义与前面的定义相同,只是添加了一个显式构造函数,接受一个double。仍然没有显式声明默认构造函数:

\begin{cpp}
export class SpreadsheetCell
{
    public:
        SpreadsheetCell(double initialValue); // No default constructor
        // Remainder of the class definition omitted for brevity
};
\end{cpp}

使用这个定义,以下代码不再编译:

\begin{cpp}
SpreadsheetCell myCell;
myCell.setValue(6);
\end{cpp}

这是怎么回事?原因是没有声明构造函数时,编译器生成了一个不带参数的构造函数。这个编译器生成的默认构造函数,会调用该类所有对象成员的默认构造函数,但不初始化语言原生类型如int和double。尽管如此,它仍然可以创建该类的对象。如果自己声明了构造函数,编译器就不再生成默认构造函数了。

\begin{myNotic}{NOTE}
默认构造函数与零参数构造函数相同。默认构造函数不仅指没有自己声明构造函数,编译器自动生成的构造函数,它还指不需要任何参数的构造函数。
\end{myNotic}

\mySamllsection{显式声明默认构造函数}

C++11之前,如果类需要几个显式接受参数的构造函数,也需要一个默认构造函数,该构造函数什么也不做,仍然必须显式编写自己的空默认构造函数,如之前所示。

为了避免手动编写空默认构造函数,C++支持显式默认的默认构造函数。可以像下面这样编写类定义,而不需要为默认构造函数提供空实现:

\begin{cpp}
export class SpreadsheetCell
{
    public:
        SpreadsheetCell() = default;
        SpreadsheetCell(double initialValue);
        SpreadsheetCell(std::string_view initialValue);
        // Remainder of the class definition omitted for brevity
};
\end{cpp}

SpreadsheetCell定义了两个自定义构造函数,编译器仍然生成一个标准的默认构造函数,因为其中一个使用了default关键字显式地进行了默认设置。可以将=default直接放在类定义中,或者放在实现文件中。

\mySamllsection{显式删除默认构造函数}

与显式默认的默认构造函数相反,也可以显式删除默认构造函数,在某些情况下必要。可以定义一个只有静态成员函数的类(请参见第9章),不想编写构造函数,也不希望编译器生成默认构造函数。这种情况下,需要显式删除默认构造函数。

\begin{cpp}
export class MyClass
{
    public:
        MyClass() = delete;
};
\end{cpp}

\begin{myNotic}{NOTE}
如果类的数据成员具有删除的默认构造函数,则该类的默认构造函数也将自动删除。
\end{myNotic}

\mySamllsection{构造函数初始化器(ctor-initializer)}

目前为止,本章在构造函数体内初始化了数据成员:

\begin{cpp}
SpreadsheetCell::SpreadsheetCell(double initialValue)
{
    setValue(initialValue);
}
\end{cpp}

C++提供了一种替代方法来在构造函数中初始化数据成员,称为构造函数初始化器,也称为ctor-initializer或成员初始化列表。以下是使用ctor-initializer语法的相同SpreadsheetCell构造函数的重写:

\begin{cpp}
SpreadsheetCell::SpreadsheetCell(double initialValue)
: m_value { initialValue }
{}
\end{cpp}

初始化器在构造函数参数列表、构造函数体和括号之间具有语法上的位置。列表以冒号开始,由逗号分隔。列表中的每个元素都是使用函数表示法或统一初始化语法初始化数据成员,对基类构造函数的调用(请参见第10章),或者是对委托构造函数的调用,这在本书本章稍后介绍。

使用初始化器初始化数据成员,与在构造函数体内初始化数据成员的行为不同。当C++创建一个对象时,必须在调用构造函数之前创建该对象的所有数据成员。创建这些数据成员的过程中,必须调用自身的构造函数。在构造函数体内为对象分配值时,实际上并没有构造该对象,只是在修改它的值。初始化器允许在创建数据成员时提供初始值,这比稍后分配值更有效率。

如果类有一个默认构造函数的数据成员,则不需要在初始化器中显式初始化那个对象。如果数据成员是一个std::string,默认构造函数会将字符串初始化为空字符串,所初始化器中初始化为""是多余的。

另一方面,如果类有一个没有默认构造函数的数据成员,必须使用初始化器来正确构造那个对象。考虑以下SpreadsheetCell类:

\begin{cpp}
export class SpreadsheetCell
{
    public:
        SpreadsheetCell(double d);
};
\end{cpp}

这个类只有一个接受double的构造函数,没有默认构造函数。可以将这个类作为另一个类的数据成员使用:

\begin{cpp}
class SomeClass
{
    public:
        SomeClass();
    private:
        SpreadsheetCell m_cell;
};
\end{cpp}

SomeClass构造函数可以实现为:

\begin{cpp}
SomeClass::SomeClass() { }
\end{cpp}

使用这个实现,代码无法编译。因为编译器不知道如何初始化SomeClass的m\_cell数据成员,所以没有默认构造函数。

必须使用初始化器在SomeClass的构造函数中初始化m\_cell数据成员:

\begin{cpp}
SomeClass::SomeClass() : m_cell { 1.0 } { }
\end{cpp}

\begin{myNotic}{NOTE}
ctor-initializers允许在创建数据成员时初始化数据成员。
\end{myNotic}

一些开发者更喜欢在构造函数体内分配初始值,这可能效率较低。然而,某些数据类型必须在初始化器中或在类定义中初始化。在这里做一下总结:

% Please add the following required packages to your document preamble:
% \usepackage{longtable}
% Note: It may be necessary to compile the document several times to get a multi-page table to line up properly
\begin{longtable}{|l|l|}
\hline
\textbf{数据类型} &
\textbf{解释} \\ \hline
\endfirsthead
%
\endhead
%
const数据成员 &
\begin{tabular}[c]{@{}l@{}}const变量创建后,不能为其赋值,值都必须在创建时提供。\end{tabular} \\ \hline
引用数据成员 &
\begin{tabular}[c]{@{}l@{}}需要引用存在的对象,并且创建后,引用就不能更改。\end{tabular} \\ \hline
\begin{tabular}[c]{@{}l@{}}无默认构造函数的\\对象数据成员\end{tabular} &
\begin{tabular}[c]{@{}l@{}}C++尝试使用默认构造函数初始化成员对象。如果不存在默\\认构造函数,则不能初始化对象,必须显式地说明调用哪个\\构造函数。\end{tabular} \\ \hline
\begin{tabular}[c]{@{}l@{}}无默认构造函数的\\基类\end{tabular} &
将在第10章介绍。 \\ \hline
\end{longtable}

使用构造函数初始化列表时有一个重要的注意事项:按照类定义中出现的顺序初始化数据成员,而不是构造函数初始化列表中的顺序!以下是一个名为 Foo 的类的定义。其构造函数只是存储一个 double 类型的值,并将其打印到控制台。

\begin{cpp}
class Foo
{
    public:
        Foo(double value);
    private:
        double m_value { 0 };
};

Foo::Foo(double value) : m_value { value }
{
    println("Foo::m_value = {}", m_value);
}
\end{cpp}

假设有另一个名为 MyClass 的类,包含一个 Foo 对象作为其数据成员。

\begin{cpp}
class MyClass
{
    public:
        MyClass(double value);
    private:
        double m_value { 0 };
        Foo m_foo;
};
\end{cpp}

其构造函数的实现:

\begin{cpp}
MyClass::MyClass(double value) : m_value { value }, m_foo { m_value }
{
    println("MyClass::m_value = {}", m_value);
}
\end{cpp}

构造函数初始化列表首先将给定的值存储在 m\_value 中,然后使用 m\_value 作为参数调用 Foo 构造函数。可以创建一个 MyClass 的实例:

\begin{cpp}
MyClass instance { 1.2 };
\end{cpp}

程序的输出:

\begin{cpp}
Foo::m_value = 1.2
MyClass::m_value = 1.2
\end{cpp}

现在对 MyClass 定义做一个微小的更改,只反转 m\_value 和 m\_foo 数据成员的顺序。

\begin{cpp}
class MyClass
{
    public:
        MyClass(double value);
    private:
        Foo m_foo;
        double m_value { 0 };
};
\end{cpp}

现在程序的输出取决于执行程序的系统。输出可能是这样的:

\begin{shell}
Foo::m_value = -9.255963134931783e+61
MyClass::m_value = 1.2
\end{shell}

这并非期望的结果。基于构造函数初始化列表假设在调用 Foo 构造函数之前会初始化 m\_value,但 C++ 并非如此。数据成员是按照类定义中顺序初始化,而不是构造函数初始化列表中的顺序!因此,Foo 构造函数首先调用,但 m\_value 还未初始化。

当构造函数初始化列表中的顺序与类定义中的顺序不匹配时,一些编译器会发出警告。

对于这个例子,有一个简单的修复方法。不要将 m\_value 传递给 Foo 构造函数,只需传递 value:

\begin{cpp}
MyClass::MyClass(double value) : m_value { value }, m_foo { value }
{
    println("MyClass::m_value = {}", m_value);
}
\end{cpp}

\begin{myWarning}{WARNING}
构造函数初始化列表按照类定义中声明的顺序初始化数据成员,而不是在构造函数初始化列表中的顺序。
\end{myWarning}

\mySamllsection{复制构造函数}

C++中有一种特殊的构造函数称为复制构造函数,允许创建一个与另一个对象完全相同的副本。以下是在 SpreadsheetCell 类中声明复制构造函数的方式:

\begin{cpp}
export class SpreadsheetCell
{
    public:
        SpreadsheetCell(const SpreadsheetCell& src);
        // Remainder of the class definition omitted for brevity
};
\end{cpp}

复制构造函数接受一个对源对象的常量引用。与其他构造函数一样,不返回值。复制构造函数应该复制源对象的所有数据成员。从技术上讲,可以在复制构造函数中做想做的事情,但最好遵循预期行为,将新对象初始化为旧对象的副本。以下是一个 SpreadsheetCell 复制构造函数的实现示例,注意其对构造函数初始化列表的使用。

\begin{cpp}
SpreadsheetCell::SpreadsheetCell(const SpreadsheetCell& src)
: m_value { src.m_value }
{}
\end{cpp}

如果不自己定义复制构造函数,C++ 会生成一个,将新对象中的每个数据成员从源对象中相应的数据成员初始化。对于对象数据成员,将调用复制构造函数。给定一组数据成员,m1、m2、… mn,编译器生成的复制构造函数可以表示为如下实现:

\begin{cpp}
classname::classname(const classname& src)
: m1 { src.m1 }, m2 { src.m2 }, ... mn { src.mn } { }
\end{cpp}

大多数情况下,不需要指定复制构造函数!

\begin{myNotic}{NOTE}
SpreadsheetCell复制构造函数仅用于演示目的,因为默认的编译器生成的复制构造函数已经足够好,所以可以省略复制构造函数。某些条件下,编译器生成的复制构造函数会不够用。这些将在第 9 章中详细介绍。
\end{myNotic}

\mySamllsection{何时调用复制构造函数}

C++中向函数传递参数的默认语义是按值传递,所以函数接收到的是值或对象的副本。每当将对象传递给函数时,编译器都会调用新对象的复制构造函数,有一个按值接受 std::string 参数的 printString() 函数:

\begin{cpp}
void printString(string value)
{
    println("{}", value);
}
\end{cpp}

std::string 实际上是一个类,而不是内置类型。代码调用 printString() 并传递一个字符串参数时,字符串参数 value 通过调用其复制构造函数进行初始化。复制构造函数的参数是传递给 printString() 的字符串。在以下示例中,字符串复制构造函数为 printString() 中的 value 对象执行,其参数为 name:

\begin{cpp}
string name { "heading one" };
printString(name); // Copies name
\end{cpp}

printString()成员函数结束时,value 销毁。其只是 name 的副本,所以 name 不变。可以通过按“引用转常量”的方式,传递参数来避免复制构造函数的开销。

当从函数按值返回对象时,也可能会调用复制构造函数。这将在本章后面的章节中介绍。

\mySamllsection{显式调用复制构造函数}

以显式使用复制构造函数,将一个对象构造为另一个对象的副本。创建一个 SpreadsheetCell 对象的副本:

\begin{cpp}
SpreadsheetCell myCell1 { 4 };
SpreadsheetCell myCell2 { myCell1 }; // myCell2 has the same values as myCell1
\end{cpp}

\mySamllsection{通过引用传递对象}

为了避免将对象传递给函数时复制对象,应该声明函数接受对象的引用。通过引用传递对象通常比按值传递更有效,只复制了对象的地址,而不是整个对象。

通过引用传递避免了对象中动态内存分配的问题,这将在第 9 章讨论。

通过引用传递对象时,使用对象引用的函数可能会更改原始对象。当为了效率而使用引用传递时,应该通过将对象声明为 const ,称为以“常量引用式”传递对象。

\begin{myNotic}{NOTE}
出于性能原因,最好通过“常量引用式”传递对象。第 9 章引入移动语义后,稍微修改了这条规则,允许在某些情况下按值传递对象。
\end{myNotic}

SpreadsheetCell 类有许多接受 std::string\_view 作为参数的成员函数。如第 2 章所述,string\_view 实际上只是一个指针和长度。复制成本很低,通常按值传递。

基本类型,如 int、double 等也应该按值传递。将此类类型作为“引用转常量”传递并不会有任何收益。

SpreadsheetCell 类的 doubleToString() 成员函数总是按值返回一个字符串,该成员函数的实现创建了一个局部字符串对象,在成员函数结束时将该字符串返回给调用者。不能返回对此字符串的引用不起作用,当函数退出时,将销毁所引用的字符串。

\mySamllsection{显式默认和删除复制构造函数}

可以为类显式默认或删除编译器生成的默认构造函数一样,也可以显式默认或删除编译器生成的复制构造函数:

\begin{cpp}
SpreadsheetCell(const SpreadsheetCell& src) = default;
\end{cpp}

或者

\begin{cpp}
SpreadsheetCell(const SpreadsheetCell& src) = delete;
\end{cpp}

通过删除复制构造函数,就不能再复制对象了,可以用来禁止按值传递对象。

\begin{myNotic}{NOTE}
如果类具有已删除或私有的复制构造函数的数据成员,即使显式地默认(=default)一个,类的复制构造函数也会自动删除。
\end{myNotic}

\mySamllsection{初始化列表构造函数}

初始化列表构造函数是一个构造函数,其第一个参数是 std::initializer\_list(参见第 1 章),并且没有其他参数或只有具有默认值的参数。initializer\_list 类模板定义在 <initializer\_list> 中。下面演示了其使用方式,该类只接受包含偶数个元素的 initializer\_list;否则,会抛出异常。

\begin{cpp}
class EvenSequence
{
    public:
        EvenSequence(initializer_list<double> values)
        {
            if (values.size() % 2 != 0) {
                throw invalid_argument { "initializer_list should "
                    "contain even number of elements." };
            }
            m_sequence.reserve(values.size());
            for (const auto& value : values) {
                m_sequence.push_back(value);
            }
        }

        void print() const
        {
            for (const auto& value : m_sequence) {
                std::print("{}, ", value);
            }
            println("");
        }
    private:
        vector<double> m_sequence;
};
\end{cpp}

初始化列表构造函数内部,可以使用基于范围的 for 循环访问初始化列表的元素,也可以使用 size() 成员函数获取 初始化列表 中的元素数量。

EvenSequence 初始化列表构造函数使用基于范围的 for 循环将元素从给定的 initializer\_list 复制到 m\_sequence,可以使用 vector 的 assign() 成员函数。vector 的不同成员函数,包括 assign(),会在第 18 章中详细介绍。先了解 vector 的功能,这里用了 assign() 的初始化列表构造函数:

\begin{cpp}
EvenSequence(initializer_list<double> values)
{
    if (values.size() % 2 != 0) {
        throw invalid_argument { "initializer_list should "
            "contain even number of elements." };
    }
    m_sequence.assign(values);
}
\end{cpp}

EvenSequence对象可以使用以下方式构造:

\begin{cpp}
try {
    EvenSequence p1 { 1.0, 2.0, 3.0, 4.0, 5.0, 6.0 };
    p1.print();
    EvenSequence p2 { 1.0, 2.0, 3.0 };
} catch (const invalid_argument& e) {
    println("{}", e.what());
}
\end{cpp}

p2 的构造将抛出异常,其初始化列表中有奇数个元素。

标准库完全支持初始化列表构造函数,std::vector 容器可以使用初始化列表初始化:

\begin{cpp}
vector<string> myVec { "String 1", "String 2", "String 3" };
\end{cpp}

如果没有初始化列表构造函数,初始化 vector 的方法是使用多个 push\_back() :

\begin{cpp}
vector<string> myVec;
myVec.push_back("String 1");
myVec.push_back("String 2");
myVec.push_back("String 3");
\end{cpp}

初始化列表不仅限于构造函数,也可以用于普通函数。

\begin{myNotic}{NOTE}
当一个类既有初始化列表构造函数,又有另一个单参数构造函数时,需要确定使用了正确的构造函数。std::vector有一个初始化列表构造函数,用于用给定的元素集初始化vector;还有一个接受单个参数的构造函数,即新vector的期望大小。使用大括号 \{\} 调用初始化列表构造函数,使用小括号 () 调用其他单参数构造函数。例如:

\begin{cpp}
vector<int> v1 { 6 }; // Constructs a vector with a single element, 6.
vector<int> v2 ( 6 ); // Constructs a vector with 6 default-
                      // initialized elements.
\end{cpp}
\end{myNotic}

\mySamllsection{委托构造函数}

委托构造函数允许构造函数调用同一类中的另一个构造函数,这个调用不能放在构造函数体中;必须位于构造函数初始化列表中,必须是列表中唯一的成员初始化器:

\begin{cpp}
SpreadsheetCell::SpreadsheetCell(string_view initialValue)
: SpreadsheetCell { stringToDouble(initialValue) }
{}
\end{cpp}

调用 string\_view 的构造函数(委托构造函数)时,首先将调用委托给目标构造函数,本例中是 double 构造函数。当目标构造函数返回时,将执行委托构造函数。

使用委托构造函数时,请确保避免构造函数的递归调用:

\begin{cpp}
class MyClass
{
    MyClass(char c) : MyClass { 1.2 } { }
    MyClass(double d) : MyClass { 'm' } { }
};
\end{cpp}

第一个构造函数将调用第二个构造函数,后者又委托回第一个构造函数。这种代码为标准中的“未定义行为”,并且结果依赖于编译器。

\mySamllsection{转换构造函数和显式构造函数}

SpreadsheetCell 类的构造函数如下所示:

\begin{cpp}
export class SpreadsheetCell
{
    public:
        SpreadsheetCell() = default;
        SpreadsheetCell(double initialValue);
        SpreadsheetCell(std::string_view initialValue);
        SpreadsheetCell(const SpreadsheetCell& src);
    // Remainder omitted for brevity
};
\end{cpp}

单参数 double 和 string\_view 构造函数可以用来将 double 或 string\_view 转换为 SpreadsheetCell,这样的构造函数称为转换构造函数。编译器可以使用这样的构造函数执行隐式转换:

\begin{cpp}
SpreadsheetCell myCell { 4 };
myCell = 5;
myCell = "6"sv; // A string_view literal (see Chapter 2).
\end{cpp}

这可能不是预期的行为。可以通过将构造函数标记为explicit,避免编译器执行这样的隐式转换。explicit关键字只能放在类定义中:

\begin{cpp}
export class SpreadsheetCell
{
    public:
        SpreadsheetCell() = default;
        SpreadsheetCell(double initialValue);
        explicit SpreadsheetCell(std::string_view initialValue);
        SpreadsheetCell(const SpreadsheetCell& src);
    // Remainder omitted for brevity
};
\end{cpp}

有了这个改变,以下行将无法通过编译:

\begin{cpp}
myCell = "6"sv; // A string_view literal (see Chapter 2).
\end{cpp}

C++11 之前,转换构造函数只能有一个参数,如 SpreadsheetCell 示例所示。自 C++11 之后,由于支持列表初始化,转换构造函数可以有多个参数:

\begin{cpp}
class MyClass
{
    public:
        MyClass(int) { }
        MyClass(int, int) { }
};
\end{cpp}

这个类有两个构造函数,自 C++11 以来,两者都是转换构造函数。以下示例显示编译器如何自动将 1、\{1\} 和 \{1,2\} 等参数转换为 MyClass 实例,并使用这些转换构造函数:

\begin{cpp}
void process(const MyClass& c) { }

int main()
{
    process(1);
    process({ 1 });
    process({ 1, 2 });
}
\end{cpp}

为了避免这些隐式转换,两个转换构造函数都可以标记为explicit:

\begin{cpp}
class MyClass
{
    public:
    explicit MyClass(int) { }
    explicit MyClass(int, int) { }
};
\end{cpp}

现在,必须显式执行这些转换:

\begin{cpp}
process(MyClass{ 1 });
process(MyClass{ 1, 2 });
\end{cpp}

将布尔参数传递给explicit,将其转换为条件explicit:

\begin{cpp}
explicit(true) MyClass(int);
\end{cpp}

explicit(true) 就等同于 explicit,但在使用类型特性进行泛型编程时,这会更加实用。通过类型特性,可以查询给定类型的某些属性,一个类型是否可以转换为另一个类型。这样的类型特性的输出可以作为 explicit() 的参数使用。

\begin{myNotic}{NOTE}
建议将可以接受单个参数调用的构造函数标记为explicit,以避免意外的隐式转换。如果有隐式转换的实际用例,可以将构造函数标记为 explicit(false)。这样做可以向类的用户解释,隐式转换是有意允许的。
\end{myNotic}

\mySamllsection{总结——编译器生成的构造函数}

编译器可以为每个类,自动生成默认构造函数和复制构造函数。编译器自动生成的构造函数,取决于按以下表格规则定义的构造函数。

% Please add the following required packages to your document preamble:
% \usepackage{longtable}
% Note: It may be necessary to compile the document several times to get a multi-page table to line up properly
\begin{longtable}{|l|l|l|}
\hline
\textbf{如果定义…} &
\textbf{\begin{tabular}[c]{@{}l@{}}编译器生成…\end{tabular}} &
\textbf{可以创建对象…} \\ \hline
\endfirsthead
%
\endhead
%
{[}没有构造函数{]} &
\begin{tabular}[c]{@{}l@{}}默认构造函数\\复制构造函数\end{tabular} &
\begin{tabular}[c]{@{}l@{}}无参数: SpreadsheetCell a;\\ 复制: SpreadsheetCell b\{a\};\end{tabular} \\ \hline
\begin{tabular}[c]{@{}l@{}}只有默认构造函数\end{tabular} &
复制构造函数 &
\begin{tabular}[c]{@{}l@{}}无参数: SpreadsheetCell a;\\ 复制: SpreadsheetCell b\{a\};\end{tabular} \\ \hline
\begin{tabular}[c]{@{}l@{}}只有复制构造函数\end{tabular} &
没有构造函数 &
\begin{tabular}[c]{@{}l@{}}理论上没有构造函数,可构\\造另一个对象的副本(因为\\没有非复制构造函数,不能\\创建对象)。\end{tabular} \\ \hline
\begin{tabular}[c]{@{}l@{}}单参数或多参数非\\复制构造函数\end{tabular} &
复制构造函数 &
\begin{tabular}[c]{@{}l@{}}有参数: SpreadsheetCell a\{6\};\\ 复制: SpreadsheetCell b\{a\};\end{tabular} \\ \hline
\begin{tabular}[c]{@{}l@{}}默认构造函数,以\\及单参数或多参数非复制构造函数\end{tabular} &
复制构造函数 &
\begin{tabular}[c]{@{}l@{}}无参数: SpreadsheetCell a;\\ 有参数: SpreadsheetCell b\{5\};\\ 复制: SpreadsheetCell c\{a\};\end{tabular} \\ \hline
\end{longtable}

注意默认构造函数和复制构造函数之间的不对称性。只要没有显式定义复制构造函数,编译器就会创建一个。另一方面,定义了构造函数,编译器就会停止生成默认构造函数。

默认构造函数和默认复制构造函数的自动生成,可以将它们显式定义为默认(=default)或显式删除(=delete)的影响。

\begin{myNotic}{NOTE}
还有一种最终类型的构造函数,即移动构造函数,是实现移动语义所必需的。移动语义在某些情况下用于提高性能,并在第 9 章中进行了详细介绍。
\end{myNotic}

\mySubsubsection{8.3.2.}{销毁对象}

当销毁一个对象时,会发生两件事:调用对象的析构函数成员函数,所占用的内存会释放。析构函数是为对象执行清理工作的机会,例如:释放动态分配的内存或关闭文件句柄。如果没有声明析构函数,编译器会生成,递归销毁成员,并删除对象。类的析构函数是一个成员函数,其名称是类名前缀一个波浪号 (~)。析构函数不返回值,也没有参数。以下是一个简单的析构函数示例,只是向标准输出写入一些内容:

\begin{cpp}
export class SpreadsheetCell
{
    public:
        ~SpreadsheetCell(); // Destructor.
        // Remainder of the class definition omitted for brevity
};

SpreadsheetCell::~SpreadsheetCell()
{
    println("Destructor called.");
}
\end{cpp}

当前函数或其他执行块结束时,栈上的对象在它们的作用域结束时销毁。当代码遇到一个结束的花括号,在这个花括号内创建的所有栈对象都会销毁。以下程序演示了这种行为:

\begin{cpp}
int main()
{
    SpreadsheetCell myCell { 5 };
    if (myCell.getValue() == 5) {
        SpreadsheetCell anotherCell { 6 };
    } // anotherCell is destroyed as this block ends.
    println("myCell: {}", myCell.getValue());
} // myCell is destroyed as this block ends.
\end{cpp}

栈上的对象是以声明(和构造)的逆序销毁。以下代码段中,myCell2 在 anotherCell2 之前创建,所以 anotherCell2 在 myCell2 之前销毁(可以在程序的任何地方开始一个新的代码块,使用花括号):

\begin{cpp}
{
    SpreadsheetCell myCell2 { 4 };
    SpreadsheetCell anotherCell2 { 5 }; // myCell2 constructed before anotherCell2
} // anotherCell2 destroyed before myCell2
\end{cpp}

这种顺序也适用于类的数据成员对象,数据成员按类中的声明顺序初始化。遵循对象按构造逆序销毁的规则,数据成员对象按类中的声明顺序逆序销毁。

没有智能指针帮助的情况下,在堆上分配的对象不会自动销毁。必须通过调用 delete 或 delete[] 来调用其析构函数并释放其内存,这取决于内存通过 new 还是 new[] 分配,或者使用智能指针!

\begin{myWarning}{WARNING}
以下示例是个反例。没有删除 cellPtr2 。确保是通过调用 delete 或 delete[] 释放动态分配的内存,这取决于内存通过 new 还是 new[] 分配,更好的做法是使用智能指针!
\end{myWarning}

\begin{cpp}
int main()
{
    SpreadsheetCell* cellPtr1 { new SpreadsheetCell { 5 } };
    SpreadsheetCell* cellPtr2 { new SpreadsheetCell { 6 } };
    println("cellPtr1: {}", cellPtr1->getValue());
    delete cellPtr1; // Destroys cellPtr1
    cellPtr1 = nullptr;
} // cellPtr2 is NOT destroyed because delete was not called on it.
\end{cpp}

\mySubsubsection{8.3.3.}{赋值于对象}

可以在 C++ 中将一个 int 的值赋给另一个 int,也可以将一个对象的值赋给另一个对象。以下代码将 myCell 的值赋给 anotherCell:

\begin{cpp}
SpreadsheetCell myCell { 5 }, anotherCell;
anotherCell = myCell;
\end{cpp}

有些读者可能会认为 myCell 会“复制”到 anotherCell。然而,在 C++ 的世界里,“复制”只发生在对象初始化时。如果一个对象已经有值并且正在覆盖,更准确的术语是“赋于”。C++ 提供用于复制的设施是复制构造函数。由于它是一个构造函数,只能用于对象创建,而不能用于对象后来的赋值。

C++ 在每个类中提供另一个成员函数来执行赋值,这个成员函数称为赋值操作符。名称是 operator=,因为它实际上是该类的 = 操作符的重载。上面的例子中,就是调用anotherCell 的赋值操作符,而myCell 是参数。

\begin{myNotic}{NOTE}
本节中解释的赋值操作符有时为复制赋值操作符,因为数据是从右操作数对象复制到左操作数对象。第 9 章讨论了另一种赋值操作符,即移动赋值操作符,其中数据是移动而不是复制,这在某些用例中可以提高性能。
\end{myNotic}

如果不编写赋值操作符,C++ 会为生成一个,以便对象可以相互赋值。默认的 C++ 赋值行为与默认的复制行为相同:递归地将源对象中的每个数据成员赋值给目标对象。

\mySamllsection{声明赋值操作符}

以下是 SpreadsheetCell 类的赋值操作符:

\begin{cpp}
export class SpreadsheetCell
{
    public:
        SpreadsheetCell& operator=(const SpreadsheetCell& rhs);
        // Remainder of the class definition omitted for brevity
};
\end{cpp}

赋值操作符通常接受一个对源对象的常量引用,就像复制构造函数一样。源对象称为 rhs,代表等号右侧,可以随意命名。调用赋值操作符的对象是等号左侧。

与复制构造函数不同,赋值操作符返回一个 SpreadsheetCell 对象的引用,赋值可以链式调用:

\begin{cpp}
myCell = anotherCell = aThirdCell;
\end{cpp}

执行该行时,首先发生的是以 aThirdCell 作为“右操作数”参数调用 anotherCell 的赋值操作符。接下来,调用 myCell 的赋值操作符。但参数不是 anotherCell;它的右操作数是 aThirdCell 赋值给 anotherCell 的结果。等号只是简写,实际是成员函数调用为:

\begin{cpp}
myCell.operator=(anotherCell.operator=(aThirdCell));
\end{cpp}

可以看到 anotherCell 的 operator= 调用必须返回一个值,这个值传递给 myCell 的 operator= 调用。正确返回的值是对 anotherCell 本身的引用,这样它就可以作为赋值给 myCell 。

\begin{myWarning}{WARNING}
可以声明赋值操作符返回想要的类型,包括 void。应该返回调用对象的引用,这是客户端期望的。
\end{myWarning}

\mySamllsection{定义赋值操作符}

赋值操作符的实现类似于复制构造函数,但存在一些重要差异。首先,复制构造函数仅用于初始化,所以目标对象还没有有效值,赋值操作符可以覆盖对象中的当前值。这种考虑在对象中拥有动态分配的资源时(如内存),才会发挥作用。请参阅第 9 章以获取详细信息。

其次,将一个对象赋值给自己在 C++ 中完全合法:

\begin{cpp}
SpreadsheetCell cell { 4 };
cell = cell; // Self-assignment
\end{cpp}

赋值操作符需要考虑到自赋值的可能性。在 SpreadsheetCell 类中,这并不重要,它唯一数据成员是基本类型 double。当类拥有动态分配的内存或其他资源时,考虑自赋值至关重要。为了避免在这种情况下出现问题,赋值操作符通常首先检查自赋值,并在那钟情况时立即返回。

以下是 SpreadsheetCell 类的赋值操作符的定义的开始部分:

\begin{cpp}
SpreadsheetCell& SpreadsheetCell::operator=(const SpreadsheetCell& rhs)
{
    if (this == &rhs) {
\end{cpp}

这第一行检查自赋值,但可能有些难以理解。自赋值发生在等号左侧和右侧相同时。判断两个对象是否相同的一种方法是,是否占用相同的内存位置——更具体地说,如果指向它们的指针相等。回忆一下,this 是一个指向对象的可从成员函数访问的指针,所以this 是左侧对象的指针。类似地,\&rhs 是右侧对象的指针。如果这些指针相等,赋值必须是自我赋值,但返回类型是 SpreadsheetCell\&,因此必须正确返回一个值。所有赋值操作符都以以下方式返回 *this:

\begin{cpp}
        return *this;
    }
\end{cpp}

this 是成员函数执行的对象的指针,*this 是对象本身。编译器返回对象的引用,以匹配声明的返回类型。如果它不是自赋值,就需要对每个成员进行赋值:

\begin{cpp}
    m_value = rhs.m_value;
    return *this;
}
\end{cpp}

成员函数复制了值,最后返回了 *this。

细心的读者可能会注意到,复制赋值操作符和复制构造函数之间存在一些代码重复;都需要复制所有数据成员。第 9 章介绍了复制和交换模式,以防止此类代码重复。

\begin{myNotic}{NOTE}
SpreadsheetCell 赋值操作符仅用于演示目的。实际上,可以省略赋值操作符,因为默认的编译器生成的赋值操作符已经足够好;对所有数据成员进行了简单的成员逐个赋值。然而,在某些条件下,这个编译器生成的赋值操作符是不够的。这些条件会在第 9 章节中进行讨论。
\end{myNotic}

\begin{myWarning}{WARNING}
如果类需要对复制操作进行特殊处理,请始终自行实现复制构造函数和复制赋值操作符。
\end{myWarning}

\mySamllsection{显式默认和删除赋值操作符}

可以显式默认或删除编译器生成的赋值操作符:

\begin{cpp}
SpreadsheetCell& operator=(const SpreadsheetCell& rhs) = default;
\end{cpp}

或

\begin{cpp}
SpreadsheetCell& operator=(const SpreadsheetCell& rhs) = delete;
\end{cpp}

\mySubsubsection{8.3.4.}{编译器生成的复制构造函数和复制赋值操作符}

C++11 在类具有用户声明的复制赋值操作符或析构函数时,不会生成复制构造函数。如果在这种情况下仍然需要编译器生成的复制构造函数,可以使用=default:

\begin{cpp}
MyClass(const MyClass& src) = default;
\end{cpp}

C++11 在类具有用户声明的复制赋值操作符或析构函数时,不会生成复制赋值操作符。如果在这种情况下仍然需要编译器生成的复制赋值操作符,可以使用=default:

\begin{cpp}
MyClass& operator=(const MyClass& rhs) = default;
\end{cpp}

\mySubsubsection{8.3.5.}{复制和赋值}

有时很难判断对象是通过复制构造函数初始化,还是通过赋值操作符赋值。看起来像声明的东西会使用复制构造函数,而看起来像赋值语句的东西会由赋值操作符处理。来看看以下代码:

\begin{cpp}
SpreadsheetCell myCell { 5 };
SpreadsheetCell anotherCell { myCell };
\end{cpp}

AnotherCell 使用复制构造函数构造。

\begin{cpp}
SpreadsheetCell aThirdCell = myCell;
\end{cpp}

aThirdCell 也使用复制构造函数构造,这是一个声明,没有调用operator= !这种语法等价于 SpreadsheetCell aThirdCell\{myCell\} 。

\begin{cpp}
anotherCell = myCell; // Calls operator= for anotherCell
\end{cpp}

anotherCell 已经构造好了,所以编译器调用 operator=。

\mySamllsection{对象作为返回值}

当从函数返回对象时,有时很难清楚地看到发生了哪些复制和赋值操作。例如,SpreadsheetCell::getString() 的实现如下:

\begin{cpp}
string SpreadsheetCell::getString() const
{
    return doubleToString(m_value);
}
\end{cpp}

看看如下代码:

\begin{cpp}
SpreadsheetCell myCell2 { 5 };
string s1;
s1 = myCell2.getString();
\end{cpp}

当 getString() 返回字符串时,编译器实际上通过调用字符串复制构造函数创建了一个临时字符串对象。将这个结果赋值给 s1 时,调用赋值操作符,参数是临时字符串,再销毁临时字符串对象。因此,这一行代码可能会调用复制构造函数和赋值操作符(针对两个不同的对象)。

如果还没有感到困惑,再看看这里:

\begin{cpp}
SpreadsheetCell myCell3 { 5 };
string s2 = myCell3.getString();
\end{cpp}

getString() 仍然在返回时创建了一个临时字符串对象,但现在 s2 调用的是它的复制构造函数,而不是赋值操作符。

有了移动语义,编译器可以使用移动构造函数或移动赋值操作符返回字符串。这在某些情况下可能更有效率,编译器可以自由地(并且通常甚至是要求)实现复制消除,来优化返回值中的昂贵的复制操作或移动操作。

如果忘记了这些事情发生的顺序或调用了哪个构造函数或操作符,可以在代码中临时包含有用的输出或使用调试器逐行执行代码来找出答案。

\mySamllsection{复制构造函数和对象成员}

还应该注意在构造函数中,复制构造函数调用和赋值操作符调用的区别。如果一个对象包含其他对象,编译器生成的复制构造函数会递归地调用每个包含对象的复制构造函数。当编写自己的复制构造函数时,可以通过使用构造函数初始化列表来提供相同的语义。如果从构造函数初始化列表中省略了一个数据成员,编译器会在执行构造函数体中的代码之前对其进行默认初始化(对对象调用默认构造函数)。在构造函数体执行时,所有对象数据成员都已经初始化完毕。

可以这样编写 SpreadsheetCell 的复制构造函数:

\begin{cpp}
SpreadsheetCell::SpreadsheetCell(const SpreadsheetCell& src)
{
    m_value = src.m_value;
}
\end{cpp}

当在复制构造函数体中给数据成员赋值时,它们已经初始化了,所以使用的是赋值操作符,而不是复制构造函数。

如果这样编写复制构造函数,m\_value 就需要使用复制构造函数进行初始化:

\begin{cpp}
SpreadsheetCell::SpreadsheetCell(const SpreadsheetCell& src)
: m_value { src.m_value }
{}
\end{cpp}



















