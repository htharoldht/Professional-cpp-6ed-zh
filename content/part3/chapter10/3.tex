
编写派生类时,需要注意父类和子类之间的交互。诸如创建顺序、构造函数链和类型转换等,都可能是错误的来源。

\mySubsubsection{10.3.1}{父类的构造函数}

对象并不是一下子就生成的,必须与父类以及所包含的对象一起构造。C++定义了以下创建顺序:

\begin{enumerate}
\item
如果类有基类,则执行基类的默认构造函数,除非在初始化器中有对基类构造函数的调用。

\item
类的非静态数据成员,按照声明的顺序构造。

\item
执行类的构造函数。
\end{enumerate}

这些规则可以递归应用。如果类有一个祖父类,那么在父类需要初始化祖父类,依此类推。以下代码显示了这种创建顺序。此代码的正确执行输出为123。

\begin{cpp}
class Something
{
    public:
        Something() { print("2"); }
};

class Base
{
    public:
        Base() { print("1"); }
};

class Derived : public Base
{
    public:
        Derived() { print("3"); }
    private:
        Something m_dataMember;
};

int main()
{
    Derived myDerived;
}
\end{cpp}

创建myDerived对象时,首先调用Base的构造函数,输出字符串"1"。接下来,初始化m\_dataMember,调用Something构造函数,输出字符串"2"。最后,调用Derived构造函数,输出"3"。

注意,Base构造函数是自动调用的。如果存在,C++会自动调用父类的默认构造函数。父类中不存在默认构造函数,或者想使用另一个父构造函数,可以在初始化器中链式调用构造函数,就像初始化数据成员一样。以下代码展示了一个没有默认构造函数的Base类,Derived必须明确告诉编译器如何调用Base构造函数,否则代码将无法编译。

\begin{cpp}
class Base
{
    public:
        explicit Base(int i) {}
};

class Derived : public Base
{
    public:
        Derived() : Base { 7 } { /* Other Derived's initialization ... */ }
};
\end{cpp}

Derived构造函数将常量(7)传递给Base构造函数,Derived也可以传递一个变量:

\begin{cpp}
Derived::Derived(int i) : Base { i } { /* Other Derived's initialization ... */ }
\end{cpp}

从派生类向基类传递构造函数参数正常且常见的,传递数据成员将不起作用,但代码可以编译。需要注意的是,数据成员直到基类构造后才会初始化。如果将数据成员作为参数传递给父构造函数,成员变量将处于未初始化状态。

\mySubsubsection{10.3.2}{父类的析构函数}

因为析构函数不能接受参数,所以可以自动调用父类的析构函数。销毁的顺序与构造顺序相反:

\begin{enumerate}
\item
调用类的析构函数。

\item
以与构造相反的顺序销毁类的数据成员。

\item
如果有父类的话,销毁父类。
\end{enumerate}

再次强调,这些规则递归适用,创建链的最底层成员总是首先销毁。以下代码添加了析构函数,析构函数声明为虚!如果执行,此代码输出"123321"。

\begin{cpp}
class Something
{
    public:
        Something() { print("2"); }
        virtual ~Something() { print("2"); }
};

class Base
{
    public:
        Base() { print("1"); }
        virtual ~Base() { print("1"); }
};

class Derived : public Base
{
    public:
        Derived() { print("3"); }
        virtual ~Derived() override { print("3"); }
    private:
        Something m_dataMember;
};
\end{cpp}

如果前面的析构函数没有声明为虚,代码似乎可以正常工作。如果代码曾经对一个实际上指向Derived实例的Base指针调用delete,析构链将出错错误。从前面代码的所有析构函数中移除virtual和override关键字,当一个Derived对象当作指向Base的指针访问并删除时,就会出现问题:

\begin{cpp}
Base* ptr { new Derived{} };
delete ptr;
\end{cpp}

这段代码的输出令人震惊:“1231”。当删除ptr变量时,因为析构函数没有声明为虚,所以只调用了Base的析构函数。结果,没有调用Derived析构函数,也没有调用其数据成员的析构函数!

从技术上讲,可以通过将Base析构函数标记为虚来修复前面的问题。这种“虚性”会自动应用于所有派生类。然而,我建议明确地将所有析构函数都设置为虚,这样就永远不需要担心这个问题了。

\begin{myWarning}{WARNING}
总是让析构函数为虚!编译器生成的默认析构函数不为虚,所以应该定义(或显式地设置为默认)一个虚析构函数,至少对于非final基类来说是要这样做的。
\end{myWarning}

\mySubsubsection{10.3.3}{构造函数和析构函数中使用虚成员函数}

构造函数和析构函数中,虚成员函数的行为有所不同。如果派生类覆写了基类中的一个虚成员函数,从基类的构造函数或析构函数中调用该成员函数时,会调用基类实现的虚成员函数!换句话说,从构造函数或析构函数内部对虚成员函数的使用,是在编译时静态解析的。

这种行为背后原因是派生类实例初始化的顺序。创建派生类实例时,基类的构造函数都会首先调用,然后才是派生类实例的初始化。在派生类尚未完全初始化的情况下,调用覆写的虚成员函数非常危险。析构函数的情况也类似,因为销毁对象时,析构函数的调用顺序相反。

如果确实需要在构造函数中实现多态行为(并不推荐这么做),可以在基类中定义一个initialize()虚成员函数,派生类可以覆写它。创建类实例,需要在构造完成后调用这个initialize()成员函数。

同样,如果需要在析构函数中实现多态行为(同样不推荐),可以定义一个shutdown()虚成员函数,在对象销毁前调用。

\mySubsubsection{10.3.4}{引用父类名称}

在一个派生类中覆写成员函数时,实际上是在其他代码的考虑中替换了原始函数。然而,父类的那个成员函数仍然存在,有时可能想要使用它。一个覆写的成员函数可能希望继续执行基类实现的功能,并在此基础上增加一些其他功能。WeatherPrediction类中的getTemperature()成员函数,返回当前温度的字符串:

\begin{cpp}
export class WeatherPrediction
{
    public:
        virtual std::string getTemperature() const;
        // Remainder omitted for brevity.
};
\end{cpp}

可以在MyWeatherPrediction类中这样覆写这个成员函数:

\begin{cpp}
export class MyWeatherPrediction : public WeatherPrediction
{
    public:
        std::string getTemperature() const override;
        // Remainder omitted for brevity.
};
\end{cpp}

假设派生类首先调用基类的getTemperature()成员函数,并在其后添加°F:

\begin{cpp}
string MyWeatherPrediction::getTemperature() const
{
    // Note: \u00B0 is the ISO/IEC 10646 representation of the degree symbol.
    return getTemperature() + "\u00B0F"; // BUG
}
\end{cpp}

这没什么用,根据C++的名称解析规则,首先解析本地作用域,然后解析类作用域,结果最终调用MyWeatherPrediction::getTemperature()。这会导致无限递归,直到栈空间耗尽(一些编译器在编译时检测到这个错误并报告)。

为了让这个功能正常工作,需要使用作用域解析运算符:

\begin{cpp}
string MyWeatherPrediction::getTemperature() const
{
    // Note: \u00B0 is the ISO/IEC 10646 representation of the degree symbol.
    return WeatherPrediction::getTemperature() + "\u00B0F";
}
\end{cpp}

\begin{myNotic}{NOTE}
Microsoft Visual C++支持非标准的\_\_super关键字(有两个下划线)。可以这样写:

\begin{cpp}
return __super::getTemperature() + "\u00B0F";
\end{cpp}
\end{myNotic}

调用父类成员函数,在C++中是一种常见模式。如果有一系列的派生类,每个类可能都希望执行基类已经定义的操作,但也有自己独有的功能。

来看另一个例子,如图10.6所示的书籍类型类层次结构。

\myGraphic{0.3}{content/part3/chapter10/images/6.png}{图 10.6}

类层次结构中,每个更低的类别指定了书籍的类型,因此获取书籍描述的成员函数确实需要考虑层次结构中的所有级别。这可以通过链接到父成员函数来实现:

\begin{cpp}
class Book
{
    public:
        virtual ~Book() = default;
        virtual string getDescription() const { return "Book"; }
        virtual int getHeight() const { return 120; }
};

class Paperback : public Book
{
    public:
        string getDescription() const override {
            return "Paperback " + Book::getDescription();
        }
};

class Romance : public Paperback
{
    public:
        string getDescription() const override {
            return "Romance " + Paperback::getDescription();
        }
        int getHeight() const override { return Paperback::getHeight() / 2; }
};

class Technical : public Book
{
    public:
        string getDescription() const override {
            return "Technical " + Book::getDescription();
        }
};

int main()
{
    Romance novel;
    Book book;
    println("{}", novel.getDescription()); // Outputs "Romance Paperback Book"
    println("{}", book.getDescription()); // Outputs "Book"
    println("{}", novel.getHeight()); // Outputs "60"
    println("{}", book.getHeight()); // Outputs "120"
}
\end{cpp}

Book基类有两个虚成员函数:getDescription()和getHeight()。所有派生类都覆写了getDescription(),只有Romance类通过调用其父类(Paperback)的getHeight()并除以2来覆写getHeight()。Paperback没有覆写getHeight(),但C++会向上遍历类的层次结构,找到一个实现getHeight()的类。这个例子中,Paperback::getHeight()会解析为Book::getHeight()。

\mySubsubsection{10.3.5}{上下转换}

可以将对象强制转换为其父类,或者赋值给它的父类:

\begin{cpp}
Derived myDerived;
Base myBase { myDerived }; // Slicing!
\end{cpp}

这种情况下会发生切片,因为最终结果是一个Base对象,而Base对象缺少在Derived类中定义的功能。如果将派生类赋值给指向其基类的指针或引用,则不会发生切片:

\begin{cpp}
Base& myBase { myDerived }; // No slicing!
\end{cpp}

通常,这是正确地以基类的形式引用派生类的方式,也称为向上转换。这就是为什么函数接受类的引用,而不是直接使用这些类的对象。通过使用引用,派生类可以在不切片的情况下进行传递。

\begin{myWarning}{WARNING}
进行向上转换时,请使用指向基类的指针或引用来避免切片。
\end{myWarning}

从基类转换到其派生类,也称为向下转换,通常不受专业C++开发者的欢迎。因为不保证对象确实属于那个派生类,并且向下转换是设计不良的标志:

\begin{cpp}
void presumptuous(Base* base)
{
    Derived* myDerived { static_cast<Derived*>(base) };
    // Proceed to access Derived member functions on myDerived.
}
\end{cpp}

如果presumptuous()的作者也编写了调用presumptuous()的代码,因为作者知道函数期望参数是Derived类型的,一切可能都会好,尽管代码很丑。如果其他开发者调用presumptuous(),可能会传递一个Base。没有编译时检查可以执行来强制参数的类型,并且函数盲目地假设base实际上是指向Derived对象的指针。

向下转换有时是必要的,可以在受控的情况下使用。如果要进行向下转换,应该使用dynamic\_cast(),使用对象的内置类型来拒绝不合理的转换。这种内置类型通常位于vtable中,所以dynamic\_cast()只对具有vtable的对象有效,即至少有一个虚拟成员的对象。如果dynamic\_cast()在指针上转换失败,结果将是nullptr,而不是指向非意义数据的指针。如果dynamic\_cast()在对象引用上转换失败,将抛出std::bad\_cast异常。本章的最后会详细介绍不同的转换方式。

上一个示例可以这样写:

\begin{cpp}
void lessPresumptuous(Base* base)
{
    Derived* myDerived { dynamic_cast<Derived*>(base) };
    if (myDerived != nullptr) {
        // Proceed to access Derived member functions on myDerived.
    }
}
\end{cpp}

然而,使用向下转换通常是设计不良的标志。应该重新思考并修改设计,尽可能避免向下转换。例如,lessPresumptuous()函数只真正适用于Derived对象,所以应该接受Derived指针,而不是Base指针,这将消除向下转换的必要。如果函数应该与从Base继承的不同派生类一起工作,可以使用多态性的解决方案,这将在下一节中介绍。

\begin{myWarning}{WARNING}
仅在必要时使用向下转换,并使用dynamic\_cast()。
\end{myWarning}






