
已经熟悉了继承的基本语法,现在是时候探索继承了。继承是一种允许利用现有代码的工具。本节介绍了一个继承的示例,目的是为了复用代码。

\mySubsubsection{10.2.1}{WeatherPrediction类}

目前有一个编程任务,该程序能够发布简单的天气预报,处理华氏度和摄氏度。天气预报可能超出了你作为开发者的擅长领域,所以去找了一个第三方类库,该类库是根据当前温度和木星与火星之间的当前距离编写,用于预测天气(嘿,这是可能的)。这个第三方包作为编译库分发,需要保护预测算法的知识产权,但确实可以看到类的定义。weather\_prediction模块接口文件如下所示:

\begin{cpp}
export module weather_prediction;
import std;
// 使用当前温度和木星与火星之间的距离,使用经过验证的新技术预测天气。如果这些值没有提供,仍然会给出一个猜测,但准确度只有99%。
export class WeatherPrediction
{
    public:
        // Virtual destructor
        virtual ~WeatherPrediction();
        // 设置当前华氏温度
        virtual void setCurrentTempFahrenheit(int temp);
        // 设置木星与火星之间的当前距离
        virtual void setPositionOfJupiter(int distanceFromMars);
        // 获取明天温度的预测
        virtual int getTomorrowTempFahrenheit() const;
        // 获取明天降雨的概率。1表示确定降雨。0表示没有降雨的可能。
        virtual double getChanceOfRain() const;
        // 以这种格式向用户显示结果:
        // Result: x.xx chance. Temp. xx
        virtual void showResult() const;
        // 返回温度的字符串表示
        virtual std::string getTemperature() const;
    private:
        int m_currentTempFahrenheit { 0 };
        int m_distanceFromMars { 0 };
};
\end{cpp}

注意,该类假定它们可能在派生类中覆写,所以这个类将所有成员函数设置为虚。

这个类解决了程序的大部分问题,但并不完全符合需求。首先,所有的温度都用华氏度给出,程序还需要以摄氏度进行,showResult()成员函数可能不会按要求的方式显示结果。

\mySubsubsection{10.2.2}{派生类中添加功能}

在第5章中学习继承时,添加功能是第一个介绍的。程序需要一个类似于WeatherPrediction类的类,但带有一些功能。听起来继承是一个很好的重用代码的案例,定义一个新类MyWeatherPrediction,继承自WeatherPrediction:

\begin{cpp}
import weather_prediction;

export class MyWeatherPrediction : public WeatherPrediction
{
};
\end{cpp}

前面的类定义编译没问题。MyWeatherPrediction类可以用来代替WeatherPrediction,提供相同的功能,但目前还没有新的功能。为了这个修改,需要在类中添加摄氏度的相关内容。这里有一个难题,不知道类内部是如何工作的。如果所有的内部计算都用华氏度进行,如何添加对摄氏度的支持?一种方法是让派生类作为中介,在用户(可以使用其他单位)和基类(只理解华氏度)之间做中介。

支持摄氏度的第一步是在类中添加新的成员函数,允许客户端以摄氏度设置当前温度,并以摄氏度获取明天的预测。还需要私有的辅助函数,在两个方向上进行摄氏度和华氏度的转换。它们对所有实例都是相同的,所以这些函数可以是静态的。

\begin{cpp}
export class MyWeatherPrediction : public WeatherPrediction
{
    public:
        virtual void setCurrentTempCelsius(int temp);
        virtual int getTomorrowTempCelsius() const;
    private:
        static int convertCelsiusToFahrenheit(int celsius);
        static int convertFahrenheitToCelsius(int fahrenheit);
};
\end{cpp}

新成员函数遵循与父类相同的命名约定。从其他代码的角度来看,MyWeatherPrediction对象具有MyWeatherPrediction和WeatherPrediction中定义的所有功能,采用父类的命名约定可以提供一致的接口。

摄氏度/华氏度转换函数的实现留给读者作为练习——这是一个有趣的练习!另外两个成员函数更有趣。为了以摄氏度设置当前温度,首先转换温度,然后以基类理解的单位呈现给基类:

\begin{cpp}
void MyWeatherPrediction::setCurrentTempCelsius(int temp)
{
    int fahrenheitTemp { convertCelsiusToFahrenheit(temp) };
    setCurrentTempFahrenheit(fahrenheitTemp);
}
\end{cpp}

温度转换时,成员函数就会调用基类的现有功能。类似地,getTomorrowTempCelsius()的实现使用父类的现有功能,来获取华氏度,但需要在返回结果前进行转换:

\begin{cpp}
int MyWeatherPrediction::getTomorrowTempCelsius() const
{
    int fahrenheitTemp { getTomorrowTempFahrenheit() };
    return convertFahrenheitToCelsius(fahrenheitTemp);
}
\end{cpp}

它们以提供新接口的方式“包装”了现有的功能,可以说这两个新成员函数有效地重用了父类。

可以完全添加与父类现有功能无关的新功能。添加一个成员函数,从互联网获取其他预报,或者根据预测的天气添加一个建议活动的成员函数。

\mySubsubsection{10.2.3}{派生类中的功能替换}

继承的另一个主要技术是替换现有功能。WeatherPrediction类中的showResult()成员函数急需改头换面,MyWeatherPrediction可以覆写这个成员函数,用自己的实现来替换其行为。

MyWeatherPrediction的新类定义如下:

\begin{cpp}
export class MyWeatherPrediction : public WeatherPrediction
{
    public:
        virtual void setCurrentTempCelsius(int temp);
        virtual int getTomorrowTempCelsius() const;
        void showResult() const override;
    private:
        static int convertCelsiusToFahrenheit(int celsius);
        static int convertFahrenheitToCelsius(int fahrenheit);
};
\end{cpp}

这是覆写的showResult()成员函数:

\begin{cpp}
void MyWeatherPrediction::showResult() const
{
    println("Tomorrow will be {} degrees Celsius ({} degrees Fahrenheit)",
        getTomorrowTempCelsius(), getTomorrowTempFahrenheit());
    println("Chance of rain is {}%", getChanceOfRain() * 100);
    if (getChanceOfRain() > 0.5) { println("Bring an umbrella!"); }
}
\end{cpp}

对于使用这个类的使用者来说,好像旧版本的showResult()从未存在过。只要对象是MyWeatherPrediction对象,就会调用新版本。MyWeatherPrediction作为一个新类,具有针对更具体目的的新功能,因为它利用了其基类的现有功能,所以无需编写大量代码。






