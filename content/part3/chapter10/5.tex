
正如第5章中所言,多重继承通常认为是面向对象编程中复杂且不必要的部分。关于它是否有用,我不做评论。不过,这一节会来介绍C++中多重继承的机制。

\mySubsubsection{10.5.1}{继承于多个类}

继承多个基类在语法上很简单,只需要在声明类名时列出基类即可。

\begin{cpp}
class Baz : public Foo, public Bar { /* Etc. */ };
\end{cpp}

通过列出多个基类,Baz对象具有以下特征:

\begin{itemize}
\item
Baz对象支持Foo和Bar的public成员函数,并包含这两个类的数据成员。

\item
Baz类的成员函数可以访问Foo和Bar的protected的数据和成员函数。

\item
可以将Baz对象向上转换为Foo或Bar。

\item
创建新的Baz对象会自动调用Foo和Bar的默认构造函数,按照在类定义中的类的顺序。

\item
删除Baz对象会自动调用Foo和Bar类的析构函数,按照在类定义中列出类的相反顺序。
\end{itemize}

以下示例展示了一个具有两个父类的类DogBird——一个Dog类和一个Bird类,如图10.9所示,这个例子确实有些荒谬。

\myGraphic{0.3}{content/part3/chapter10/images/9.png}{图 10.9}

\begin{cpp}
class Dog
{
    public:
        virtual void bark() { println("Woof!"); }
};

class Bird
{
    public:
        virtual void chirp() { println("Chirp!"); }
};

class DogBird : public Dog, public Bird
{
};
\end{cpp}

使用多个父类的类的对象与单个父类的对象的用法没有区别,客户端代码甚至不必知道类有两个父类。真正重要的是类支持的特性和行为,DogBird对象支持Dog和Bird的所有public成员函数。

\begin{cpp}
DogBird myConfusedAnimal;
myConfusedAnimal.bark();
myConfusedAnimal.chirp();
\end{cpp}

程序的输出如下所示:

\begin{shell}
Woof!
Chirp!
\end{shell}

\mySubsubsection{10.5.2}{命名冲突和歧义基类}

构建一个多重继承因会崩溃的场景并不困难,以下示例展示了必须考虑的“边缘”情况。

\mySamllsection{命名歧义}

如果Dog类和Bird类都有一个名为eat()的成员函数,会发生什么?由于Dog和Bird没有关系,成员函数并不会覆写——在DogBird派生类中都继续存在。 只要客户端代码从不调用成员函数eat(),就没有问题。

尽管它有两个版本的eat(),DogBird类还是可以正确编译。如果客户端代码尝试在DogBird上调用eat()成员函数,编译器将报错,指出调用eat()有歧义,编译器不知道调用哪个eat():

\begin{cpp}
class Dog
{
    public:
        virtual void bark() { println("Woof!"); }
        virtual void eat() { println("The dog ate."); }
};

class Bird
{
    public:
        virtual void chirp() { println("Chirp!"); }
        virtual void eat() { println("The bird ate."); }
};

class DogBird : public Dog, public Bird
{
};

int main()
{
    DogBird myConfusedAnimal;
    myConfusedAnimal.eat(); // Error! Ambiguous call to member function eat()
}
\end{cpp}

如果注释掉从main()调用eat()的最后一行,代码可以正确编译。

解决歧义的方法是,使用dynamic\_cast()显式向上转换对象,从而在编译器中隐藏不需要的成员函数版本,或者使用歧义消除语法。以下代码展示了两种方式调用Dog的eat():

\begin{cpp}
dynamic_cast<Dog&>(myConfusedAnimal).eat(); // Calls Dog::eat()
myConfusedAnimal.Dog::eat(); // Calls Dog::eat()
\end{cpp}

派生类的成员函数也可以通过使用访问父成员函数的相同语法,来明确消除对同名不同版本成员函数的歧义,即::作用域解析运算符。DogBird类可以通过定义自己的eat()成员函数,来防止其他代码中的歧义错误。在这个成员函数中,可以确定调用哪个父类eat()。

\begin{cpp}
class DogBird : public Dog, public Bird
{
    public:
        void eat() override
        {
            Dog::eat(); // Explicitly call Dog's version of eat()
        }
};
\end{cpp}

避免歧义错误的另一种方法是使用using声明,明确指出DogBird应该继承哪个版本的eat()。这是一个例子:

\begin{cpp}
class DogBird : public Dog, public Bird
{
    public:
        using Dog::eat; // Explicitly inherit Dog's version of eat()
};
\end{cpp}

\mySamllsection{基类歧义}

另一种引发歧义的方式是两次继承同一个类。如果多个父类本身有共同的父类,这就有可能会发生。也许Bird和Dog都继承自一个Animal类,如图10.10所示。

\myGraphic{0.3}{content/part3/chapter10/images/10.png}{图 10.10}

C++允许这种类型的类层次结构,仍可能发生命名歧义。Animal类有一个public成员函数sleep(),这个成员函数不能在DogBird对象上调用,因为编译器不知道是调用Dog的,还是Bird的。

使用“钻石”类层次结构的最佳方式是,使最顶层的类成为一个纯虚抽象基类,其中所有成员函数都声明为纯虚。因为该类只声明成员函数而不提供定义,基类中没有成员函数可以调用,所以在这个层面上没有歧义。

以下示例实现了一个钻石形状的类层次结构,其中Animal抽象基类有一个必须由每个派生类定义的纯虚eat()成员函数。DogBird类仍然需要明确使用哪个父类的eat()成员函数,但歧义是因Dog和Bird具有相同成员函数引起的,而不是因为它们继承自同一个类。

\begin{cpp}
class Animal
{
    public:
        virtual void eat() = 0;
};

class Dog : public Animal
{
    public:
        virtual void bark() { println("Woof!"); }
        void eat() override { println("The dog ate."); }
};

class Bird : public Animal
{
    public:
        virtual void chirp() { println("Chirp!"); }
        void eat() override { println("The bird ate."); }
};

class DogBird : public Dog, public Bird
{
    public:
        using Dog::eat;
};
\end{cpp}

对于钻石层次结构的顶部类,处理更精细的机制,即虚基类,将在后面介绍。

\mySamllsection{多重继承的用途}

现在,可能想知道为什么要在代码中使用多重继承。多重继承的最直接用途是定义一个既是某种东西,又是另一种东西的对象类。正如第5章所说,遵循这种模式的真实世界对象,都不太好转换成代码。

多重继承最有说服力和简单的用途是实现混合类(mixin classes)。混合类在第5章中引入,并在第32章中进行了更详细的介绍。人们有时使用多重继承的另一个原因是,模拟基于组件的类。

第5章给出了一个飞机模拟器的例子。Airplane类有一个引擎、机身、控制装置和其他组件。虽然Airplane类的典型实现会让这些组件成为单独的数据成员,但也可以使用多重继承。飞机类将从Engine、Fuselage和Controls继承,实际上获取所有组件的行为和属性。我建议远离这种类型的代码,它混淆了清晰的“has a”关系与继承,继承应该用于“is a”关系。推荐的解决方案是让Airplane类包含类型为Engine、Fuselage和Controls的数据成员。

