
流这个概念需要一点时间去适应,流可能看起来比传统的C风格I/O,比如printf()。实际上,流看起来复杂,只是因为流比printf()更复杂。不过别担心:通过几个例子,一切都会变得清晰起来。

\mySubsubsection{13.1.1}{流是什么?}

第1章,将cout流比作数据槽,将一些变量投入流中,就会输出到用户的屏幕或控制台。更一般地说,所有流都可以视为数据槽。流在方向和关联的源或目的地方面有所不同。熟悉的cout流是一个输出流,它的方向是“出”。将数据写入控制台,所以关联目的地是“控制台”。cout中的c并不像你预期的代表“控制台”,而是代表“字符”,它是一个基于字符的流。还有一个标准流叫做cin,其接受用户的输入。的方向是“入”,关联源是“控制台”。与cout一样,cin中的c代表“字符”。cout和cin都是std命名空间中可用的流的预定义实例。下表简要描述了<iostream>中定义的所有预定义流。

流可以是缓冲的或无缓冲的。区别在于,缓冲流不会立即将数据发送到目的地。相反,会缓冲流会收集传入数据,然后以块的形式发送。另一方面,无缓冲流会立即将数据发送到目的地。缓冲通常是为了提高性能,因为某些目的地,如文件,在一次性写入更大的块时性能更好。注意,可以通过使用flush()成员函数刷新其缓冲区,始终强制缓冲流将其当前所有缓冲的数据发送到目的地。缓冲和刷新将在本章稍后更详细地进行介绍。

% Please add the following required packages to your document preamble:
% \usepackage{longtable}
% Note: It may be necessary to compile the document several times to get a multi-page table to line up properly
\begin{longtable}{|l|l|}
\hline
\textbf{流} & \textbf{描述}                                          \\ \hline
\endfirsthead
%
\endhead
%
cin             & 输入流,从“输入控制台”读取数据。          \\ \hline
cout            & 缓冲的输出流,将数据写入“输出控制台”。
 \\ \hline
cerr & \begin{tabular}[c]{@{}l@{}}无缓冲的输出流,将数据写入“错误控制台”,与“输出控制台”相同。\end{tabular} \\ \hline
clog            & cerr的缓冲版本。                                    \\ \hline
\end{longtable}

第1章中,std::print()和println()默认打印到cout,但想要打印到不同的流,可以将一个流作为这些函数的第一个参数:

\begin{cpp}
println(cerr, "This is an error printed to cerr.");
\end{cpp}

这些流还有宽字符、wchar\_t版本的,名称以w开头:wcin、wcout、wcerr和wclog。宽字符可以用来处理有更多字符的语言,比如中文。宽字符将在第21章中介绍。

\begin{myNotic}{NOTE}
每个输入流都有一个关联的源,每个输出流都有一个关联的目的地。
\end{myNotic}

流不仅包含数据,还具有当前位置。当前位置是流中下一个读或写操作将发生的位置。

\begin{myNotic}{NOTE}
图形界面应用通常没有控制台;如果写入cout,用户将看不到它。如果正在编写一个库,永远不应该假设cout、cin、cerr或clog的存在,永远不知道这个库是否会在控制台或GUI应用程序中使用。
\end{myNotic}

\mySubsubsection{13.1.2}{流的源和目的地}

流作为一个概念,可用于任受数据或发出数据的对象。可以编写一个基于流的网络类,或者基于流的方式访问音乐乐器数字接口(MIDI)乐器。在C++中,流有四个常见的源和目的地:控制台、文件、字符串和固定缓冲区数组,固定缓冲区数组支持在C++23中加入。

我们已经了解了许多用户或控制台流的例子。控制台输入流通过允许运行时用户输入,使程序具有交互性。控制台输出流向用户提供反馈并输出结果。

文件流,从文件系统读取数据并写入数据。文件输入流对于读取配置数据、保存的文件或批量处理基于文件的数据很有用。文件输出流对于保存状态和提供输出很有用。如果熟悉C风格的输入和输出,则文件流包含了C函数fprintf()、fwrite()和fputs()的输出功能,以及fscanf()、fread()和fgets()的输入功能。由于这些C风格函数在C++中不推荐使用,所以在这里就不多说了。

字符串流是将流的概念应用于字符串类型。使用字符串流,可以像处理其他流一样处理字符数据。大多数情况下,这是可以通过字符串类的成员函数处理的功能的便捷语法,而使用流语法回提供了优化的机会,并且通常比直接使用字符串类更方便、更高效。字符串流包含了sprintf()、sprintf\_s()、sscanf()和其他形式的C风格字符串格式化函数的功能,这在本书中不再讨论。

与固定缓冲区数组一起工作的流,允许使用流的概念处理内存块,而不管该缓冲区的内存如何分配。

本节的其余部分讨论控制台流(cin和cout)。本章稍后提供了文件、字符串和固定缓冲区数组流的示例。其他类型的流,如打印机输出或网络I/O,通常是平台相关的,所以本书不涉及这些内容。

\mySubsubsection{13.1.3}{使用输出流}

本节将简要回顾一些输出流的基础知识,并引入更高级内容。

\mySamllsection{基本使用}

输出流在<ostream>中定义。还有<iostream>,包括了输入流和输出流的功能。<iostream>还声明了所有预定义的流实例:cout、cin、cerr、clog,以及宽字符版本。

<{}<运算符是使用输出流的最简单方式。C++的基本类型,如int、指针、double和字符,都可以使用<{}<输出。C++的string类与<{}<兼容,C风格字符串也能正确输出。以下是一些使用<{}<的示例:

\begin{cpp}
int i { 7 };
cout << i << endl;

char ch { 'a' };
cout << ch << endl;

string myString { "Hello World." };
cout << myString << endl;
\end{cpp}

输出结果如下:

\begin{shell}
7
a
Hello World.
\end{shell}

cout流是用于写入控制台或标准输出的内置流,可以“链式”使用<{}<来输出多个数据项。因为operator<{}<返回流的一个引用作为结果,所以可以立即在同一流上再次使用<{}<:

\begin{cpp}
int j { 11 };
cout << "The value of j is " << j << "!" << endl;
\end{cpp}

输出结果如下:

\begin{shell}
The value of j is 11!
\end{shell}

C++流能正确解析C风格转义序列,如包含\verb|\|n的字符串。还可以使用std::endl开始新的一行。使用\verb|\|n和endl的区别在于,\verb|\|n只是开始新的一行,而endl还会刷新缓冲区。注意endl,过多的刷新可能会影响性能。以下示例使用endl输出并刷新多行文本,仅用一行代码:

\begin{cpp}
cout << "Line 1" << endl << "Line 2" << endl << "Line 3" << endl;
\end{cpp}

输出结果为:

\begin{shell}
Line 1
Line 2
Line 3
\end{shell}

\begin{myWarning}{WARNING}
endl刷新目标缓冲区,在性能关键代码中要谨慎使用,例如:在紧凑的循环中。
\end{myWarning}

\mySamllsection{输出流的成员函数}

<{}<运算符无疑是输出流中最有用的部分,还有其他功能值得探索。查看<ostream>中的内容,会看到许多重载的<{}<运算符的定义,支持输出各种不同的数据类型,还会发现一些有用的public成员函数。

\mySamllsection{put()和write()}

put()和write()是原始输出成员函数。put()接受单个字符,而write()接受字符数组。传递给这些成员函数的数据将原样输出,没有经过特殊格式化或处理。以下代码段显示了如何不使用<{}<,将C风格字符串输出到控制台:

\begin{cpp}
const char* test { "hello there" };
cout.write(test, strlen(test));
\end{cpp}

下一个代码段展示了,如何使用put()成员函数向控制台写入单个字符:

\begin{cpp}
cout.put('a');
\end{cpp}

\mySamllsection{flush()}

当写入输出流时,流不一定会立即将数据写入其目标。大多数输出流缓冲或积累数据,为了提高性能,不是一有数据就立即写入。某些流目标,如文件,如果数据以较大的块写入,而不是例如逐个字符写入,性能会更高。当出现以下情况时,流会刷新或写出积累的数据:

\begin{itemize}
\item
遇到endl操纵符。

\item
流超出作用域并销毁。

\item
流缓冲区已满。

\item
明确告诉流,刷新其缓冲区。

\item
从相应的输入流请求输入(即,当你使用cin进行输入时,cout将刷新)。在“文件流”一节中,将学习如何与这种类型建立链接。
\end{itemize}

告诉流刷新的一种方法是调用其flush()成员函数:

\begin{cpp}
cout << "abc";
cout.flush(); // abc is written to the console.
cout << "def";
cout << endl; // def is written to the console.
\end{cpp}

\begin{myNotic}{NOTE}
并非所有输出流都是缓冲的。例如,cerr流不缓冲其输出。
\end{myNotic}

\mySamllsection{处理输出错误}

输出错误可能在各种情况下出现,也许是试图打开一个不存在的文件,也许磁盘错误阻止了写操作的成功,例如:磁盘已满。目前,看到的流的代码都没有考虑这些可能性,主要是为了简单,处理发生的错误也至关重要。

当流处于正常可用状态时,认为是“良好”的。可通过good()成员函数,以确定流当前是否良好:

\begin{cpp}
if (cout.good()) {
    cout << "All good" << endl;
}
\end{cpp}

good()提供了一种获取关于流基本信息的简单方法,但它没有说为什么流不可用。有一个名为bad()的成员函数提供了更多信息。如果bad()返回true,则发生了致命错误(与非致命条件如文件结束,eof()相对)。另一个成员函数fail(),如果最近的操作失败则返回true,但它并不清楚下一个操作的情况,下一个操作也可能成功或失败。例如,输出流上调用flush()之后,可以调用fail()以确保成功刷新:

\begin{cpp}
cout.flush();
if (cout.fail()) {
    cerr << "Unable to flush to standard out" << endl;
}
\end{cpp}

流有一个转换为bool类型的转换运算符,这个转换运算符返回的值与调用!fail()相同,前面的代码段可以重写为:

\begin{cpp}
cout.flush();
if (!cout) {
    cerr << "Unable to flush to standard out" << endl;
}
\end{cpp}

重要的是,如果达到文件结束,good()和fail()都返回false。关系如下:good() == (!fail() \&\& !eof())。

还可以告诉流在发生失败时抛出异常,然后编写一个捕获处理程序来捕获ios\_base::failure异常,可以使用what()成员函数来获取错误的描述,使用code()成员函数来获取错误代码。然而,是否能得到有用的信息,取决于标准库的实现。

\begin{cpp}
cout.exceptions(ios::failbit | ios::badbit | ios::eofbit);
try {
    cout << "Hello World." << endl;
} catch (const ios_base::failure& ex) {
    cerr << "Caught exception: " << ex.what()
    << ", error code = " << ex.code() << endl;
}
\end{cpp}

使用clear()重置流的错误状态:

\begin{cpp}
cout.clear();
\end{cpp}

与文件输出或输入流相比,控制台输出流的错误检查较少。这里讨论的成员函数,也适用于其他类型的流。

\mySamllsection{输出操作符}

流的一个特性是,不仅只是将数据投入通道。C++流还识别操作符,这些对象改变流的行为,而不是为流提供数据。

已经见到过了一个操作符:endl操作符,封装了数据和行为,告诉流输出一个行结束序列,并刷新其缓冲区。以下是一些其他有用的操作符的非详尽列表,其中许多定义在<ios>和<iomanip>中。这里,展示了如何使用它们:

\begin{itemize}
\item
boolalpha和noboolalpha:告诉流以true和false(boolalpha)或1和0(noboolalpha)输出布尔值。默认是noboolalpha。

\item
hex, oct和dec: 分别以十六进制、八进制和十进制输出数字。

\item
fixed, scientific和defaultfloat: 分别使用定点、科学或默认格式输出小数。

\item
setprecision: 设置使用fixed或scientific格式输出小数时的十进制位数,或者输出的总数字位数,是一个带参数的操作符(接受一个参数)。

\item
setw: 设置输出数据的字段宽度,是一个带参数的操作符。

\item
setfill: 设置流的新的填充字符,填充字符可根据setw设置的宽度填充输出,是一个带参数的操作符。

\item
showpoint和noshowpoint: 强制流不显示没有小数部分的浮点数的十进制点。

\item
put\_money: 一个带参数的操作符,将格式化的货币值写入流。

\item
put\_time: 一个带参数的操作符,将格式化的时间写入流。

\item
quoted: 一个带参数的操作符,用引号括起给定的字符串,并转义嵌入的引号。
\end{itemize}

这些操作符在流中后续的输出中保持有效,直到重置(除了setw),只对下一个单独的输出有效。以下示例,使用这些操作符中的几个来自定义输出:

\begin{cpp}
// Boolean values
bool myBool { true };
cout << "This is the default: " << myBool << endl;
cout << "This should be true: " << boolalpha << myBool << endl;
cout << "This should be 1: " << noboolalpha << myBool << endl;

// Simulate println-style "{:6}" with streams
int i { 123 };
println("This should be ' 123': {:6}", i);
cout << "This should be ' 123': " << setw(6) << i << endl;

// Simulate println-style "{:0>6}" with streams
println("This should be '000123': {:0>6}", i);
cout << "This should be '000123': " << setfill('0') << setw(6) << i << endl;

// Fill with *
cout << "This should be '***123': " << setfill('*') << setw(6) << i << endl;
// Reset fill character
cout << setfill(' ');

// Floating-point values
double dbl { 1.452 };
double dbl2 { 5 };
cout << "This should be ' 5': " << setw(2) << noshowpoint << dbl2 << endl;
cout << "This should be @@1.452: " << setw(7) << setfill('@') << dbl << endl;
// Reset fill character
cout << setfill(' ');

// Instructs cout to start formatting numbers according to your location.
// Chapter 21 explains the details of the imbue() call and the locale object.
cout.imbue(locale { "" });

// Format numbers according to your location
cout << "This is 1234567 formatted according to your location: " << 1234567
     << endl;

// Monetary value. What exactly a monetary value means depends on your
// location. For example, in the USA, a monetary value of 120000 means 120000
// dollar cents, which is 1200.00 dollars.
cout << "This should be a monetary value of 120000, "
     << "formatted according to your location: "
     << put_money("120000") << endl;

// Date and time
time_t t_t { time(nullptr) }; // Get current system time.
tm t { *localtime(&t_t) }; // Convert to local time.
cout << "This should be the current date and time "
     << "formatted according to your location: "
     << put_time(&t, "%c") << endl;

// Quoted string
cout << "This should be: \"Quoted string with \\\"embedded quotes\\\".\": "
     << quoted("Quoted string with \"embedded quotes\".") << endl;
\end{cpp}

\begin{myNotic}{NOTE}
使用Microsoft Visual C++时,这个示例在调用localtime()时可能会给出一个与安全性相关的错误或警告。可以切换到使用localtime\_s(),或使用\#pragma指令禁用这个警告。
\end{myNotic}

如果不喜欢操作符,也可以不使用。流的成员函数(如precision()),也提供了许多相同的功能:

\begin{cpp}
cout << "This should be '1.2346': " << setprecision(5) << 1.23456789 << endl;
\end{cpp}

可以转换为使用成员函数调用来实现。使用成员函数调用的优点是会返回之前的值,可以进行恢复。

\begin{cpp}
cout.precision(5);
cout << "This should be '1.2346': " << 1.23456789 << endl;
\end{cpp}

有关所有流成员函数和操作符的详细描述,请参阅标准库手册。

\mySubsubsection{13.1.4}{使用输入流}

输入流提供了一种简单的方式,来读取结构化或非结构化数据。本节中,将在cin的上下文中讨论输入技术,cin是控制台输入流。

\mySamllsection{基本使用}

使用输入流读取数据有两种简单方法。第一种是输出数据到输出流的<{}<运算符,相应的读取数据的运算符是>{}>。当使用>{}>从输入流读取数据时,提供的变量是接收值的存储位置。以下程序从用户读取一个单词,并将其放入一个字符串中,然后字符串输出到控制台:

\begin{cpp}
string userInput;
cin >> userInput;
println("User input was {}.", userInput);
\end{cpp}

通常,>{}>运算符根据空白字符对值进行标记化。如果用户运行上一个程序并输入hello there作为输入,只有第一个空白字符(在此例中为空格字符)之前的字符会捕获存储到userInput变量中。输出如下所示:

\begin{shell}
User input was hello.
\end{shell}

要在输入中包含空白字符,可以使用get()。

\begin{myNotic}{NOTE}
C++中的空白字符包括空格(' ')、换页('\verb|\|f')、换行('\verb|\|n')、回车('\verb|\|r')、水平制表符('\verb|\|t')和垂直制表符('\verb|\|v')。
\end{myNotic}

>{}>运算符适用于不同的变量类型,就像<{}<运算符一样。要读取一个整数的话,代码只会在变量类型上有所不同:

\begin{cpp}
int userInput;
cin >> userInput;
println("User input was {}.", userInput);
\end{cpp}

可以使用输入流读取多个值,根据需要混合和匹配类型。以下函数,摘自一个餐厅预订系统,询问用户的姓氏和派对的人数:

\begin{cpp}
void getReservationData()
{
    string guestName;
    int partySize;
    print("Name and number of guests: ");
    cin >> guestName >> partySize;
    println("Thank you, {}.", guestName);
    if (partySize > 10) {
        println("An extra gratuity will apply.");
    }
}
\end{cpp}

>{}>运算符根据空白字符对值进行标记化,所以getReservationData()不允许输入带有空白字符的名字。本章后面将讨论使用unget()的解决方案。还请注意,第一次使用cout时没有使用endl或flush()显式刷新缓冲区,但文本仍然会写入控制台,因为使用cin会立即刷新cout缓冲区(以这种方式链接在一起)。

\begin{myNotic}{NOTE}
如果对于<{}<和>{}>之间感到困惑,只需考虑角度指向其目的地即可。输出流中,<{}<指向流本身,因为正在发送数据到流中。输入流中,>{}>指向变量,因为正在存储数据。
\end{myNotic}

\mySamllsection{处理输入错误}

输入流有一系列成员函数来检测不正常的情况。与输入流相关的错误条件大多数发生在没有数据可读取时,例如:流已达到其末尾(即使对于非文件流也称为文件结束)。查询输入流状态最常见的方法是在条件语句中访问,例如:以下循环会持续循环,只要cin处于良好状态。这种模式利用了在条件上下文中评估输入流会导致错误的情况。所以遇到错误时,流会评估为false。实现这种行为的底层转换操作细节,将在第15章中介绍。

\begin{cpp}
while (cin) { ... }
\end{cpp}

可以在同时输入数据:

\begin{cpp}
while (cin >> ch) { ... }
\end{cpp}

good()、bad()和 fail() 成员函数可以对输入流进行调用,就像对输出流一样。还有一个名为eof()的成员函数,当流达到其末尾时返回true。与输出流类似,good()和fail()在达到文件结束时都返回false。关系如下:good() == (!fail() \&\& !eof())。

应该在阅读数据后检查流的状态,以便从错误的输入中恢复。

以下程序展示了从流中读取数据并处理错误的常见模式。该程序从标准输入读取数字,并在达到文件结束时显示它们的和。命令行环境中,文件结束是由用户输入特定字符来指定。在Unix和Linux中,是Control+D;在Windows中是Control+Z,然后是Enter。确切的字符取决于操作系统,所以需要了解正在使用的操作系统。

\begin{cpp}
println("Enter numbers on separate lines to add.");
println("Use Control+D followed by Enter to finish (Control+Z in Windows).");
int sum { 0 };

if (!cin.good()) {
    println(cerr, "Standard input is in a bad state!");
    return 1;
}

while (!cin.bad()) {
    int number;
    cin >> number;
    if (cin.good()) {
        sum += number;
    } else if (cin.eof()) {
        break; // Reached end-of-file.
    } else if (cin.fail()) {
        // Failure!
        cin.clear(); // Clear the failure state.
        string badToken;
        cin >> badToken; // Consume the bad input.
        println(cerr, "WARNING: Bad input encountered: {}", badToken);
    }
}
println("The sum is {}.", sum);
\end{cpp}

这里是在此程序上的示例输出。输出中的\^{}Z字符是在按下Control+Z时出现的。

\begin{shell}
Enter numbers on separate lines to add.
Use Control+D followed by Enter to finish (Control+Z in Windows).
12
test
WARNING: Bad input encountered: test
3
^Z
The sum is 6.
\end{shell}

\mySamllsection{输入流的成员函数}

与输出流一样,输入流有几种成员函数,允许使用更常见的 >{}> 运算符提供的功能更底层的访问。

\mySamllsection{get()}

get()成员函数允许从流中进行原始输入数据。get()的最简单版本返回流中的下一个字符,尽管存在其他版本,可以一次读取多个字符。get()最常用于避免与 >{}> 运算符自动进行分词,以下函数从输入流中读取一个名字。该名字可以由几个单词组成,直到流结束为止:

\begin{cpp}
string readName(istream& stream)
{
    string name;
    while (stream) { // Or: while (!stream.fail()) {
        int next { stream.get() };
        if (!stream || next == std::char_traits<char>::eof())
        break;
        name += static_cast<char>(next);// Append character.
    }
    return name;
}
\end{cpp}

这个readName()函数,有一些有趣的观察:

\begin{itemize}
\item
其参数是一个对istream的引用为非常量,而不是引用为常量。从流中读取数据的成员函数将改变实际的流(尤其是其位置),所以不是常量成员函数,不能在引用为常量上调用它们。

\item
因为get()可以返回特殊非字符值,所以get()的返回值存储在int中,而不是char中,如std::char\_traits::eof()(文件结束)。

\item
get()读取的新行和其他转义字符,将出现在由readName()返回的字符串中。如果不在行的开头按下Ctrl+D或Ctrl+Z,也会出现在返回的字符串中。
\end{itemize}

readName()有点奇怪,因为退出循环有两种方式:要么流进入失败状态,要么达到流的末尾。更常见的从流中读取数据的模式使用了一个不同的get()版本,该版本接受一个字符的引用,并返回流的引用。在条件上下文中计算输入流,当流不在错误状态下时才为true。以下是相同函数更简洁的版本:

\begin{cpp}
string readName(istream& stream)
{
    string name;
    char next;
    while (stream.get(next)) {
        name += next;
    }
    return name;
}
\end{cpp}

\mySamllsection{unget()}

正确地看待输入流的方式是将其视为单向通道,数据沿着通道落入变量。unget()成员函数以一种方式打破了这种模型,允许将数据推回通道。

调用unget()会使流回退一个位置,实质上将之前读取的字符放回流中,可以使用fail()成员函数来查看unget()是否成功。如果当前位置在流的开始,unget()可能会失败。

本章前面显示的getReservationData()函数,不允许输入带有空白字符的名字。以下代码使用unget()允许名字中包含空白字符。代码逐个字符读取,并检查字符是否为数字。如果字符不是数字,会添加到guestName中。如果是数字,字符会使用unget()放回流中,循环停止,并使用>{}>运算符输入整数partySize。noskipws输入操纵器告诉流不要跳过空白字符,所以空白字符可以像其他字符一样读取。

\begin{cpp}
void getReservationData()
{
    print("Name and number of guests: ");
    string guestName;
    int partySize { 0 };
    // Read characters until we find a digit
    char ch;
    cin >> noskipws;
    while (cin >> ch) {
        if (isdigit(ch)) {
            cin.unget();
            if (cin.fail()) { println(cerr, "unget() failed."); }
            break;
        }
        guestName += ch;
    }
    // Read partySize, if the stream is not in error state
    if (cin) { cin >> partySize; }
    if (!cin) {
        println(cerr, "Error getting party size.");
        return;
    }

    println("Thank you '{}', party of {}.", guestName, partySize);
    if (partySize > 10) {
        println("An extra gratuity will apply.");
    }
}
\end{cpp}


\mySamllsection{putback()}

putback()成员函数与unget()类似,允许在输入流中回退一个字符。不同之处在于,putback()将放回流中的字符作为参数:

\begin{cpp}
char c;
cin >> c;
println("Retrieved {}.", c);

cin.putback('e'); // 'e' will be the next character read off the stream.
println("Called putback('e').");

while (cin >> c) { println("Retrieved {}.", c); }
\end{cpp}

输出如下所示:

\begin{shell}
wow
Retrieved w.
Called putback('e').
Retrieved e.
Retrieved o.
Retrieved w.
\end{shell}

\mySamllsection{peek()}

peek()成员函数允许预览调用get()将返回的下一个值。可以将其视为向上看通道,但实际上没有值掉下来。

peek()适用于需要在读取值之前向前看的情况。以下代码实现了getReservationData(),允许名字中包含空白字符,这里会使用peek(),而不是unget():

\begin{cpp}
void getReservationData()
{
    print("Name and number of guests: ");
    string guestName;
    int partySize { 0 };
    // Read characters until we find a digit.
    cin >> noskipws;
    while (true) {
        // 'peek' at next character.
        char ch { static_cast<char>(cin.peek()) };
        if (!cin) { break; }
        if (isdigit(ch)) {
            // Next character will be a digit, so stop the loop.
            break;
        }
        // Next character will be a non-digit, so read it.
        cin >> ch;
        if (!cin) { break; }
        guestName += ch;
    }
    // Read partySize, if the stream is not in error state.
    if (cin) { cin >> partySize; }
    if (!cin) {
        println(cerr, "Error getting party size.");
        return;
    }

    println("Thank you '{}', party of {}.", guestName, partySize);
    if (partySize > 10) {
        println("An extra gratuity will apply.");
    }
}
\end{cpp}

\mySamllsection{getline()}

从输入流中获取单行数据非常常见,有一个成员函数可以完成这个任务。getline()成员函数用指定大小的字符缓冲区填充一行数据。这个指定的大小包括\verb|\|0字符。以下代码从cin读取最多BufferSize-1个字符,或者直到读取到行结束序列:

\begin{cpp}
char buffer[BufferSize] { 0 };
cin.getline(buffer, BufferSize);
\end{cpp}

当调用getline()时,其会从输入流中读取一行数据,直到包括行结束序列,但行结束字符或字符不会出现在字符串中。注意,行结束序列取决于平台,可以是\verb|\|r\verb|\|n、\verb|\|n或\verb|\|n\verb|\|r。

get()有一种形式,执行与getline()相同的功能,只保留输入流中新行的序列。

还有一个非成员函数std::getline(),可以用于C++字符串。定义在<string>中,属于std命名空间。使用这个版本的getline()的好处是,不需要指定缓冲区大小。

\begin{cpp}
string myString;
getline(cin, myString);
\end{cpp}

getline()成员函数和std::getline()函数,都接受一个可选的分隔符作为最后一个参数,默认分隔符是\verb|\|n。通过更改这个分隔符,这些函数可以用来读取直到遇到给定分隔符的多行文本。以下代码读取多行文本,直到@字符为止:

\begin{cpp}
print("Enter multiple lines of text. "
      "Use an @ character to signal the end of the text.\n> ");
string myString;
getline(cin, myString, '@');
println("Read text: \"{}\"", myString);
\end{cpp}

可能的输出如下所示:

\begin{shell}
Enter multiple lines of text. Use an @ character to signal the end of the text.
> This is some
text on multiple
lines.@
Read text: "This is some
text on multiple
lines."
\end{shell}


\mySamllsection{输入操作符}

与输出流一样,输入流也支持多种输入操作符。已经见过noskipws了,它告诉输入流在标记化时不要跳过空白字符。以下列表显示了其他可用的内置输入操作符,允许按自定义的方式读取数据:

\begin{itemize}
\item
boolalpha和noboolalpha: 如果使用boolalpha,字符串false将解释为布尔值false;其他内容都将视为布尔值true。如果设置noboolalpha,零将解释为false,其他内容为true。流中默认是noboolalpha。

\item
dec, hex和oct: 分别以十进制、十六进制或八进制表示读取数字。十进制基数10的数字207在十六进制中是cf,在八进制中是317。

\item
skipws和noskipws: 告诉流在标记化时跳过空白字符,还是将空白字符作为标记。流中默认是skipws。

\item
ws: 方便的操作符,简单地跳过流当前位置的空白字符。

\item
get\_money: 参数化操作符,从流中读取货币值。

\item
get\_time: 参数化操作符,从流中读取格式化时间。

\item
quoted: 参数化操作符,读取用引号包围的字符串,转义其中嵌入的引号。本章后面将展示这个操作符的例子。
\end{itemize}

输入可以是区域化的,以下代码使cin使用系统区域。区域化将在第21章中介绍:

\begin{cpp}
cin.imbue(locale { "" });
int i;
cin >> i;
\end{cpp}

如果你的系统区域设置是美式英语,可以输入1,000,程序会解析为1000。如果输入1.000,程序会解析为1。另外,如果系统区域设置是荷兰比利时,可以输入1.000来得到1000的值,但输入1,000会得到1。这两种情况下,也可以直接输入1000,没有数字分隔符,从而得到1000的值。

\mySubsubsection{13.1.5}{对象的输入和输出}

如果熟悉C语言中用于输出的旧式printf()函数,它不够灵活,也不支持自定义类型。printf()了解几种数据类型,但并没有办法扩展。例如,以下简单的类:

\begin{cpp}
class Muffin final
{
    public:
        const string& getDescription() const { return m_description; }
        void setDescription(string description)
        {
            m_description = std::move(description);
        }

        int getSize() const { return m_size; }
        void setSize(int size) { m_size = size; }

        bool hasChocolateChips() const { return m_hasChocolateChips; }
        void setHasChocolateChips(bool hasChips)
        {
            m_hasChocolateChips = hasChips;
        }
    private:
        string m_description;
        int m_size { 0 };
        bool m_hasChocolateChips { false };
};
\end{cpp}

使用printf()输出Muffin类对象时,如果能指定它作为一个参数,也许\%m可以作为一个很好的占位符:

\begin{cpp}
printf("Muffin: %m\n", myMuffin); // BUG! printf doesn't understand Muffin.
\end{cpp}

不幸的是,printf()函数对Muffin类型一无所知,无法输出Muffin类型对象。更糟糕的是,由于printf()函数的声明方式,这会导致运行时错误,而不是编译时错误(好的编译器会给出警告)。

用printf()的最好方式是,在Muffin类中添加一个新的output()成员函数:

\begin{cpp}
class Muffin final
{
    public:
    void output() const
    {
        printf("%s, size is %d, %s", getDescription().c_str(), getSize(),
            (hasChocolateChips() ? "has chips" : "no chips"));
    }
    // Omitted for brevity
};
\end{cpp}

要在另一行文本中间输出一个Muffin,需要将这行文本分成两个调用,并在两者之间调用Muffin::output():

\begin{cpp}
printf("The muffin is a ");
myMuffin.output();
printf(" -- yummy!\n");
\end{cpp}

更好的选项是,为Muffin对象编写自定义的std::formatter特化。以下是Muffins的简单自定义格式化器。为了保持示例简洁,这个格式化器不支持任何格式说明符,parse()简化的成员函数模板不需要解析,只需返回context的起始位置即可。

\begin{cpp}
template <>
class std::formatter<Muffin>
{
    public:
    constexpr auto parse(auto& context) { return begin(context); }

    auto format(const Muffin& muffin, auto& ctx) const
    {
        ctx.advance_to(format_to(ctx.out(), "{}, size is {}, {}",
            muffin.getDescription(), muffin.getSize(),
            (muffin.hasChocolateChips() ? "has chips" : "no chips")));
        return ctx.out();
    }
};
\end{cpp}

有了这个自定义格式化器,可以使用现代的std::print()和println()函数输出Muffin:

\begin{cpp}
println("The muffin is a {} -- yummy!", myMuffin);
\end{cpp}

打印Muffins的另一种选项是在<{}<运算符上重载,之后像输出字符串一样输出Muffin——只需将其作为<{}<运算符的参数提供。还可以重载>{}>运算符,以便可以从输入流中输入Muffins。第15章介绍了重载<{}<和>{}>运算符的详情。

\mySubsubsection{13.1.6}{自定义操作符}

标准库包含许多内置的流操作符,也可以编写自定义操作符。

编写自己的非参数化操作符很简单。这里有一个简单的示例,定义了一个制表符输出操作符,将制表符输出到给定的流中:

\begin{cpp}
ostream& tab(ostream& stream) { return stream << '\t'; }

int main()
{
    cout << "Test" << tab << "!" << endl;
}
\end{cpp}

编写自定义的参数化操作符很复杂,涉及到使用ios\_base的功能,如xalloc()、iword()、pword()和register\_callback()。由于这样的操作符很少见,本文不再进一步讨论。如果感兴趣,可以查阅标准库手册。



