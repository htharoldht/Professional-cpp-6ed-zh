\noindent
\textbf{内容概要}

\begin{itemize}
\item
流的概念

\item
数据流的输入和输出

\item
标准库中可用的标准流

\item
文件系统库
\end{itemize}

本章的所有代码示例都可以在\url{https://github.com/Professional-CPP/edition-6}获得。

程序的基本工作是接受输入并产生输出,不产生输出的程序毫无用处。所有语言都提供某种I/O机制,要么是语言内置的一部分,要么是通过特定的操作系统API。好的I/O系统既灵活又易于使用,灵活的I/O系统支持通过各种设备进行输入和输出,如文件和用户控制台。文件可以是标准文件,也可以是来自各种来源的数据,如物联网(IoT)设备、网络服务等。也可能是来自气象设备或股票经纪网络服务的数据,灵活的I/O系统还支持读取和写入不同类型的数据。I/O易于出错,因为用户输入的数据可能是错误的,或者底层文件系统或其他数据源可能无法访问,所以好的I/O系统也应该能够处理错误条件。

如果熟悉C语言,无疑使用过printf()和scanf()。作为I/O机制,printf()和scanf()当然很灵活。通过转义码和变量占位符(类似于std::format()、print()和println()中讨论的格式说明符和替换字段),可以定制为读取特殊格式的数据或输出任何格式代码允许的值。支持的类型仅限于整数/字符值、浮点值和字符串。printf()和scanf()在好I/O系统的其他方面表现不佳,特别不擅长处理错误。例如,告诉它们将浮点数解释为整数,它们会很乐意这样做。此外,它们不够灵活,无法处理自定义数据类型,而且不安全,而且在像C++这样的面向对象语言中,它们不是面向对象的。

C++提供了一种更精细、更灵活、更面向对象的I/O方法。流封装在类中,这形成了一个用户友好且安全的解决方案。本章中,将首先了解什么是流,然后学习如何使用流进行数据输出和输入。还将了解如何使用流机制从各种来源读取,并向各种目的地写入,如用户控制台、文件,甚至字符串。本章涵盖了最常用的I/O功能。

本书中几乎所有的示例都使用print()和println()将文本打印到用户控制台,另一种选择是使用本章中讨论的I/O流功能。我建议使用print()和println()代替标准输出的流,因为前者更容易阅读,更紧凑,性能更好。本章详细讨论了I/O流,因为它在C++中仍然很重要,我们肯定有需要处理使用I/O流的代码。

本章的最后部分讨论了C++标准库提供的文件系统库。这个库允许处理路径、目录和文件,很好地补充了流提供的I/O机制。










