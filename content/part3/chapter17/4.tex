

C++ 标准库的迭代器支持算法独立于实际容器工作,其抽象了遍历容器元素的方法,大多数算法需要一个迭代器对,包括一个指向序列中第一个元素的 begin 迭代器,以及一个指向最后一个元素之后的 end 迭代器。这使得算法能够处理各种容器,但始终需要提供两个迭代器来指定一个元素序列,并确保提供匹配的迭代器。范围库提供的范围是对迭代器之上的抽象层,消除了不匹配迭代器的错误,并添加了其他功能,例如允许范围适配器延迟过滤和转换底层元素序列。范围库,定义在 <ranges> 中,由以下组件组成:

\begin{itemize}
\item
范围:范围是一个概念(参见第 12 章),定义了一个类型,允许迭代其元素。支持 begin() 和 end() 的数据结构都是有效的范围。例如,std::array、vector、string\_view、span、固定大小的 C 风格数组等,都是有效的范围。

\item
受约束的算法:第 16 章和第 20 章讨论了接受迭代器对的可用标准库算法,以执行其工作。对于这些算法中的大多数,都有等效的范围基础和受约束的变体,接受迭代器对或范围。

\item
投射:许多受约束的算法接受一个投射回调。这个回调对范围内的每个元素进行调用,并可以将一个元素转换为另一个值,然后再传递给算法。

\item
视图:视图可以用来转换或过滤底层范围中的元素。视图可以组合在一起,形成应用于范围的操作流水线。

\item
工厂:范围工厂用于构造生成值的视图。
\end{itemize}

对范围内的元素进行迭代可以使用可以从访问器(如 ranges::begin()、end()、rbegin() 等)检索到的迭代器。范围还支持 ranges::empty()、data()、cdata() 和 size()。后者返回范围内的元素数量,但仅在可以以常数时间检索大小时才有效。否则,使用 std::distance() 计算范围内的 begin 和 end 迭代器之间的元素数量。这些访问器都不是成员函数,而是独立函数,需要一个范围作为参数。

此外,std::format()、print() 和 println() 支持格式化和打印范围,这一点在本节中会通过许多示例进行展示。

\mySubsubsection{17.4.1}{受约束的算法}

std::sort() 算法是需要用 begin 和 end 迭代器指定元素序列算法的例子。算法在第 16 章介绍,在第 20 章详细讨论。sort() 算法使用起来很简单。以下代码是对一个 vector 中的所有元素进行排序:

\begin{cpp}
vector data { 33, 11, 22 };
sort(begin(data), end(data));
\end{cpp}

这段代码对 data 容器中的所有元素进行排序,但必须指定一个 begin/end 迭代器对。代码中准确地描述想做什么岂不更好?这就是范围基础和受约束的算法(本书中简单称为受约束的算法)发挥作用的地方。这些算法位于 std::ranges 命名空间中,并在相应的无约束版本的同一头文件中定义,可以简单地写为:

\begin{cpp}
ranges::sort(data);
\end{cpp}

这段代码清楚地描述了意图,即对 data 容器中的所有元素进行排序。由于不再指定迭代器,这些受约束的算法消除了提供不匹配 begin 和 end 迭代器的可能性。受约束的算法对其模板类型参数有适当的类型约束(参见第 12 章),允许编译器在向需要特定类型迭代器的算法提供不正确的容器时,提供更清晰的错误信息。例如,对一个 std::list 调用 ranges::sort() 算法会导致编译器错误,明确指出 sort() 需要一个随机访问范围,而 list 不提供。类似于迭代器,有输入、输出、前向、双向、随机访问和连续范围,以及相应的概念,如 ranges::contiguous\_range、ranges::random\_access\_range 等。

\begin{myNotic}{NOTE}
大多数标准库算法(在第 16 章介绍,在第 20 章详细介绍),在 std::ranges 命名空间中都有受约束的等价版本。如果可能的话,建议使用这些受约束的算法,因为有类型约束,所以编译器可以在使用不正确类型时提供更可读的错误信息。
\end{myNotic}

\mySamllsection{投射}

许多受限算法都有一个投射参数,是一个回调函数,用于在将每个元素传递给算法之前对其进行转换。假设一个简单的类:

\begin{cpp}
class Person
{
    public:
        explicit Person(string first, string last)
            : m_firstName { move(first) }, m_lastName { move(last) } { }
        const string& getFirstName() const { return m_firstName; }
        const string& getLastName() const { return m_lastName; }
    private:
        string m_firstName;
        string m_lastName;
};
\end{cpp}

以下代码在vector中存储了几个Person对象:

\begin{cpp}
vector persons { Person {"John", "White"}, Person {"Chris", "Blue"} };
\end{cpp}

由于 Person 类没有实现 operator<,所以不能使用普通的 std::sort() 算法来排序这个vector。所以,下面的代码不会编译:

\begin{cpp}
sort(begin(persons), end(persons)); // Error: does not compile.
\end{cpp}

切换到有限制的ranges::sort() 算法也没用。下面的代码仍然不会编译,算法仍然不知道如何比较范围中的元素:

\begin{cpp}
ranges::sort(persons); // Error: does not compile.
\end{cpp}

然而,可以根据他们的名字对Person排序。指定一个投射函数给 sort 算法,用于将每个 Person 映射到他们的名字。投射参数是第三个,所以还需要指定第二个参数,即使用的比较器,默认是 std::ranges::less。在下面的调用中,\{\} 指定了使用默认比较器,投射函数是一个 Lambda 表达式。

\begin{cpp}
ranges::sort(persons, {},
    [](const Person& person) { return person.getFirstName(); });
\end{cpp}

或者更简单:

\begin{cpp}
ranges::sort(persons, {}, &Person::getFirstName);
\end{cpp}

\begin{myNotic}{NOTE}
本章中关于范围的讨论使用了几个基本的 Lambda 表达式。Lambda 表达式在第19章中有详细讨论,但所有的细节对于当前的讨论来说并不重要。现在,只需知道基本用法就足够了。一个 Lambda 表达式,如本章所使用,具有以下语法如下所示:

\begin{cpp}
[](const Person& person) { return person.getFirstName(); }
\end{cpp}

方括号 [] 标志着 Lambda 表达式的开始。接下来是一个逗号分隔的参数列表,与函数的参数列表类似。最后,Lambda 表达式的主体使用一对花括号括起来。

基本上,Lambda 表达式允许在需要的地方编写小型的、未命名的内联函数。前面的 Lambda 表达式可以用以下独立函数替换:

\begin{cpp}
auto getFirstName(const Person& person) {
    return person.getFirstName(); }
\end{cpp}

Lambda 表达式参数的类型也可以是 auto:

\begin{cpp}
[](const auto& person) { return person.getFirstName(); }
\end{cpp}
\end{myNotic}

\mySubsubsection{17.4.2}{视图}

视图允许对底层范围的元素执行操作,如过滤和转换。视图可以组合在一起形成一个管道,对范围的元素执行多个操作。组合视图很容易,只需使用按位或操作符 | 组合不同的操作。可以先过滤一个范围的元素,然后转换剩余的元素。相比之下,如果想用不受限制的算法做类似的事情,即先过滤后转换,需要创建临时容器来存储中间结果,代码将难以理解,性能也可能很差。

视图具有以下重要属性:

\begin{itemize}
\item
延迟计算:仅构建视图并不会执行任何操作,视图操作仅在迭代视图的元素并解引用这样的迭代器时应用。

\item
非拥有视图[C++23 对视图的定义进行了轻微修改,当它不可复制,或者在常数时间内(O(1))可复制时,允许视图拥有其元素。大多数视图将是非拥有的,因此拥有视图在本文本中不再进一步讨论。]: 视图不拥有元素,是一个对存储在某些容器中的范围的元素的视图,并且是那个容器拥有数据。视图只是允许以不同的方式查看数据。视图中的元素数量不会影响视图的复制、移动或销毁的开销。这与第2章中的 std::string\_view 和第18章中的 std::span 类似。

\item
非修改:视图永远不会修改底层范围中的数据。
\end{itemize}

视图本身也是一个范围,但并非每个范围都是视图。容器是一个范围,但不是视图,因为它拥有元素。

可以使用范围适配器创建视图。范围适配器接受一个元素序列,以及一些可选参数,并创建一个新的视图。以下表格列出了标准库提供的范围适配器。如果标准库适配器中没有适合你需求的,可以编写自己的范围适配器,这些适配器需要与现有的适配器正确互操作。然而,编写一个完整的、高质量的、可生产的范围适配器并不简单,而且超出了本书的范围,想要获得更多信息,请参阅标准库手册。

% Please add the following required packages to your document preamble:
% \usepackage{longtable}
% Note: It may be necessary to compile the document several times to get a multi-page table to line up properly
\begin{longtable}{|l|l|}
\hline
\textbf{范围适配器} &
\textbf{描述} \\ \hline
\endfirsthead
%
\endhead
%
views::all &
创建包含范围中所有元素的视图。
 \\ \hline
\begin{tabular}[c]{@{}l@{}}filter\_view\\ views::filter\end{tabular} &
\begin{tabular}[c]{@{}l@{}}根据给定的谓词过滤底层序列的元素。如果谓词返回 true,则保留该元素,\\否则跳过。
\end{tabular} \\ \hline
\begin{tabular}[c]{@{}l@{}}transform\_view\\ views::transform\end{tabular} &
\begin{tabular}[c]{@{}l@{}}对底层序列的每个元素应用回调,以转换元素为其他值,可能是不同类型。
\end{tabular} \\ \hline
\begin{tabular}[c]{@{}l@{}}take\_view\\ views::take\end{tabular} &
创建另一个视图的前 n 个元素的视图。
 \\ \hline
\begin{tabular}[c]{@{}l@{}}take\_while\_view\\ views::take\_while\end{tabular} &
\begin{tabular}[c]{@{}l@{}}创建一个视图,该视图包含底层序列的初始元素,直到达到一个元素,对于\\该元素,给定的谓词返回 false。
\end{tabular} \\ \hline
\begin{tabular}[c]{@{}l@{}}drop\_view\\ views::drop\end{tabular} &
通过丢弃另一个视图的前 n 个元素来创建一个视图。
 \\ \hline
\begin{tabular}[c]{@{}l@{}}drop\_while\_view\\ views::drop\_while\end{tabular} &
\begin{tabular}[c]{@{}l@{}}通过丢弃底层序列的所有初始元素,直到达到一个元素,对于该元素,给定\\的谓词返回 false,来创建一个视图。
\end{tabular} \\ \hline
\begin{tabular}[c]{@{}l@{}}join\_view\\ views::join\end{tabular} &
\begin{tabular}[c]{@{}l@{}}将一个范围视图展开成一个视图。例如,将一个 vector<vector<int>{}>\\展开成一个 vector<int>。
\end{tabular} \\ \hline
\begin{tabular}[c]{@{}l@{}}lazy\_split\_view\\ views::lazy\_split\\ split\_view\\ views::split\end{tabular} &
\begin{tabular}[c]{@{}l@{}}给定分隔符,将给定视图按分隔符分割成子范围。分隔符可以是单个元素或\\元素视图。
\end{tabular} \\ \hline
\begin{tabular}[c]{@{}l@{}}reverse\_view\\ views::reverse\end{tabular} &
\begin{tabular}[c]{@{}l@{}}创建一个视图,以相反的顺序遍历另一个视图的元素,所以该视图必须是双\\向视图。
\end{tabular} \\ \hline
\begin{tabular}[c]{@{}l@{}}elements\_view\\ views::elements\end{tabular} &
\begin{tabular}[c]{@{}l@{}}需要一个元组样式的元素视图,创建一个元组样式元素的第 n 个元素的视图。
\end{tabular} \\ \hline
\begin{tabular}[c]{@{}l@{}}keys\_view\\ views::keys\end{tabular} &
\begin{tabular}[c]{@{}l@{}}需要一个类似对的元素视图,创建每个对的第一元素的视图。
\end{tabular} \\ \hline
\begin{tabular}[c]{@{}l@{}}values\_view\\ views::values\end{tabular} &
\begin{tabular}[c]{@{}l@{}}需要一个类似对的元素视图,创建每个对的第二个元素的视图。
\end{tabular} \\ \hline
\begin{tabular}[c]{@{}l@{}}common\_view\\ views::common\end{tabular} &
\begin{tabular}[c]{@{}l@{}}根据范围的类型,begin() 和 end() 可能返回不同类型的对象,例如一个\\begin 迭代器和 end 哨兵,所以不能将这样的迭代器对传递给期望具有相\\同类型的函数。common\_view 可以用来将这样的范围转换为一个 common \\范围,这是一个 begin() 和 end() 返回相同类型的范围。我们将在练习\\中使用这个范围适配器。
\end{tabular} \\ \hline
\end{longtable}

C++23添加了下面的范围适配器:

% Please add the following required packages to your document preamble:
% \usepackage{longtable}
% Note: It may be necessary to compile the document several times to get a multi-page table to line up properly
\begin{longtable}{|l|l|}
\hline
\textbf{范围适配器} &
\textbf{描述} \\ \hline
\endfirsthead
%
\endhead
%
\begin{tabular}[c]{@{}l@{}}as\_const\_view\\ views::as\_const\end{tabular} &
\begin{tabular}[c]{@{}l@{}}创建一个视图,这个视图不能修改底层序列的元素。
\end{tabular} \\ \hline
\begin{tabular}[c]{@{}l@{}}as\_rvalue\_view\\ views::as\_rvalue\end{tabular} &
\begin{tabular}[c]{@{}l@{}}创建一个视图,包含底层序列所有元素的右值。
\end{tabular} \\ \hline
\begin{tabular}[c]{@{}l@{}}enumerate\_view\\ views::enumerate\end{tabular} &
\begin{tabular}[c]{@{}l@{}}创建一个视图,其中每个元素代表底层序列所有元素的位置和值。
\end{tabular} \\ \hline
\begin{tabular}[c]{@{}l@{}}zip\_view\\ views::zip\end{tabular} &
\begin{tabular}[c]{@{}l@{}}创建一个视图,由所有给定视图对应元素的引用组成的元组。
\end{tabular} \\ \hline
\begin{tabular}[c]{@{}l@{}}zip\_transform\_view\\ views::zip\_transform\end{tabular} &
\begin{tabular}[c]{@{}l@{}}创建一个视图,其第 i 个元素是应用给定可调用对象到所有给\\定视图的第 i 个元素的结果。
\end{tabular} \\ \hline
\begin{tabular}[c]{@{}l@{}}adjacent\_view\\ views::adjacent\end{tabular} &
\begin{tabular}[c]{@{}l@{}}对于给定的 n,创建一个视图,其第 i 个元素是一个元组,包\\含对给定视图的第 i 个到第 (i + n - 1) 个元素的引用。
\end{tabular} \\ \hline
\begin{tabular}[c]{@{}l@{}}adjacent\_transform\_view\\ views::adjacent\_transform\end{tabular} &
\begin{tabular}[c]{@{}l@{}}对于给定的 n,创建一个视图,其第 i 个元素是应用给定可调\\用对象到给定视图的第 i 个到第 (i + n - 1) 个元素的结果。
\end{tabular} \\ \hline
\begin{tabular}[c]{@{}l@{}}views::pairwise\\ views::pairwise\_transform\end{tabular} &
\begin{tabular}[c]{@{}l@{}}辅助类型分别表示 views::adjacent<2> 和 \\views::adjacent\_transform<2>。
\end{tabular} \\ \hline
\begin{tabular}[c]{@{}l@{}}join\_with\_view\\ views::join\_with\end{tabular} &
\begin{tabular}[c]{@{}l@{}}给定分隔符,展开给定视图的元素,在视图的元素之间插入分隔\\符的每个元素。分隔符可以是单个元素或元素视图。
\end{tabular} \\ \hline
\begin{tabular}[c]{@{}l@{}}stride\_view\\ views::stride\end{tabular} &
\begin{tabular}[c]{@{}l@{}}对于给定的 n,创建一个视图,该视图在每次前进时覆盖 n 个\\底层序列的元素,而不是一个接一个。
\end{tabular} \\ \hline
\begin{tabular}[c]{@{}l@{}}slide\_view\\ views::slide\end{tabular} &
\begin{tabular}[c]{@{}l@{}}对于给定的 n,创建一个视图,其第 i 个元素是原始视图的第\\ i 个到第 (i + n - 1) 个元素上的视图。类似于 \\views::adjacent,但窗口大小 n 是 \\slide 的运行时参数,而对于 adjacent 是一个模板参数。
\end{tabular} \\ \hline
\begin{tabular}[c]{@{}l@{}}chunk\_view\\ views::chunk\end{tabular} &
\begin{tabular}[c]{@{}l@{}}对于给定的 n,创建一个视图范围,这些视图是不重叠的连续\\原始视图元素块,每个块的大小为 n,按顺序排列。
\end{tabular} \\ \hline
\begin{tabular}[c]{@{}l@{}}chunk\_by\_view\\ views::chunk\_by\end{tabular} &
\begin{tabular}[c]{@{}l@{}}将给定视图按给定谓词返回 false 的每对相邻元素之间分割成\\子范围。
\end{tabular} \\ \hline
\begin{tabular}[c]{@{}l@{}}cartesian\_product\_view\\ views::cartesian\_product\end{tabular} &
\begin{tabular}[c]{@{}l@{}}给定 n 个范围,创建一个视图,该视图由提供的范围 n 元笛卡\\尔积计算得到的元组组成。
\end{tabular} \\ \hline
\end{longtable}

表中的范围适配器在第一列展示了 std::ranges 命名空间中的类名和 std::ranges::views 命名空间中的相应范围适配器对象。标准库提供了一个名为 std::views 的命名空间别名,等于 std::ranges::views。

每个范围适配器可以通过调用其构造函数,并传递所需的参数来构造。第一个参数始终是要操作的范围,后面可以有零个或更多的参数:

\begin{cpp}
std::ranges::operation_view { range, arguments... }
\end{cpp}

通常,不会使用这些范围适配器的构造函数来创建,而是使用 std::ranges::views 命名空间中的范围适配器对象,结合按位或操作符|:

\begin{cpp}
range | std::ranges::views::operation(arguments...)
\end{cpp}

这些范围适配器的一些实际应用。下面的示例首先定义了一个简化的函数模板 printRange(),用于打印一个消息以及给定范围内的所有元素。接下来,main() 函数首先创建一个包含 1 到 10 的整数vector,然后应用多个范围适配器,每次都调用 printRange() 的结果,以便跟踪发生了什么。之后,展示了几个新的 C++23 范围适配器。该示例使用了本章早些时候介绍的 myCopy() 函数。

\begin{cpp}
void printRange(string_view msg, auto&& range) { println("{}{:n}", msg, range); }
int main()
{
    vector values { 1, 2, 3, 4, 5, 6, 7, 8, 9, 10 };
    printRange("Original sequence: ", values);

    // Filter out all odd values, leaving only the even values.
    auto result1 { values
        | views::filter([](const auto& value) { return value % 2 == 0; }) };
    printRange("Only even values: ", result1);

    // Transform all values to their double value.
    auto result2 { result1
        | views::transform([](const auto& value) { return value * 2.0; }) };
    printRange("Values doubled: ", result2);

    // Drop the first 2 elements.
    auto result3 { result2 | views::drop(2) };
    printRange("First two dropped: ", result3);

    // Reverse the view.
    auto result4 { result3 | views::reverse };
    printRange("Sequence reversed: ", result4);

    // C++23: views::zip
    vector v1 { 1, 2 };
    vector v2 { 'a', 'b', 'c' };
    auto result5 { views::zip(v1, v2) };
    printRange("views::zip: ", result5);

    // C++23: views::adjacent
    vector v3 { 1, 2, 3, 4, 5 };
    auto result6 { v3 | views::adjacent<2> };
    printRange("views::adjacent: ", result6);

    // C++23: views::chunk
    auto result7 { v3 | views::chunk(2) };
    printRange("views::chunk: ", result7);

    // C++23: views::stride
    auto result8 { v3 | views::stride(2) };
    printRange("views::stride: ", result8);

    // C++23: views::enumerate + views::split
    string lorem { "Lorem ipsum dolor sit amet" };
    for (auto [index, word] : lorem | views::split(' ') | views::enumerate) {
        print("{}:'{}' ", index, string_view { word });
    }
    println("");

    // C++23: views::as_rvalue
    vector<string> words { "Lorem", "ipsum", "dolor", "sit", "amet" };
    vector<string> movedWords;
    auto rvalueView { words | views::as_rvalue };
    myCopy(begin(rvalueView), end(rvalueView), back_inserter(movedWords));
    printRange("movedWords: ", movedWords);

    // C++23: Cartesian product of vector v with itself.
    vector v { 0, 1, 2 };
    for (auto&&[a, b] : views::cartesian_product(v, v)) {print("({},{}) ", a, b);}
}
\end{cpp}

输出如下所示:

\begin{shell}
Original sequence: 1, 2, 3, 4, 5, 6, 7, 8, 9, 10
Only even values: 2, 4, 6, 8, 10
Values doubled: 4, 8, 12, 16, 20
First two dropped: 12, 16, 20
Sequence reversed: 20, 16, 12
views::zip: (1, 'a'), (2, 'b')
views::adjacent: (1, 2), (2, 3), (3, 4), (4, 5)
views::chunk: (1, 2), (3, 4), (5)
views::stride: 1, 3, 5
0:'Lorem' 1:'ipsum' 2:'dolor' 3:'sit' 4:'amet'
movedWords: Lorem, ipsum, dolor, sit, amet
(0,0) (0,1) (0,2) (1,0) (1,1) (1,2) (2,0) (2,1) (2,2)
\end{shell}

重要的是要重复说明,视图是惰性计算的。这个例子中,result1 视图的构建并没有进行实际的过滤。过滤发生在 printRange() 函数中,遍历 result1 元素的时候。

代码段使用了 std::ranges::views 中的范围适配器对象,也可以使用其构造函数来构造范围适配器。例如,result1 视图可以按照以下方式构建:

\begin{cpp}
auto result1 { ranges::filter_view { values,
        [](const auto& value) { return value % 2 == 0; } } };
\end{cpp}

这个例子创建了几个中间视图,result1、result2、result3 和 result4,以便能够输出它们的元素,以便更容易跟踪每一步发生了什么。如果不需要这些中间视图,可以将它们全部串联在一个管道中:

\begin{cpp}
vector values { 1, 2, 3, 4, 5, 6, 7, 8, 9, 10 };
printRange("Original sequence: ", values);
auto result { values
    | views::filter([](const auto& value) { return value % 2 == 0; })
    | views::transform([](const auto& value) { return value * 2.0; })
    | views::drop(2)
    | views::reverse };
printRange("Final sequence: ", result);
\end{cpp}

输出如下。最后一行显示最终序列与之前的 result4 视图相同。

\begin{shell}
Original sequence: 1, 2, 3, 4, 5, 6, 7, 8, 9, 10
Final sequence: 20, 16, 12
\end{shell}

\mySamllsection{通过视图修改元素}

一些范围是只读的,例如:views::transform 的结果是一个只读视图,其创建了一个包含转换后元素的视图,但没有转换底层范围的实际值。如果一个范围不是只读的,就可以通过视图修改该范围中的元素。

以下示例构建了一个包含十个元素的vector,然后创建了一个覆盖偶数值的视图,丢弃了前两个偶数值,最后反转了元素。基于范围的 for 循环然后将结果视图中的元素乘以 10。最后一行输出原始 values 中的元素,以确认元素是否通过视图进行了修改。

\begin{cpp}
vector values { 1, 2, 3, 4, 5, 6, 7, 8, 9, 10 };
printRange("Original sequence: ", values);

// Filter out all odd values, leaving only the even values.
auto result1 { values
    | views::filter([](const auto& value) { return value % 2 == 0; }) };
printRange("Only even values: ", result1);

// Drop the first 2 elements.
auto result2 { result1 | views::drop(2) };
printRange("First two dropped: ", result2);

// Reverse the view.
auto result3 { result2 | views::reverse };
printRange("Sequence reversed: ", result3);

// Modify the elements using a range-based for loop.
for (auto& value : result3) { value *= 10; }
printRange("After modifying elements through a view, vector contains:\n", values);
\end{cpp}

程序的输出如下所示:

\begin{shell}
Original sequence: 1, 2, 3, 4, 5, 6, 7, 8, 9, 10
Only even values: 2, 4, 6, 8, 10
First two dropped: 6, 8, 10
Sequence reversed: 10, 8, 6
After modifying elements through a view, vector contains:
1, 2, 3, 4, 5, 60, 7, 80, 9, 100
\end{shell}

\mySamllsection{映射元素}

将范围中的元素转换成另一种类型,不需要产生一个具有相同类型的范围,可以将元素映射到另一种类型。以下示例从一个整数范围开始,过滤出所有奇数元素,只保留前三个偶数值,并使用 std::format() 将这些值转换为字符串:

\begin{cpp}
vector values { 1, 2, 3, 4, 5, 6, 7, 8, 9, 10 };
printRange("Original sequence: ", values);

auto result { values
    | views::filter([](const auto& value) { return value % 2 == 0; })
    | views::take(3)
    | views::transform([](const auto& v) { return format("{}", v); }) };
printRange("Result: ", result);
\end{cpp}

输出如下所示:

\begin{shell}
Original sequence: 1, 2, 3, 4, 5, 6, 7, 8, 9, 10
Result: "2", "4", "6"
\end{shell}

\mySubsubsection{17.4.3}{范围工厂}

范围库提供了以下范围工厂来构建按需生成元素的视图:

% Please add the following required packages to your document preamble:
% \usepackage{longtable}
% Note: It may be necessary to compile the document several times to get a multi-page table to line up properly
\begin{longtable}{|l|l|}
\hline
\textbf{范围工厂} & \textbf{描述}                        \\ \hline
\endfirsthead
%
\endhead
%
empty\_view            & 创建一个空视图。
                      \\ \hline
single\_view           & 创建一个包含单个给定元素的视图。
 \\ \hline
iota\_view &
\begin{tabular}[c]{@{}l@{}}创建一个包含从初始值开始的元\\素,其中每个后续元素都是前一个元素加一的无穷大或有限视图。
\end{tabular} \\ \hline
repeat\_view(C++23) &
\begin{tabular}[c]{@{}l@{}}创建一个重复给定值的视图。结果视图可以是无穷大(无限)或有限,\\具体情况由给定数量的值决定。
\end{tabular} \\ \hline
\begin{tabular}[c]{@{}l@{}}basic\_istream\_view\\ istream\_view\end{tabular} &
\begin{tabular}[c]{@{}l@{}}创建一个视图,包含通过调用底层输入流提取操作符 operator>{}> \\获取的元素。
\end{tabular} \\ \hline
\end{longtable}

正如前一节中的范围适配器一样,范围工厂表中的名称是 std::ranges 命名空间中的类名,可以直接使用其构造函数来创建。或者,可以使用 std::ranges::views 命名空间中可用的工厂函数。以下两个语句等效,并且都创建了一个从10开始的无穷大序列:

\begin{cpp}
std::ranges::iota_view { 10 }
std::ranges::views::iota(10)
\end{cpp}

看一个范围工厂的实际应用。以下示例基于之前的示例,但不是构建一个包含10个元素的vector,而是使用 iota 范围工厂创建一个从10开始的惰性加载无穷序列。然后移除所有奇数值,将剩余的元素加倍,并最终只保留前10个元素,这些元素随后通过 printRange() 输出到控制台。

\begin{cpp}
// Create an infinite sequence of the numbers 10, 11, 12, ...
auto values { views::iota(10) };
// Filter out all odd values, leaving only the even values.
auto result1 { values
    | views::filter([](const auto& value) { return value % 2 == 0; }) };
// Transform all values to their double value.
auto result2 { result1
    | views::transform([](const auto& value) { return value * 2.0; }) };
// Take only the first ten elements.
auto result3 { result2 | views::take(10) };
printRange("Result: ", result3);
\end{cpp}

输出如下所示:

\begin{shell}
Result: 20, 24, 28, 32, 36, 40, 44, 48, 52, 56
\end{shell}

values 范围代表一个无穷范围,随后进行过滤和转换。范围可能是无穷的,因为这些操作都是惰性(按需)求值的,只在 printRange() 遍历视图的元素时进行,因此不能调用 printRange() 来输出 values、result1 或 result2 的内容,这些都是无穷范围。

当然,可以消除这些中间视图,直接构建一个大的流水线。以下代码会产生与之前相同的结果:

\begin{cpp}
auto result { views::iota(10)
    | views::filter([](const auto& value) { return value % 2 == 0; })
    | views::transform([](const auto& value) { return value * 2.0; })
    | views::take(10) };
printRange("Result: ", result);
\end{cpp}

另一个范围工厂示例演示了如何使用 repeat\_view:

\begin{cpp}
printRange("Repeating view: ", views::repeat(42, 5));
\end{cpp}

这会输出以下内容:

\begin{shell}
Repeating view: 42, 42, 42, 42, 42
\end{shell}

\mySamllsection{输入流作为视图}

basic\_istream\_view/istream\_view 范围工厂可以用来构造一个从输入流(如标准输入)读取的元素的视图,使用 operator>{}> 读取元素。

例如,以下代码片段会不断从标准输入读取整数。对于每次读取的数字,如果数字小于 5,则将数字加倍并打印到标准输出。当输入 5 或更大的数字,循环就会停止。

\begin{cpp}
println("Type integers, an integer >= 5 stops the program.");
for (auto value : ranges::istream_view<int> { cin }
    | views::take_while([](const auto& v) { return v < 5; })
    | views::transform([](const auto& v) { return v * 2; })) {
    println("> {}", value);
}
println("Terminating...");
\end{cpp}

以下是一个可能的输出序列:

\begin{shell}
Type integers, an integer >= 5 stops the program.
1 2
> 2
> 4
3
> 6
4
> 8
5
Terminating...
\end{shell}

\CXXTwentythreeLogo{-40}{-55}

\mySubsubsection{17.4.4}{将范围转换为容器}

C++23 之前,将范围转换为容器并不容易。C++23 引入了 std::ranges::to() 使得这样的转换变得简单直接。因为视图是一个范围,所以也可以用来将视图的元素转换为容器。由于容器也是一个范围,可以使用 ranges::to() 将一个容器转换为另一个容器,即使元素类型不同。

以下代码演示了 ranges::to() 的几种使用方式。示例还展示了 set 和 string 的特殊构造函数,这些构造函数接受 std::from\_range 标签作为第一个参数,并将给定的范围转换为 set 或 string。现在所有标准库容器都包括这样的构造函数。

\begin{cpp}
// Convert a vector to a set with the same element type.
vector vec { 33, 11, 22 };
auto s1 { ranges::to<set>(vec) };
println("{:n}", s1);

// Convert a vector of integers to a set of doubles, using the pipe operator.
auto s2 { vec | ranges::to<set<double>>() };
println("{:n}", s2);

// Convert a vector of integers to a set of doubles, using from_range constructor.
set<double> s3 { from_range, vec };
println("{:n}", s3);

// Lazily generate the integers from 10 to 14, divide these by 2,
// and store the result in a vector of doubles.
auto vec2 { views::iota(10, 15)
    | views::transform([](const auto& v) { return v / 2.0; })
    | ranges::to<vector<double>>() };
println("{:n}", vec2);

// Use views::split() and views::transform() to create a view
// containing individual words of a string, and then convert
// the resulting view to a vector of strings containing all the words.
string lorem { "Lorem ipsum dolor sit amet" };
auto words { lorem | views::split(' ')
    | views::transform([](const auto& v) { return string { from_range, v }; })
    | ranges::to<vector>() };
println("{:n:?}", words);
\end{cpp}

输出如下所示:

\begin{shell}
11, 22, 33
11, 22, 33
11, 22, 33
5, 5.5, 6, 6.5, 7
"Lorem", "ipsum", "dolor", "sit", "amet"
\end{shell}

C++23 还引入了标准库容器的一组新成员函数,提供了容器和范围之间的互操作性。这些成员函数的形式为 xyz\_range(...),其中 xyz 可以是 insert、append、prepend、assign、replace、push、push\_front 或 push\_back。第 18 章详细讨论了所有标准库容器,但请查阅标准库手册,了解哪些容器支持哪个成员函数。以下是使用 vector 的 append\_range() 和 insert\_range() 成员函数的示例:

\begin{cpp}
vector<int> vec3;
vec3.append_range(views::iota(10, 15));
println("{:n}", vec3);
vec3.insert_range(begin(vec3), views::iota(10, 15) | views::reverse);
println("{:n}", vec3);
\end{cpp}

输出如下所示:

\begin{shell}
10, 11, 12, 13, 14
14, 13, 12, 11, 10, 10, 11, 12, 13, 14
\end{shell}








