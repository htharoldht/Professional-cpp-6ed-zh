

C++20之前,可以使用以下预处理器宏来获取关于源代码中位置的信息:

% Please add the following required packages to your document preamble:
% \usepackage{longtable}
% Note: It may be necessary to compile the document several times to get a multi-page table to line up properly
\begin{longtable}{|l|l|}
\hline
\textbf{宏} & \textbf{描述}                                     \\ \hline
\endfirsthead
%
\endhead
%
\_\_FILE\_\_   & 用当前源代码文件名替换
           \\ \hline
\_\_LINE\_\_   & 用源代码中的当前行号替换
 \\ \hline
\end{longtable}

此外,每个函数都有一个局部定义的静态字符数组,名为 \_\_func\_\_,包含函数的名称。
从C++20开始,\_\_func\_\_ 以及这些C风格的预处理器宏的面向对象替代品可用,形式为std::source\_location类,定义在<source\_location>中。source\_location类的实例具有以下公共访问器:


% Please add the following required packages to your document preamble:
% \usepackage{longtable}
% Note: It may be necessary to compile the document several times to get a multi-page table to line up properly
\begin{longtable}{|l|l|}
\hline
\textbf{访问器} & \textbf{描述}                                  \\ \hline
\endfirsthead
%
\endhead
%
file\_name()      & 包含当前源文件的文件名             \\ \hline
function\_name() & \begin{tabular}[c]{@{}l@{}}包含当前函数名,如果当前位置在函数内部。
\end{tabular} \\ \hline
line()            & 包含源代码中的当前行号。
  \\ \hline
column()          & 包含源代码中的当前列号
 \\ \hline
\end{longtable}

提供了一个静态成员函数 current(),该函数基于成员函数调用的源代码位置创建一个 source\_location 实例。

\mySubsubsection{14.6.1}{日志记录的源位置}

source\_location类在日志记录方面非常有用。以前,日志记录通常涉及编写C风格的宏来自动收集当前文件名、函数名和行号,以便可以包含在日志输出中。现在,有了source\_location,可以编写一个纯C++函数来执行日志记录,并自动收集需要的定位数据。实现这个功能的好方法是定义一个像下面的logMessage()函数。这次,代码前面加了行号,以便更好地解释发生了什么。

\begin{shell}
5.  void logMessage(string_view message,
6.  const source_location& location = source_location::current())
7.  {
8.      println("{}({}): {}: {}", location.file_name(),
9.      location.line(), location.function_name(), message);
10. }
11.
12. void foo()
13. {
14.     logMessage("Starting execution of foo().");
15. }
16.
17. int main()
18. {
19.     foo();
20. }
\end{shell}

logMessage()的第二个参数是一个source\_location,其默认值为source\_location::current()静态成员函数的结果。current()的调用不是在行6,而是在调用logMessage()的地方,即行14,而这正是我们感兴趣的位置。

使用Microsoft Visual C++执行此程序后,输出如下所示:

\begin{shell}
./01_Logging.cpp(14): void __cdecl foo(void): Starting execution of foo().
\end{shell}

第14行确实对应于调用logMessage()的行。函数的确切名称,如这里的void \_\_cdecl foo(void),具体名称取决于编译器。

\mySubsubsection{14.6.2}{自定义异常中自动嵌入源位置}

source\_location在自定义异常中的另一个用途是自动存储异常抛出的位置:

\begin{cpp}
class MyException : public exception
{
    public:
        explicit MyException(string message,
            source_location location = source_location::current())
            : m_message { move(message) }
            , m_location { move(location) }
        { }

        const char* what() const noexcept override { return m_message.c_str(); }
        virtual const source_location& where() const noexcept{ return m_location; }
    private:
        string m_message;
        source_location m_location;
};

void doSomething()
{
    throw MyException { "Throwing MyException." };
}

int main()
{
    try {
        doSomething();
    } catch (const MyException& e) {
        const auto& location { e.where() };
        println(cerr, "Caught: '{}' at line {} in {}.",
            e.what(), location.line(), location.function_name());
    }
}
\end{cpp}

使用Microsoft Visual C++会输出类如下内容:

\begin{shell}
Caught: 'Throwing MyException.' at line 26 in void __cdecl doSomething(void).
\end{shell}


























