
标准库支持使用日历日期。目前,只支持公历(格里高利历),也可以实现<chrono>的其余功能(如科普特历和儒略历),并操作自己的日历。

标准库提供了相当多的类和函数来处理日期(和时区,稍后讨论)。本文讨论了最重要的类和函数。查阅标准库手册(见附录B)以获得完整内容。

\mySubsubsection{22.5.1}{创建日期}

以下日历类可用于创建日期,都在std::chrono中定义:

% Please add the following required packages to your document preamble:
% \usepackage{longtable}
% Note: It may be necessary to compile the document several times to get a multi-page table to line up properly
\begin{longtable}{|l|l|}
\hline
\textbf{类型} &
\textbf{描述} \\ \hline
\endfirsthead
%
\endhead
%
year &
\begin{tabular}[c]{@{}l@{}}表示在范围 {[}-32767, 32767{]} 内的年份。年份有一个名为\\is\_leap()的成员函数,如果给定年份是闰年,则返回 true,\\否则返回 false。min() 和 max() 静态成员函数分别返回\\最小和最大年份。\end{tabular} \\ \hline
month &
\begin{tabular}[c]{@{}l@{}}表示在范围 {[}1, 12{]} 内的月份。此外,还为12个月提供了\\12个命名常量,例如:std::chrono::January。\end{tabular} \\ \hline
day &
表示在范围 {[}1, 31{]} 内的一天。 \\ \hline
weekday &
\begin{tabular}[c]{@{}l@{}}表示在范围 {[}0, 6{]} 内的一周中的某一天,其中 0 表示星期\\日。此外,还为七个工作日提供了七个命名常量,例如:\\std::chrono::Sunday。\end{tabular} \\ \hline
weekday\_indexed &
\begin{tabular}[c]{@{}l@{}}表示一个月的第一、第二、第三、第四或第五个工作日。可以\\很容易地从工作日构造,例如:Monday{[}2{]} 是一个月的第二\\个星期一。 \end{tabular} \\ \hline
weekday\_last &
表示某个月最后一个工作日。 \\ \hline
month\_day &
表示一个特定的月份和日期。 \\ \hline
month\_day\_last &
表示一个特定月份的最后一天。 \\ \hline
month\_weekday &
表示一个特定月份的第 n 个工作日。 \\ \hline
month\_weekday\_last &
表示一个特定月份的最后一个工作日。 \\ \hline
year\_month &
表示一个特定的年份和月份。 \\ \hline
year\_month\_day &
表示一个特定的年份、月份和日期。 \\ \hline
year\_month\_day\_last &
表示一个特定年份和月份的最后一天。 \\ \hline
year\_month\_weekday &
表示一个特定年份和月份的第 n 个工作日。 \\ \hline
year\_month\_weekday\_last &
表示一个特定年份和月份的最后一个工作日。 \\ \hline
\end{longtable}

这些类都有一个名为 ok() 的成员函数,如果给定对象在有效范围内,则返回 true,否则返回 false。在 std::literals::chrono\_literals 中提供了两个额外的标准字面量:y 用于创建年份,d 用于创建日期。可以使用 operator/ 按三种顺序(Y/M/D、M/D/Y、D/M/Y)指定年、月、日来构造完整的日期。以下是一些创建日期的示例:

\begin{cpp}
year y1 { 2020 };
auto y2 { 2020y };

month m1 { 6 };
auto m2 { June };

day d1 { 22 };
auto d2 { 22d };

// Create a date for 2020-06-22.
year_month_day fulldate1 { 2020y, June, 22d };
auto fulldate2 { 2020y / June / 22d };
auto fulldate3 { 22d / June / 2020y };

// Create a date for the 3rd Monday of June 2020.
year_month_day fulldate4 { Monday[3] / June / 2020 };

// Create a month_day for June 22 of an unspecified year.
auto june22 { June / 22d };

// Create a year_month_day for June 22, 2020.
auto june22_2020 { 2020y / june22 };

// Create a month_day_last for the last day of a June of an unspecified year.
auto lastDayOfAJune { June / last };

// Create a year_month_day_last for the last day of June for the year 2020.
auto lastDayOfJune2020 { 2020y / lastDayOfAJune };

// Create a year_month_weekday_last for the last Monday of June 2020.
auto lastMondayOfJune2020 { 2020y / June / Monday[last] };
\end{cpp}

sys\_time 是一个类型别名,用于表示具有特定持续时间的 system\_clock 的 time\_point:

\begin{cpp}
template <typename Duration>
using sys_time = std::chrono::time_point<std::chrono::system_clock, Duration>;
\end{cpp}

基于 sys\_time 类型别名,定义了另外两个类型别名,分别表示具有秒精度的 sys\_time 和具有天精度的 sys\_time:

\begin{cpp}
using sys_seconds = sys_time<std::chrono::seconds>;
using sys_days = sys_time<std::chrono::days>;
\end{cpp}

sys\_days 表示自 system\_clock 纪元以来的天数,它是一个基于序列的类型,只包含一个数字(自纪元以来的天数)。另一方面,year\_month\_day是一个基于字段的类型,在单独的字段中存储年、月和日。当对日期进行大量算术运算时,基于序列的类型将比基于字段的类型性能更好。

类似的类型别名也存在,用于处理本地时间:local\_time、local\_seconds 和 local\_days,这些将在后面关于时区的部分进行演示。

可以使用以下方式创建一个表示今天的 sys\_days。floor() 用于将 time\_point 截断为天的精度:

\begin{cpp}
auto today { floor<days>(system_clock::now()) };
\end{cpp}

sys\_days 可以用于将 year\_month\_day 转换为 time\_point:

\begin{cpp}
system_clock::time_point t1 { sys_days { 2020y / June / 22d } };
\end{cpp}

将 time\_point 转换为 year\_month\_day,可以使用 year\_month\_day 构造函数来完成:

\begin{cpp}
year_month_day yearmonthday { floor<days>(t1) };
year_month_day today2 { floor<days>(system_clock::now()) };
\end{cpp}

也可以构建包括时间的完整日期:

\begin{cpp}
// Full date with time: 2020-06-22 09:35:10 UTC.
auto t2 { sys_days { 2020y / June / 22d } + 9h + 35min + 10s };
\end{cpp}

\mySubsubsection{22.5.2}{打印日期}

日期可以使用熟悉的操作符写入流:

\begin{cpp}
cout << yearmonthday << endl;
\end{cpp}

也支持打印和格式化日期,L 格式指定符根据当前全局区域设置格式化输出。

\begin{cpp}
println("{:L}", yearmonthday);
\end{cpp}

输出有时可能出人意料。lastMondayOfJune2020 之前定义为:

\begin{cpp}
// Create a year_month_weekday_last for the last Monday of June 2020.
auto lastMondayOfJune2020 { 2020y / June / Monday[last] };
\end{cpp}

当打印时,输出是 “2020/Jun/Mon[last]”:

\begin{cpp}
println("{:L}", lastMondayOfJune2020); // 2020/Jun/Mon[last]
\end{cpp}

如果想输出确切的日期,“2020-06-29”,需要将 year\_month\_weekday\_last 转换为 year\_month\_day,然后输出结果:

\begin{cpp}
year_month_day lastMondayOfJune2020YMD { sys_days { lastMondayOfJune2020 } };
println("{:L}", lastMondayOfJune2020YMD); // 2020-06-29
\end{cpp}

如果日期无效,打印将插入错误。对于无效的 year\_month\_day,将输出字符串 “is not a valid date”。

使用 L 格式指定符,根据当前全局区域设置,正确地本地化星期和月份的名称。以下代码首先将全局区域设置为荷兰语,nl-NL,然后使用 L 格式指定符以荷兰语打印星期一。\%A 格式指定符导致打印全名而不是缩写名。查阅标准库手册,以获取支持的所有日期格式指定符的完整列表。

\begin{cpp}
locale::global(locale { "nl-NL" });
println("Monday in Dutch is {:L%A}", Monday);
\end{cpp}

输出为:

\begin{shell}
Monday in Dutch is maandag
\end{shell}

\mySubsubsection{22.5.3}{日期的算术运算}

可以对日期进行算术运算:

\begin{cpp}
// Full date with time: 2020-06-22 09:35:10 UTC.
auto t2 { sys_days { 2020y / June / 22d } + 9h + 35min + 10s };
auto t3 { t2 + days { 5 } }; // Add 5 days to t2.
auto t4 { t3 + years { 1 } }; // Add 1 year to t3.
\end{cpp}

结果可能并不总是如预期的那样:

\begin{cpp}
auto t5 { sys_days { 2020y / June / 22d } + 9h + 35min + 10s };
auto t6 { t5 + years { 1 } }; // Add 1 year to t5
println("t5 = {:L}", t5);
println("t6 = {:L}", t6);
\end{cpp}

结果如下:

\begin{shell}
t5 = 2020-06-22 09:35:10
t6 = 2021-06-22 15:24:22
\end{shell}

查看结果时,可以看到年份更新了,但时间也发生了变化。因为正在使用一个序列类型:sys\_days 是一个 time\_point,它是一个序列类型。向这样的序列类型添加 1 年并不会添加 86,400 * 365 = 31,536,000 秒。标准规定,添加 1 年必须添加 1 个平均年,以考虑闰年,因此必须添加 86,400 * ((365 * 400) + 97) / 400 = 31,556,952 秒。

如果需要精确地添加 1 年,最好使用一个基于字段的类型:

\begin{cpp}
// Split t5 into days and remaining seconds.
sys_days t5_days { time_point_cast<days>(t5) };
seconds t5_seconds { t5 - t5_days };
// Convert the t5_days serial type to field-based type.
year_month_day t5_ymd { t5_days };
// Add 1 year.
year_month_day t7_ymd { t5_ymd + years { 1 } };
// Convert back to a serial type.
auto t7 { sys_days { t7_ymd } + t5_seconds };
println("t7 = {:L}", t7);
\end{cpp}

结果为:

\begin{shell}
t7 = 2021-06-22 09:35:10
\end{shell}




