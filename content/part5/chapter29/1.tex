
深入了解细节之前,定义一下本书中使用的性能和效率这两个术语。程序的性能可能涉及几个方面,比如速度、内存使用、磁盘访问和网络使用。本章专注于速度性能,一个高效的程序在给定条件下尽可能快地完成任务。如果一个应用程序禁止快速执行,则程序可以有效率,而不是快速。

\begin{myNotic}{NOTE}
高效或高性能的程序会尽可能快地完成特定的任务。
\end{myNotic}

注意,本章的标题“编写高效的C++”,意味着编写运行效率高的程序,而不是高效地编写程序。也就是说,通过阅读本章节省的时间将属于用户,而不是开发者!

\mySubsubsection{29.1.1}{两种高效的方法}

语言级别的效率涉及尽可能高效地使用语言;例如,通过引用而不是通过值传递对象,但这只能让您走得更远。更重要的是设计级别的效率,这包括选择高效的算法、避免不必要的步骤和计算,以及选择适当的设计优化。优化现有代码涉及用更好、更高效的一个替换不好的算法或数据结构。

\mySubsubsection{29.1.2}{两种类型的程序}

效率对所有应用程序领域都很重要。还有一小部分程序,如系统级软件、嵌入式系统、密集计算应用程序和实时游戏,需要极高的效率。可以将其视为建造普通家用汽车和赛车。每辆汽车都必须有合理的效率,但运动汽车需要极高的性能。运动汽车肯定会超过70英里每小时运行,但家用汽车的速度没必要浪费时间去进行优化。

\mySubsubsection{29.1.3}{C++是一种低效的语言吗?}

C程序员经常抵制使用C++进行高性能应用程序开发。他们认为,由于C++包含了高级概念,如异常和虚成员函数,因此该语言本质上是比C或类似的过程式语言效率低。然而,这种论点存在问题。

讨论语言的效率时,不能忽略编译器的影响。回想一下,编写的C或C++代码并不是计算机执行的代码。编译器首先将代码转换为机器语言,这个过程中应用优化。所以不能简单地对C和C++程序运行基准测试并比较结果,这实际上是在比较语言的编译器优化,而不是语言本身。C++编译器可以优化掉语言中的许多高级构造,以生成与类似C程序生成的机器代码相似或甚至更好的机器代码。如今,投入到C++编译器的研究和开发远远超过了投入到C编译器的,C++代码可能会得到更好的优化,并且可能会比C代码运行得更快。

然而,批评者仍然坚持认为C++的一些特性无法优化掉。例如,正如第10章所解释的,虚成员函数需要存在一个vtable,也称为虚表,以及在运行时会增加,这可能会使它们比常规的非虚函数调用慢。但当认真思考这个问题时,这个论点并不可信。虚成员函数调用提供的不仅仅是函数调用:还会在运行时选择调用哪个函数,非虚函数,需要一个条件语句来决定调用哪个函数。如果不需要其他语义,可以使用非虚函数。C++语言的一个通用设计规则是:“不需要为不使用的特性付费。”如果不使用虚成员函数,就不需要为可以使用而付出性能代价。因此,C++中的非虚函数在性能方面,与C中的函数相同。

更重要的是,C++的高级构造使能够编写更清晰的程序,这些程序在设计级别上更高效、更易读、更容易维护,并且避免了积累不必要的和无效的代码。

我相信,选择C++而不是C这样的过程式语言,将在开发、性能和代码维护方面得到更好的体验。

还有其他更高级的面向对象语言,如C\#和Java,它们都运行在虚拟机上。C++代码直接由CPU执行;没有所谓的虚拟机来运行代码。C++更接近硬件,在大多数情况下,它比C\#和Java运行得更快。









