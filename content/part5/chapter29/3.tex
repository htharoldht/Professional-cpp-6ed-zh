
程序的设计选择对其性能的影响远大于语言细节,如按引用传递。为应用程序中的基本任务选择了一个运行时间为O($n^2$)的算法,而不是一个运行时间为O(n)的更简单的算法,可能会执行实际所需操作的平方。使用O($n^2$)算法的任务,如果执行了100万次操作,那么使用O(n)算法的任务只需执行1000次。即使这个操作在语言层面上优化到极致,但因为执行了100万次操作,性能也不会比只需执行1000次的算法更好,从而会使程序非常低效。有关算法设计选择和大O表示法的详细讨论,请参阅本书第二部分,特别是第4章。

除了算法选择之外,设计级别的效率还包括特定的技巧和窍门。尽量使用现有的数据结构和算法,如C++标准库、Boost库(boost.org)或其他库由专家编写,而不是自己编写。这些库已经广泛使用,因此可以期望大多数错误已经发现并修复。还应该考虑在设计中整合多线程,以充分利用机器上的所有处理能力。有关多线程编程的更多详细信息,请参阅第27章。本节余下部分将介绍两种更多用于优化程序的设计技术:缓存和使用对象池。

\mySubsubsection{29.3.1}{必要时使用缓存}

缓存是将项目存储以备将来使用,以避免检索或重新计算。各位读者可能熟悉这个原则,因为它在计算机硬件中广泛使用。现代计算机处理器内置了内存缓存,存储最近和频繁访问的内存值,这些值比主内存更容易访问。大多数访问过的内存位置在短时间内会访问多次,因此在硬件层面上缓存可以显著加快计算速度。

软件中的缓存遵循相同的原理。如果一个任务或计算特别慢,应该没有执行超过必要的次数。第一次执行任务时,将其结果存储在内存中,以便于未来使用。以下是通常较慢的任务列表:

\begin{itemize}
\item
磁盘访问:避免在程序中多次打开和读取相同的文件。如果内存可用,且需要频繁访问,请将文件内容保存在RAM中。

\item
网络通信:需要通过网络进行通信时,程序都可能受到网络的不确定性影响。将网络访问视为文件访问,缓存尽可能多的静态信息。

\item
数学计算:如果在多个地方需要一个复杂计算的结果,只需执行一次计算并共享结果。如果不太复杂,可直接计算,比从缓存中检索更快。可以使用分析器来确定。

\item
对象分配:如果需要在程序中创建和使用大量短期对象,考虑使用本章后面将介绍的对象池。

\item
线程创建:创建线程很慢。可以“缓存”线程在线程池中,类似于在对象池中缓存对象。
\end{itemize}

缓存的一个常见问题是,存储的数据是只包含底层信息的副本。原始数据可能在缓存的生命周期内发生变化,可能想要缓存配置文件中的值,这样就不需要重复读取。但用户需要在程序运行时更改配置文件,这将使缓存信息过时。这时,需要一个缓存失效机制:当底层数据发生变化时,必须停止使用缓存信息或重新填充。

缓存失效是,请求管理底层数据的实体在每次更改时通知程序,这可以通过程序注册的回调来实现。或者,程序可以进行轮询,这些事件会触发自动重新填充缓存。无论采用哪种特定的缓存失效技术,在依赖程序中的缓存之前,请确保已经考虑过这些问题。

在数据结构中添加缓存时,请确保设计将缓存的细节隐藏在公共接口之外。客户端代码不应该知道底层实现使用了什么类型的缓存。这也允许在不影响公共接口的情况下,更改缓存机制。

\begin{myNotic}{NOTE}
维护缓存需要代码、内存和处理时间。更重要的是,缓存可能是微妙的错误来源。只在没有性能瓶颈的情况下的特定区域,添加缓存。首先编写清晰正确的代码,然后对其进行性能分析,然后才优化其中的一部分。
\end{myNotic}

\mySubsubsection{29.3.2}{使用对象池}

本节讨论一种对象池,一次性分配一大块内存,在这个内存中,池创建更小的对象。这些对象可以分发给客户端,并在客户端使用完毕后重新使用,而无需对内存管理器进行调用,以分配或释放单个对象的内存。

以下对象池实现的优势将在基准测试中展示,特别是对于具有大数据成员的对象。是否为特定用例使用对象池,只能通过分析代码来决定。

\mySamllsection{对象池的实现}

本节提供了一个对象池类模板的实现,可以在程序中使用。该实现维护了一个包含T类型对象块的vector。此外,还可以跟踪对象,这些对象存储在一个包含所有自由对象指针的vector中。对象池通过acquireObject()成员函数分配对象。如果调用acquireObject(),对象将不存在,对象池将分配另一个对象块。acquireObject()会返回一个shared\_ptr。

此实现使用了不带同步的vector标准库容器,但这个版本不是线程安全的。有关如何使实现线程安全的讨论,请参阅第27章。

以下是带有注释解释细节的类定义。该类模板参数化在要存储在池中的类型,以及用于分配和释放内存块的分配器类型。

\begin{cpp}
// Provides an object pool that can be used with any class.
//
// acquireObject() returns an object from the list of free objects. If
// there are no more free objects, acquireObject() creates a new chunk
// of objects.
// The pool only grows: objects are never removed from the pool, until
// the pool is destroyed.
// acquireObject() returns an std::shared_ptr with a custom deleter that
// automatically puts the object back into the object pool when the
// shared_ptr is destroyed and its reference count reaches 0.
export
template <typename T, typename Allocator = std::allocator<T>>
class ObjectPool final
{
    public:
        ObjectPool() = default;
        explicit ObjectPool(const Allocator& allocator);
        ~ObjectPool();

        // Prevent move construction and move assignment.
        ObjectPool(ObjectPool&&) = delete;
        ObjectPool& operator=(ObjectPool&&) = delete;

        // Prevent copy construction and copy assignment.
        ObjectPool(const ObjectPool&) = delete;
        ObjectPool& operator=(const ObjectPool&) = delete;

        // Reserves and returns an object from the pool. Arguments can be
        // provided which are perfectly forwarded to a constructor of T.
        template <typename... Args>
        std::shared_ptr<T> acquireObject(Args&&... args);
    private:
        // Creates a new block of uninitialized memory, big enough to hold
        // m_newChunkSize instances of T.
        void addChunk();
        // Contains chunks of memory in which instances of T will be created.
        // For each chunk, the pointer to its first object is stored.
        std::vector<T*> m_pool;
        // Contains pointers to all free instances of T that
        // are available in the pool.
        std::vector<T*> m_freeObjects;
        // The number of T instances that should fit in the first allocated chunk.
        static constexpr std::size_t ms_initialChunkSize { 5 };
        // The number of T instances that should fit in a newly allocated chunk.
        // This value is doubled after each newly created chunk.
        std::size_t m_newChunkSize { ms_initialChunkSize };
        // The allocator to use for allocating and deallocating chunks.
        Allocator m_allocator;
};
\end{cpp}

使用此对象池时,必须确保对象池本身的生命周期超过由池生成的所有对象。

构造函数很简单,只是将给定的分配器存储在数据成员中:

\begin{cpp}
template <typename T, typename Allocator>
ObjectPool<T, Allocator>::ObjectPool(const Allocator& allocator)
    : m_allocator { allocator }
{}
\end{cpp}

分配新块的addChunk()成员函数实现如下。addChunk()的第一部分实际分配了一个新块,“块”只是一个使用分配器分配的未初始化内存块,足够容纳m\_newChunkSize个T实例。通过添加对象块,实际上还没有构造对象,还没有调用对象构造函数。这是在acquireObject()中分创建实例时完成的,addChunk()的第二部分创建指向T新实例的指针,使用在<numeric>中定义的iota()算法,iota()用值填充由其前两个参数给出的范围。这些值从第三个参数的值开始,并依次为每个后续值递增。因为正在处理T*指针,所以将T*指针递增一个会将指针跳转到内存块中的下一个T。最后,m\_newChunkSize的值加倍,因此下一次添加的块将是当前添加块的 double 大小。这样做是为了性能考虑,遵循了std::vector的原则。以下是实现:

\begin{cpp}
template <typename T, typename Allocator>
void ObjectPool<T, Allocator>::addChunk()
{
    std::println("Allocating new chunk...");
    // Allocate a new chunk of uninitialized memory big enough to hold
    // m_newChunkSize instances of T, and add the chunk to the pool.
    // Care is taken that everything is cleaned up in the event of an exception.
    m_pool.push_back(nullptr);
    try {
        m_pool.back() = m_allocator.allocate(m_newChunkSize);
    } catch (...) {
        m_pool.pop_back();
        throw;
    }

    // Create pointers to each individual object in the new chunk
    // and store them in the list of free objects.
    auto oldFreeObjectsSize { m_freeObjects.size() };
    m_freeObjects.resize(oldFreeObjectsSize + m_newChunkSize);
    std::iota(begin(m_freeObjects) + oldFreeObjectsSize, end(m_freeObjects),
        m_pool.back());
    // Double the chunk size for next time.
    m_newChunkSize *= 2;
}
\end{cpp}

acquireObject(),可变参数的成员函数模板,从池中返回空闲对象,如果没有更多的空闲对象可用,则分配新块。如前所述,添加新块只是分配一个未初始化的内存块。acquireObject()负责在正确的内存位置正确构造T的新实例,使用new操作符完成。传递给acquireObject()的参数,都会完美转发给类型T的构造函数。

acquireObject()使用放置new操作符,在显式指定的内存位置构造类型T的新实例。如果类型T包含const或引用成员,通过原始指针访问新构造的对象将触发未定义行为。为了将此转换为定义良好的行为,使用<std::launder()>来清洗内存。[C++23引入了一个与之稍有关联的函数<std::start\_lifetime\_as()>。与launder()的区别在于,launder()不会创建新对象,只是清洗已经构造的对象的指针。另一方面,start\_lifetime\_as()实际上会创建一个新对象,但不会运行构造函数代码。如果有一个内存块,也知道它代表一个对象,或许是通过网络接收的,并且想将其转换为一个对象,这个函数就很有用了,例如:start\_lifetime\_as<MyObjectType>(networkBuffer)。]

最后,清洗过的T*指针包装在一个带有自定义删除器的shared\_ptr中。这个删除器不会释放内存,使用std::destroy\_at()手动调用析构函数,然后将指针放回可用对象列表中。

\begin{cpp}
template <typename T, typename Allocator>
template <typename... Args>
std::shared_ptr<T> ObjectPool<T, Allocator>::acquireObject(Args&&... args)
{
    // If there are no free objects, allocate a new chunk.
    if (m_freeObjects.empty()) { addChunk(); }

    // Get a free object.
    T* object { m_freeObjects.back() };

    // Initialize, i.e. construct, an instance of T in an
    // uninitialized block of memory using placement new, and
    // perfectly forward any provided arguments to the constructor.
    ::new(object) T { std::forward<Args>(args)... };

    // Launder the object pointer.
    T* constructedObject { std::launder(object) };

    // Remove the object from the list of free objects.
    m_freeObjects.pop_back();

    // Wrap the constructed object and return it.
    return std::shared_ptr<T> { constructedObject, [this](T* object) {
        // Destroy object.
        std::destroy_at(object);
        // Put the object back in the list of free objects.
        m_freeObjects.push_back(object);
    } };
}
\end{cpp}

最后,对象池的析构函数必须使用给定的分配器释放已分配的内存:

\begin{cpp}
template <typename T, typename Allocator>
ObjectPool<T, Allocator>::~ObjectPool()
{
    // Note: this implementation assumes that all objects handed out by this
    //       pool have been returned to the pool before the pool is destroyed.
    //       The following statement asserts if that is not the case.
    assert(m_freeObjects.size() ==
        ms_initialChunkSize * (std::pow(2, m_pool.size()) - 1));

    // Deallocate all allocated memory.
    std::size_t chunkSize { ms_initialChunkSize };
    for (auto* chunk : m_pool) {
        m_allocator.deallocate(chunk, chunkSize);
        chunkSize *= 2;
    }
    m_pool.clear();
}
\end{cpp}

assert()是一个在<cassert>中定义的宏,接受一个布尔表达式,如果表达式的值为假,则打印错误消息并终止程序。有关详细信息,请参阅第31章。assert()语句中使用的公式基于每个已分配的块大小,是前一个块的两倍。

\mySamllsection{使用对象池}

假设一个应用程序,使用大量具有大数据成员的短期对象,分配这些对象代价高昂。假设ExpensiveObject类的定义,如下所示:

\begin{cpp}
class ExpensiveObject
{
    public:
        ExpensiveObject() { /* ... */ }
        virtual ~ExpensiveObject() = default;
        // Member functions to populate the object with specific information.
        // Member functions to retrieve the object data.
        // (not shown)
    private:
        // An expensive data member.
        array<double, 4 * 1024 * 1024> m_data;
        // Other data members (not shown)
};
\end{cpp}

而不是在整个程序的生命周期中分配和释放大量这样的对象,可以使用前面章节中开发的对象池。可以使用chrono库(参见第22章)对池进行基准测试:

\begin{cpp}
using MyPool = ObjectPool<ExpensiveObject>;

shared_ptr<ExpensiveObject> getExpensiveObject(MyPool& pool)
{
    // Obtain an ExpensiveObject object from the pool.
    auto object { pool.acquireObject() };
    // Populate the object. (not shown)
    return object;
}

void processExpensiveObject(ExpensiveObject& object) { /* ... */ }

int main()
{
    const size_t NumberOfIterations { 500'000 };
    println("Starting loop using pool...");
    MyPool requestPool;
    auto start1 { chrono::steady_clock::now() };
    for (size_t i { 0 }; i < NumberOfIterations; ++i) {
        auto object { getExpensiveObject(requestPool) };
        processExpensiveObject(*object.get());
    }
    auto end1 { chrono::steady_clock::now() };
    auto diff1 { end1 - start1 };
    println("{}", chrono::duration<double, milli>(diff1));


    println("Starting loop using new/delete...");
    auto start2 { chrono::steady_clock::now() };
    for (size_t i { 0 }; i < NumberOfIterations; ++i) {
        auto object { std::make_unique<ExpensiveObject>() };
        processExpensiveObject(*object);
    }
    auto end2 { chrono::steady_clock::now() };
    auto diff2 { end2 - start2 };
    println("{}", chrono::duration<double, milli>(diff2));
}
\end{cpp}

main()函数包含了对对象池性能的小型基准测试,在循环中创建500,000个对象,并计算完成所需的时间。循环进行了两次,一次使用对象池,一次使用标准new/delete操作符。在测试机器上,代码的发布(Release)构建结果如下:

\begin{shell}
Starting loop using pool...
Allocating new chunk...
54.526ms
Starting loop using new/delete...
9463.2393ms
\end{shell}

使用对象池大约快了170倍。需要注意的是,这个对象池是为与具有大数据成员的对象一起工作而定制的。这个例子中使用的ExpensiveObject类就是这样,它有一个32MB的数组作为其数据成员。

