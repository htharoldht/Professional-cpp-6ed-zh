有时,一个类提供的抽象不适合当前设计,且无法更改,可以构建一个适配器类。适配器提供了代码使用的抽象,并作为所需抽象与实际底层代码之间的链接。主要有两种使用情况:

\begin{itemize}
\item
通过重用一些现有实现来实现特定的接口,适配器通常在幕后创建实现的一个实例。

\item
通过新接口使用现有功能,适配器的构造函数通常在其构造函数中接收底层对象的一个实例。
\end{itemize}

第18章讨论了标准库如何使用适配器模式,来实现像 stack 和 queue 这样的容器,这些容器是基于其他容器,如 deque 和 list 实现的。

\mySubsubsection{33.5.1}{示例:适配一个 Logger 类}

对于这个适配器模式示例,假设一个 Logger 类:

\begin{cpp}
// Definition of a logger interface.
export class ILogger
{
    public:
        virtual ~ILogger() = default; // Always a virtual destructor!
        enum class LogLevel { Debug, Info, Error };
        // Logs a single message at the given log level.
        virtual void log(LogLevel level, const std::string& message) = 0;
};

// Concrete implementation of ILogger.
export class Logger : public ILogger
{
    public:
        Logger();
        void log(LogLevel level, const std::string& message) override;
    private:
        // Converts a log level to a human readable string.
        std::string_view getLogLevelString(LogLevel level) const;
};
\end{cpp}

Logger 类有一个构造函数,输出一行文本到标准控制台,还有一个名为 log() 的成员函数,将给定的消息写入控制台,前面带有当前系统时间和日志级别。下面是实现:

\begin{cpp}
Logger::Logger() { println("Logger constructor"); }

void Logger::log(LogLevel level, const string& message)
{
    println("{}: [{}] {}", chrono::system_clock::now(),
    getLogLevelString(level), message);
}

string_view Logger::getLogLevelString(LogLevel level) const
{ /* See the strategy-based logger earlier in this chapter. */ }
\end{cpp}

可能想围绕这个基本 Logger 类编写适配器类的一个原因是更改其接口,也许对日志级别不感兴趣,希望只使用一个参数(即实际消息),来调用 log() 成员函数。可能还希望将接口更改为接受 std::string\_view ,而不是字符串作为 log() 成员函数的参数。

\mySubsubsection{33.5.2}{适配器的实现}

实现适配器模式的第一步,是为底层功能定义新的接口。这个新接口称为 IAdaptedLogger:

\begin{cpp}
export class IAdaptedLogger
{
    public:
        virtual ~IAdaptedLogger() = default; // Always virtual destructor!
        // Logs a single message with Info as log level.
        virtual void log(std::string_view message) = 0;
};
\end{cpp}

这个类是一个抽象类,声明了希望新日志记录器具有的所需接口。该接口只定义了一个纯虚成员函数,即接受单个类型为 string\_view 的参数的 log() 成员函数。

接下来,编写具体的新日志记录器类 AdaptedLogger,实现 IAdaptedLogger,以便具有设计的接口。该实现在内部封装了一个 Logger 实例,即组合使用。

\begin{cpp}
export class AdaptedLogger : public IAdaptedLogger
{
    public:
        AdaptedLogger();
        void log(std::string_view message) override;
    private:
        Logger m_logger;
};
\end{cpp}

新类的构造函数输出一行到标准输出,以跟踪哪些构造函数被调用,代码通过将调用转发到封装的 Logger 实例的 log() 成员函数来实现 IAdaptedLogger 的 log() 成员函数。这个调用中,给定的 string\_view 转换为字符串,日志级别硬编码为 Info。

\begin{cpp}
AdaptedLogger::AdaptedLogger() { println("AdaptedLogger constructor"); }

void AdaptedLogger::log(string_view message)
{
    m_logger.log(Logger::LogLevel::Info, string { message });
}
\end{cpp}

\mySubsubsection{33.5.3}{使用适配器}

适配器存在是为了为底层功能提供更合适的接口,所以用法应该是简单和特定的。鉴于前面的实现,以下代码使用新的简化接口进行日志记录:

\begin{cpp}
AdaptedLogger logger;
logger.log("Testing the logger.");
\end{cpp}

产生以下输出:

\begin{shell}
Logger constructor
AdaptedLogger constructor
2023-08-12 14:06:53.3694244: [INFO] Testing the logger.
\end{shell}











