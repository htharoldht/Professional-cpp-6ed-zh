
责任链模式用于希望多个对象有机会执行特定操作时,责任链模式在事件处理中尤其常见。许多现代应用程序,特别是具有图形用户界面的应用程序,都是设计为一系列事件和响应。当用户点击文件菜单并选择打开时,会触发“打开”事件。当用户在绘图程序的可绘制区域上移动鼠标时,会不断产生鼠标移动事件。如果用户按下鼠标按钮,会生成与该按钮按下相关的“鼠标按下”事件。程序可以开始关注鼠标移动事件,允许用户“绘制”某些对象,并继续这样做,直到发生“鼠标抬起”事件。每个操作系统都有其命名和使用这些事件的方式,但总体思想是相同的:当事件发生时,会以某种方式传播到不同的对象,以采取适当的行动。

可能发现责任链模式与装饰器模式相似,但它们之间有一个区别。应该将责任链模式视为if-else语句链的架构等价物;也就是说,找到第一个匹配项,而装饰器模式则扩展功能。

C++没有内置的图形编程设施,也没有事件、事件传输或事件处理的概念。责任链是一种合理的做法,可以让不同的对象有机会处理某些事件。

\mySubsubsection{33.10.1}{示例:事件处理}

考虑一个绘图程序:一个具有窗口的应用程序,其中可以在其中绘制形状。用户可以在应用程序的窗口中按下鼠标按钮。如果发生这种情况,应用程序应该判断用户是否点击了一个形状。如果是这样,形状会要求处理“鼠标按钮按下”事件。如果决定不需要处理该事件,会将事件传递给窗口,窗口将有机会处理事件。如果窗口也不感兴趣处理事件,会将事件转发给应用程序本身,这是处理事件的链中的最后一个对象。这是一个责任链,因为每个处理者都可能处理事件或将其传递给链中的下一个处理者。

\mySubsubsection{33.10.2}{责任链的实现}

假设所有可能的事件都定义在一个枚举中,如下所示:

\begin{cpp}
enum class Event { LeftMouseButtonDown, LeftMouseButtonUp,
    RightMouseButtonDown, RightMouseButtonUp };
\end{cpp}

定义 Handler 基类:

\begin{cpp}
class Handler
{
    public:
        virtual ~Handler() = default;
        // Omitted defaulted default ctor, copy/move ctor, copy/move assignment op.
        explicit Handler(Handler* nextHandler) : m_nextHandler { nextHandler } { }
        virtual void handleMessage(Event message) = 0;
    protected:
        void nextHandler(Event message)
        {
            if (m_nextHandler) { m_nextHandler->handleMessage(message); }
        }
    private:
        Handler* m_nextHandler { nullptr };
};
\end{cpp}

Application、Window 和 Shape 类都是从 Handler 类派生的具体处理者。Application 只处理 RightMouseButtonDown 消息,Window 只处理 LeftMouseButtonUp 消息,而 Shape 只处理 LeftMouseButtonDown 消息。如果处理者收到不认识的消息,会调用链中的下一个处理者。

\begin{cpp}
class Application : public Handler
{
    public:
        explicit Application(Handler* nextHandler) : Handler { nextHandler } { }
        void handleMessage(Event message) override
        {
            println("Application::handleMessage()");
            if (message == Event::RightMouseButtonDown) {
                println(" Handling message RightMouseButtonDown");
            } else { nextHandler(message); }
        }
};

class Window : public Handler
{
    public:
        explicit Window(Handler* nextHandler) : Handler { nextHandler } { }
        void handleMessage(Event message) override
        {
            println("Window::handleMessage()");
            if (message == Event::LeftMouseButtonUp) {
                println(" Handling message LeftMouseButtonUp");
            } else { nextHandler(message); }
        }
};

class Shape : public Handler
{
    public:
        explicit Shape(Handler* nextHandler) : Handler { nextHandler } { }
        void handleMessage(Event message) override
        {
            println("Shape::handleMessage()");
            if (message == Event::LeftMouseButtonDown) {
                println(" Handling message LeftMouseButtonDown");
            } else { nextHandler(message); }
        }
};
\end{cpp}

\mySubsubsection{33.10.3}{使用责任链}

上一节中实现的责任链,可以进行如下测试:

\begin{cpp}
Application application { nullptr };
Window window { &application };
Shape shape { &window };

shape.handleMessage(Event::LeftMouseButtonDown);
println("");

shape.handleMessage(Event::LeftMouseButtonUp);
println("");

shape.handleMessage(Event::RightMouseButtonDown);
println("");

shape.handleMessage(Event::RightMouseButtonUp);
\end{cpp}

输出为:

\begin{shell}
Shape::handleMessage()
    Handling message LeftMouseButtonDown

Shape::handleMessage()
Window::handleMessage()
    Handling message LeftMouseButtonUp

Shape::handleMessage()
Window::handleMessage()
Application::handleMessage()
    Handling message RightMouseButtonDown

Shape::handleMessage()
Window::handleMessage()
Application::handleMessage()
\end{shell}

实际应用程序中,必须有一个其他类来将事件分派到正确的对象,即在正确的对象上调用 handleMessage()。因为这个任务在框架或平台之间有很大的不同,以下示例显示了处理左鼠标按钮按下事件的伪代码,而不是特定于平台的 C++ 代码:

\begin{cpp}
MouseLocation location { getMouseLocation() };
Shape* clickedShape { findShapeAtLocation(location) };
if (clickedShape) {
    clickedShape->handleMessage(Event::LeftMouseButtonDown);
} else {
    window.handleMessage(Event::LeftMouseButtonDown);
}
\end{cpp}

责任链的方法非常灵活,并且对于面向对象的层次结构具有吸引人的结构。缺点是需要开发者的细心。如果一个类忘记将事件链接到链中的下一个处理者,事件将丢失。更糟糕的是,如果将事件链接到错误的类,可能会陷入无限循环!



















