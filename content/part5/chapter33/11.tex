
单例模式是设计模式中最简单的一种。英语中,“单例”意味着“独一无二的”或“唯一的”。编程中,具有类似的意义。单例模式是一种策略,用于确保程序中存在一个类的唯一实例。将单例模式应用于一个类可以保证该类类型只会创建一个对象。单例模式还规定了这个唯一对象在整个程序中都是全局可访问的,开发者通常将遵循单例模式的类称为单例类。

如果程序依赖于确切地只有一个类实例的假设,可以通过单例模式来强制这个假设。从技术上讲,可以使用全局变量和某个命名空间中的自由函数来实现相同的功能,而不是使用类,但是像 Java 这样的其他语言就没有全局变量的概念。

然而,单例模式有许多缺点。如果程序中有多个单例,有时很难保证它们在程序启动时以正确的顺序初始化,也很难保证在程序关闭时单例仍然存在。此外,单例类引入了隐藏的依赖关系,导致紧密耦合,并使单元测试变得更加复杂。例如,在单元测试中,可能想要编写一个访问网络或数据库的单例的模拟版本(参见第 30 章),但由于单例的特性,这很难做到。

一个更合适的设计模式是策略设计模式,使用策略设计模式,为提供的每个服务创建一个接口,并将组件需要的接口注入到组件中。策略模式使引入不同的实现变得容易,有助于单元测试的模拟(桩版本)。尽管如此,在这里堆单例模式进行讨论,因为在遗留代码库中会遇到它。

\begin{myWarning}{WARNING}
编写新代码时应避免使用单例模式,它存在许多问题。建议使用其他模式,例如策略设计模式。
\end{myWarning}

\mySubsubsection{33.11.1}{示例:日志机制}

许多应用程序都有一个日志的概念——一个负责将状态信息、调试数据和错误写入中央位置的类。一个日志类可能具有以下特性:

\begin{itemize}
\item
始终可用。

\item
易于使用。

\item
只有一个实例。
\end{itemize}

单例模式可以用来实现这些要求。但在新代码中,我建议避免引入新的单例。

\mySubsubsection{33.11.2}{单例的实现}

C++ 中实现单例行为有三种方法。第一种方法使用只有静态成员函数的类,这样的类不需要实例化,并且可以从程序中访问。这种技术的问题是它缺乏构建和销毁的内置机制。从技术上讲,使用所有静态成员函数的类并不真正遵循单例设计模式,而是遵循单状态设计模式,所以这样的类可以有多个实例,但只有一个状态。单例模式意味着该类只有一个实例,单状态设计模式在本节中就不再讨论。

第二种方法使用访问控制级别来控制类的创建和访问。这是真正的单例,并通过 Logger 类的示例进行说明,提供了与本章早些时候讨论的基于策略的 Logger 类似的功能。

要构建真正的单例,可以使用访问控制机制以及静态关键字。实际的 Logger 实例在运行时存在,并且类强制只创建一个实例。客户端可以通过一个名为 instance() 的静态成员函数获取那个唯一的实例。类的定义如下所示:

\begin{cpp}
export class Logger final
{
    public:
        enum class LogLevel { Debug, Info, Error };

        // Sets the name of the log file.
        // Note: needs to be called before the first call to instance()!
        static void setLogFilename(std::string logFilename);

        // Returns a reference to the singleton Logger object.
        static Logger& instance();

        // Prevent copy/move construction.
        Logger(const Logger&) = delete;
        Logger(Logger&&) = delete;

        // Prevent copy/move assignment operations.
        Logger& operator=(const Logger&) = delete;
        Logger& operator=(Logger&&) = delete;

        // Sets the log level.
        void setLogLevel(LogLevel level);

        // Logs a single message at the given log level.
        void log(std::string_view message, LogLevel logLevel);
    private:
        // Private constructor and destructor.
        Logger();
        ~Logger();

        // Converts a log level to a human-readable string.
        std::string_view getLogLevelString(LogLevel level) const;

        static inline std::string ms_logFilename;
        std::ofstream m_outputStream;
        LogLevel m_logLevel { LogLevel::Error };
};
\end{cpp}

这个实现基于Scott Meyers的单例模式,instance()成员函数包含了一个Logger类的本地静态实例。C++保证了这个本地静态实例是以线程安全的方式初始化的,所以在单例模式的这个版本中,不需要手动同步线程,这称为魔术静态变量或线程安全的静态局部变量。注意,只有初始化是线程安全的!如果多个线程将要调用Logger类的成员函数,那么也应该使Logger的成员函数本身也是线程安全的。关于如何使一个类线程安全的同步机制的详细讨论,请参见第27章。

Logger类的实现是直接的。当日志文件打开,每个日志消息都会写入文件,并附加时间戳和日志级别,然后刷新到磁盘。

当instance()成员函数中的Logger类的静态实例创建和销毁时,将自动调用构造函数和析构函数。因为构造函数和析构函数是private的,所以外部代码不能创建或删除一个Logger。

以下是setLogFilename()和instance()成员函数,以及构造函数和析构函数的实现。其他成员函数的实现与本章前面基于策略的日志器示例中的实现相同。

\begin{cpp}
void Logger::setLogFilename(string logFilename)
{ ms_logFilename = move(logFilename); }

Logger& Logger::instance()
{
    static Logger instance; // Thread-safe static local variable.
    return instance;
}

Logger::Logger()
{
    m_outputStream.open(ms_logFileName, ios_base::app);
    if (!m_outputStream.good()) {
        throw runtime_error { "Unable to initialize the Logger!" };
    }
    println(m_outputStream, "{}: Logger started.", chrono::system_clock::now());
}


Logger::~Logger()
{ println(m_outputStream, "{}: Logger stopped.", chrono::system_clock::now()); }
\end{cpp}

\mySubsubsection{33.11.3}{使用单例}

单例Logger类的测试方法如下所示:

\begin{cpp}
// Set the log filename before the first call to instance().
Logger::setLogFilename("log.out");
// Set log level to Debug.
Logger::instance().setLogLevel(Logger::LogLevel::Debug);
// Log some messages.
Logger::instance().log("test message", Logger::LogLevel::Debug);
// Set log level to Error.
Logger::instance().setLogLevel(Logger::LogLevel::Error);
// Now that the log level is set to Error, logging a Debug
// message will be ignored.
Logger::instance().log("A debug message", Logger::LogLevel::Debug);
\end{cpp}

执行后,文件 log.out 的内容为:

\begin{shell}
2023-08-12 17:36:15.1370238: Logger started.
2023-08-12 17:36:15.1372522: [DEBUG] test message
2023-08-12 17:36:15.1373057: Logger stopped.
\end{shell}





















