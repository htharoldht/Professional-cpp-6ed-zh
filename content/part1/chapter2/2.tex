
C++20之前,字符串的格式化通常是用C风格的函数(如printf())或C++ I/O流(如std::cout)完成。

\begin{itemize}
\item
C风格的函数:

\begin{itemize}
\item
不推荐使用,类型不安全,并且不能扩展支持自定义类型

\item
易于阅读,因为格式字符串和参数的分离,容易翻译成不同的语言

\item
例如:
\begin{cpp}
printf("x has value %d and y has value %d.\n", x, y);
\end{cpp}
\end{itemize}

\item
C++ I/O流:

\begin{itemize}
\item
推荐(C++20之前),类型安全且可扩展

\item
更难阅读,字符串和参数是交织在一起的,很难理解

\item
例如:
\begin{cpp}
cout << "x has value " << x << " and y has value " << y << '.' << endl;
\end{cpp}
\end{itemize}

\end{itemize}

C++20引入了<format>头文件,其中定义的std::format()来格式化字符串,结合了C风格函数和C++ I/O流的所有优点,是一种类型安全且可扩展的格式化机制:

\begin{cpp}
cout << format("x has value {} and y has value {}.", x, y) << endl;
\end{cpp}

C++23引入了std::print()和println():

\begin{cpp}
println("x has value {} and y has value {}.", x, y);
\end{cpp}

此外,std::print()和println()更好地支持将UTF-8 Unicode文本写入Unicode兼容的控制台:

\begin{cpp}
println("こんにちは世界");
\end{cpp}

这将正确地将字符串“こんにちは世界”(日语中表示“Hello World”)打印到控制台(要编译包含Unicode字符的源代码,需要传递编译器选项)。对于Visual C++,必须传递/utf-8编译器选项。对于GCC,使用命令行选项-finputcharset=UTF-8。Clang假定所有文件默认都是UTF-8格式)若尝试使用C++ I/O流输出这个字符串,根据控制台设置,输出可能会是一些乱码,例如“πüôπéôπü½πüíπü»Σ╕ûτòî”:

\begin{cpp}
cout << "こんにちは世界" << endl;
\end{cpp}

由于支持Unicode,甚至可以打印表情符号。若输控制台可以正确支持Unicode,下面的代码将打印一个笑脸符号。对此使用cout可能会输出乱码。

\newfontfamily\emojifont{Segoe UI Emoji}
println("{\emojifont 😊}");

std::print()和println()是向控制台写入文本的推荐方式,本书中的所有示例都使用了它们。它们是类型安全的,可扩展以支持自定义类型,易于阅读,支持Unicode输出,支持不同语言的本地化等。除了这些好处之外,print()和println()的性能也比直接使用C++ I/O流要好得多,print()和println()在底层仍然使用流。

\mySubsubsection{2.2.1}{格式化字符串}

std::format()、print()和println()使用格式字符串,该字符串指定在输出字符串中必须如何格式化给定参数。基本形式在前面介绍过,并且在示例中使用过,是时候看看这些格式字符串有多强大了。

格式字符串通常是format()、print()和println()的第一个参数。格式字符串可以包含一组花括号\{\},表示替换字段,可以根据需要替换的字段。format()、print()和println()的后续参数是用于填充这些替换字段的值。若在输出中需要\{和\}字符,则需要将它们转义为\{\{或\}\}。

目前为止,替换字段一直是空的花括号\{\},而这仅仅是个开始。在这些花括号内可以是格式为[index][:specifier]的字符串:

\begin{itemize}
\item
可选索引是参数索引,将在下一节讨论。

\item
可选说明符是一个格式说明符,用于规定在输出中必须如何格式化值,会在“格式说明符”一节中详细说明。
\end{itemize}

必须向format()、print()和println()传递格式字符串,不能直接打印如下值:

\begin{cpp}
int x { 42 };
println(x);
\end{cpp}

可以这样写:

\begin{cpp}
println("{}", x);
\end{cpp}

也不能只写下面的代码,使其打印一个换行符:

\begin{cpp}
println();
\end{cpp}

而需要这样使用:

\begin{cpp}
println("");
\end{cpp}

\mySubsubsection{2.2.2}{参数索引}

可以在所有替换字段中省略索引,或者为所有替换字段指定传递给format()、print()或println()的第二个及后续参数的索引,这些参数应该用于替换字段。若想要多次输出某个值,可以多次使用某个特定的索引。若省略了索引,则传递的第二个及后续参数,将按给定的顺序用于所有替换字段。

下面的 println() 调用省略了替换字段中的显式索引:

\begin{cpp}
int n { 42 };
println("Read {} bytes from {}", n, "file1.txt");
\end{cpp}

可以按如下方式指定手动索引:

\begin{cpp}
println("Read {0} bytes from {1}", n, "file1.txt");
\end{cpp}

不允许混合使用手动索引和自动索引。下面使用了一个无效的格式字符串:

\begin{cpp}
println("Read {0} bytes from {}", n, "file1.txt");
\end{cpp}

可以更改输出字符串中格式化值的顺序,而不必更改参数的实际顺序。若想在软件中翻译字符串,这是一个有用的功能。某些语言在句子中有不同的语序。例如,前面的格式字符串可以翻译为中文。中文中,句子中替换字段的顺序是颠倒的,但由于在格式字符串中使用了参数索引,println()的参数顺序可以保持不变。

\begin{cpp}
println("从{1}中读取{0}个字节。 ", n, "file1.txt");
\end{cpp}

\mySubsubsection{2.2.3}{输出到不同地方}

目前为止,每次调用print()和println()都有一个格式字符串作为第一个参数,后面跟着一些附加参数:

\begin{cpp}
println("x has value {} and y has value {}.", x, y);
\end{cpp}

这将字符串打印到标准输出流,即与std::cout相同的流。

还有std::cerr,传输到标准错误控制台。可以使用print()和println()打印到错误控制台:

\begin{cpp}
println(cerr, "x has value {} and y has value {}.", x, y);
\end{cpp}

\mySubsubsection{2.2.4}{编译时验证字符串的格式}

\CXXTwentythreeLogo{-40}{15}

从C++23起,format()、print()、 和println()的格式字符串必须是编译时常量,这样编译器就可以在编译时检查格式字符串中是否有语法错误,所以以下代码无法编译:

\begin{cpp}
string s { "Hello World!" };
println(s); // Error! Does not compile
\end{cpp}

所产生的错误依赖于编译器。但在撰写本文时,其标准相当模糊,并不总是有助于立即确定错误的确切原因。以下是Microsoft Visual C++ 2022编译器产生的错误:

\begin{shell}
error C7595: 'std::basic_format_string<char>::basic_format_string': call to immediate function is not a constant expression
\end{shell}

正确用法如下:

\begin{cpp}
string s { "Hello World!" };
println("{}", s);
\end{cpp}

因为是编译时常量,所以允许使用constexpr格式的字符串。第9章将详细介绍constexpr关键字。

\begin{cpp}
constexpr auto formatString { "Value: {}" };
println(formatString, 11); // Value: 11
\end{cpp}

\mySamllsection{非编译时常量格式字符串}

当需要为不同的语言本地化/翻译格式字符串时,格式字符串必须是编译时常量可能有点麻烦,可以使用std::vprint\_unicode()或std::vprint\_nonunicode()来代替std::print()。不过用起来有点难,不能像使用print()那样只传递参数,而是需要使用std::make\_format\_args()协助。这里有一个例子:

\begin{cpp}
enum class Language { English, Dutch };

string_view GetLocalizedFormat(Language language)
{
    switch (language) {
        case Language::English: return "Numbers: {0} and {1}.";
        case Language::Dutch: return "Getallen: {0} en {1}.";
    }
}

int main()
{
    Language language { Language::English };
    vprint_unicode(GetLocalizedFormat(language), make_format_args(1, 2));
    println("");
    language = Language::Dutch;
    vprint_unicode(GetLocalizedFormat(language), make_format_args(1, 2));
}
\end{cpp}

输出为:

\begin{shell}
Numbers: 1 and 2.
Getallen: 1 en 2.
\end{shell}

需要一个编译时常量格式字符串,下面使用print()的代码不能编译:

\begin{cpp}
print(GetLocalizedFormat(language), 1, 2);
\end{cpp}

\mySamllsection{处理非编译时常量格式字符串中的错误}

在运行时时验证格式字符串时,格式字符串错误都会引发std::format\_error异常。std::format()、print()和println()等函数永远不会抛出这样的异常,因为格式字符串都是在编译时验证的。但像std::vformat()和vprint\_unicode()(参见上一节)这样的函数不要求格式字符串是常量,因此在编译时不进行验证,而是在运行时验证,这些函数可能抛出format\_error异常:

\begin{cpp}
try {
    vprint_unicode("An integer: {5}", make_format_args(42));
} catch (const format_error& caught_exception) {
    println("{}", caught_exception.what()); // "Argument not found."
}
\end{cpp}

现在,来研究一下格式说明符到底有多强大。

\mySubsubsection{2.2.5}{格式说明符}

如前所述,格式字符串可以包含用花括号分隔的替换字段,这些花括号内可以是格式为[index][:specifier]的字符串。本节讨论替换字段的格式说明符部分。

格式说明符用于操作输出中值的格式化方式。格式说明符以冒号:作为前缀。格式说明符的一般形式如下:

\begin{shell}
[[fill]align][sign][#][0][width][.precision][L][type]
\end{shell}

方括号之间的所有部分都是可选的,各个说明符部分将在下一小节中讨论。

\mySamllsection{宽度}

宽度指定应该将给定值格式化为的字段的最小宽度。

这也可以是另一组花括号,称为动态宽度。若在花括号中指定了索引,例如\{3\},则动态宽度的值将从具有给定索引的参数中获取。若没有指定索引(如\{\}),则从参数列表中的下一个参数中获取宽度。

下面是一些例子:

\begin{cpp}
int i { 42 };
println("|{:5}|", i);        // |    42|
println("|{:{}}|", i, 7);    // |       42|
println("|{1:{0}}|", 7, i);  // |       42|
\end{cpp}


\mySamllsection{[fill]align}

[fill]align部分可选地说明要使用什么字符作为填充字符,然后是值在其字段中应该如何对齐:

\begin{itemize}
\item
< 表示左对齐(默认为非整数和非浮点数)。

\item
> 表示右对齐(默认为整数和浮点数)。

\item
\^{} 表示居中对齐。
\end{itemize}

将填充字符插入到输出中,以确保输出中的字段达到指定符[width]部分指定的所需最小宽度。若没有指定[width],则[fill]align不起作用。

当使用居中对齐时,在格式化值的左边和右边有相同数量的填充字符。若填充字符总数为奇数,则在右侧添加额外的填充字符。

下面是一些例子:

\begin{cpp}
int i { 42 };
println("|{:7}|", i);     // |     42|
println("|{:<7}|", i);    // |42     |
println("|{:_>7}|", i);   // |_____42|
println("|{:_^7}|", i);   // |__42___|
\end{cpp}

下面是一个有趣的技巧,可以输出特定次数的字符。不需要自己键入包含正确字符数的字符串字面值,而是在格式说明符中显式指定所需的字符数:

\begin{cpp}
println("|{:=>16}|", ""); // |================|
\end{cpp}

\mySamllsection{符号}

符号可以是下列形式之一:

\begin{itemize}
\item
- 表示仅显示负数的符号(默认值)。

\item
+ 表示显示负数和正数的符号。

\item
空格表示着负号应该用在负数上,空格应该用在正数上。
\end{itemize}

下面是一些例子:

\begin{cpp}
int i { 42 };
println("|{:<5}|", i);   // |42  |
println("|{:<+5}|", i);  // |+42 |
println("|{:< 5}|", i);  // | 42 |
println("|{:< 5}|", -i); // |-42 |
\end{cpp}

\mySamllsection{\#}

\#部分启用备用格式化规则。若对整型启用了该功能,并且还指定了十六进制、二进制或八进制数字格式,则替代格式将在格式化的数字前面插入0x、0x、0b、0b或0。若为浮点类型启用,则替代格式将始终输出一个小数分隔符,即使后面没有数字。

下面两个部分给出了其他格式的示例。

\mySamllsection{类型}

类型指定给定值必须格式化的类型。有几个选项:

\begin{itemize}
\item
整数类型:b(二进制),b(二进制,如果指定了\#,则用0B代替0B), d(十进制),o(八进制),x(十六进制,小写字母a, b, c, d, e, f), x(十六进制,大写字母a, b, c, d, e, f,如果指定了\#,则用0X代替0X)。若类型未指定,则d用于整型。

\item
浮点类型:支持以下浮点格式。科学格式、固定格式、通用格式和十六进制格式的结果,与本章前面讨论的std::chars\_format::scientific、固定格式、通用格式和十六进制格式的结果相同。

\begin{itemize}
\item
e, E: 以小写e或大写e作为指数表示的科学记数法,格式为给定精度,如果未指定精度则为6。

\item
f, F: 固定格式表示法为给定精度或6(若没有指定精度)。

\item
g, G: 通用表示法自动选择不带指数(固定格式)或带指数(小写e或大写e)的表示,使用给定的精度或6(若没有指定精度)进行格式化。

\item
a, A: 由小写字母(a)或大写字母(a)组成的十六进制表示法

\item
若类型未指定,则g用于浮点类型。
\end{itemize}

\item
布尔值:s(以文本形式输出真或假),b, B, c, d, o, x, X(以整数形式输出1或0)。若类型未指定,则s用于布尔类型。

\item
字符:c(字符复制到输出),?(转义字符复制到输出),b, B, d, o, x, X(整数表示)。若类型未指定,则c用于字符类型。

\item
字符串:s(字符串复制到输出),?(转义字符串复制到输出)。若类型未指定,则s用于字符串类型。

\item
指针:p(以0x为前缀的指针的十六进制表示法)。若类型未指定,则p用于指针类型。只有void*类型的指针可以被格式化。其他指针类型必须首先转换为void*类型,例如使用static\_cast<void*>(myPointer)。
\end{itemize}

下面是一些整型的例子:

\begin{cpp}
int i { 42 };
println("|{:10d}|", i);   // |        42|
println("|{:10b}|", i);   // |    101010|
println("|{:#10b}|", i);  // |  0b101010|
println("|{:10X}|", i);   // |        2A|
println("|{:#10X}|", i);  // |      0X2A|
\end{cpp}

下面是一个字符串类型的例子:

\begin{cpp}
string s { "ProCpp" };
println("|{:_^10}|", s); // |__ProCpp__|
\end{cpp}

关于精度的下一节将给出浮点类型的示例。

\mySamllsection{精度}

精度只能用于浮点型和字符串类型。其指定为一个点,后面跟着要为浮点类型输出的十进制数,或者为字符串输出的字符数。浮点类型的位数包括所有数字,包括小数点分隔符之前的数字,除非使用固定浮点表示法(对于f),精度是小数点后的位数。

与宽度一样,精度也可以是另一组花括号,这称为动态精度。精度要么来自参数列表中的下一个参数,要么来自给定索引的参数。

以下是一些使用浮点类型的例子:

\begin{cpp}
double d { 3.1415 / 2.3 };
println("|{:12g}|", d);   // |     1.36587|
println("|{:12.2}|", d);  // |         1.4|
println("|{:12e}|", d);   // |1.365870e+00|

int width { 12 };
int precision { 3 };
println("|{2:{0}.{1}f}|", width, precision, d); // |     1.366|
println("|{2:{0}.{1}}|", width, precision, d);  // |      1.37|
\end{cpp}

\mySamllsection{0}

说明符的0部分,对于数值要将零插入到格式化的值中,以达到说明符的[width]部分指定的所需最小宽度(见前面)。这些零插入到数值的前面,但是在正负号之后,以及在任何0x、0x、0b或0b前缀之后。若指定了对齐方式,则忽略0说明符。

下面是一些例子:

\begin{cpp}
int i { 42 };
println("|{:06d}|", i);   // |000042|
println("|{:+06d}|", i);  // |+00042|
println("|{:06X}|", i);   // |00002A|
println("|{:#06X}|", i);  // |0X002A|
\end{cpp}

\mySamllsection{L}

可选的L说明符支持特定于语言环境的格式化。此选项仅对算术类型有效,例如整数、浮点类型和布尔类型。当与整数一起使用时,L选项指定必须使用特定于语言环境的数字组分隔符。对于浮点类型,使用特定于语言环境的数字组和十进制分隔符。对于以文本形式输出的布尔类型,使用特定于语言环境的true和false表示。

使用L说明符时,必须将std::locale实例作为第一个参数传递给std::format()。这只适用于format(),不适用于print()和println()。下面是一个使用nl语言环境格式化浮点数的示例:

\begin{cpp}
float f { 1.2f };
cout << format(std::locale{ "nl" }, "|{:Lg}|\n", f); // |1,2|
\end{cpp}

本地化将在第21章中讨论。

\CXXTwentythreeLogo{-40}{-50}

\mySubsubsection{2.2.6}{格式化转义字符和字符串}

C++23可以使用?类型说明符。这个用例并不经常发生,但对于日志记录和调试很有帮助。输出类似于在代码中编写字符串和字符字面量的方式:以双引号或单引号开始和结束,并且使用转义字符序列。下表显示了使用转义格式时某些字符的输出:

% Please add the following required packages to your document preamble:
% \usepackage{longtable}
% Note: It may be necessary to compile the document several times to get a multi-page table to line up properly
\begin{longtable}{|l|l|}
\hline
字符       & 转义输出                   \\ \hline
\endfirsthead
%
\endhead
%
水平制表符  & \textbackslash{}t                \\ \hline
换行        & \textbackslash{}n                \\ \hline
回车 & \textbackslash{}r                \\ \hline
反斜杠       & \textbackslash{}\textbackslash{} \\ \hline
双引号    & \textbackslash{}"                \\ \hline
单引号    & \textbackslash{}'                \\ \hline
\end{longtable}

双引号的转义仅在输出为双引号字符串时发生,而单引号的转义仅在输出为单引号字符时发生。不可输出字符的转义输出为\textbackslash{}u\{十六进制数\}。

下面是一些例子:

\begin{cpp}
println("|{:?}|", "Hello\tWorld!\n");   // |Hello\tWorld!\n|
println("|{:?}|", "\"");                // |"\""|
println("|{:?}|", '\'');                // |'\''|
println("|{:?}|", '"');                 // |'"'|
\end{cpp}

\CXXTwentythreeLogo{-40}{-50}

\mySubsubsection{2.2.7}{格式范围}

第 1 章介绍了std::vector、array和pair容器来存储多个数据元素。C++23起,可以直接格式化这些元素的范围。对于如vector和array这样的范围,默认输出由方括号包围,各个元素由逗号分隔。若范围的元素是字符串,其输出默认进行转义。

范围的格式化可以通过嵌套格式说明符来控制。一般形式为:

\begin{shell}
[[fill]align][width][n][range-type][:range-underlying-spec]
\end{shell}

方括号之间的内容可选。与其他格式说明符一样,fill指定填充字符,align指定输出的对齐方式,width指定输出字段的宽度。若指定了n,输出将不包含范围的左括号和右括号。范围类型可以是以下类型:

% Please add the following required packages to your document preamble:
% \usepackage{longtable}
% Note: It may be necessary to compile the document several times to get a multi-page table to line up properly
\begin{longtable}{|l|l|}
\hline
\textbf{范围类型} & \textbf{描述}                                                                                                                     \\ \hline
\endfirsthead
%
\endhead
%
m &
\begin{tabular}[c]{@{}l@{}}仅对包含两个元素的pair和tuple可用。默认情况下,用圆括号括起来,\\并用逗号分 隔。若指定了m,则不用括号包围,并且两个元素用":"分隔。\end{tabular} \\ \hline
s                   & \begin{tabular}[c]{@{}l@{}}将范围格式化为字符串(不能与n或范围基本规格组合)。\end{tabular}          \\ \hline
?s                  & \begin{tabular}[c]{@{}l@{}}将范围格式化为转义字符串(不能与n或范围基础规范结合使用)\end{tabular} \\ \hline
\end{longtable}

范围基础规范是范围各个元素的可选格式说明符,范围说明符可以嵌套多个层次。若元素仍然是范围(例如,嵌套vector),则范围基础规范是另一个范围格式说明符,依此类推。

来看一些例子,格式化vector<int>:

\begin{cpp}
vector values { 11, 22, 33 };
println("{}", values); // [11, 22, 33]
println("{:n}", values); // 11, 22, 33
\end{cpp}

若要替换开始和结束方括号,可以将格式说明符的n说明符与开始和结束字符结合起来。下面的代码将输出用花括号括起来,希望在输出中出现的大括号需要转义为\{\{和\}\}。

\begin{cpp}
println("{{{:n}}}", values); // {11, 22, 33}
\end{cpp}

下面提供了整个范围的格式说明符。对于这两种情况,范围都在16个字符宽的字段的中心输出,并以*作为填充字符。对于第二个,n指定左括号和右括号应该省略:

\begin{cpp}
println("{:*^16}", values); // **[11, 22, 33]**
println("{:*^16n}", values); // ***11, 22, 33***
\end{cpp}

下面的代码没有为整个范围提供显式的说明符,但指定了如何格式化单个元素,单个元素输出在6个字符宽的字段的中心,*作为填充字符:

\begin{cpp}
println("{::*^6}", values); // [**11**, **22**, **33**]
\end{cpp}

这也可以与n指示符结合使用:

\begin{cpp}
println("{:n:*^6}", values); // **11**, **22**, **33**
\end{cpp}

下面是一些格式化字符串vector的示例:

\begin{cpp}
vector strings { "Hello"s, "World!\t2023"s };
println("{}", strings);     // ["Hello", "World!\t2023"]
println("{:}", strings);    // ["Hello", "World!\t2023"]
println("{::}", strings);   // [Hello, World!   2023]
println("{:n:}", strings);  // Hello, World!   2023
\end{cpp}

若有一个字符vector,可以将它们格式化为单个字符,或者使用s或?s范围类型将整个vector视为字符串:

\begin{cpp}
vector chars { 'W', 'o', 'r', 'l', 'd', '\t', '!' };
println("{}", chars);       // ['W', 'o', 'r', 'l', 'd', '\t', '!']
println("{::#x}", chars);   // [0x57, 0x6f, 0x72, 0x6c, 0x64, 0x9, 0x21]
println("{:s}", chars);     // World    !
println("{:?s}", chars);    // "World\t!"
\end{cpp}

下面是一些输出pair的示例。默认情况下,使用圆括号包围pair,而不是方括号,两个元素用逗号分隔。使用n说明符将删除左括号和右括号。m说明符还删除括号并用":"分隔元素。

\begin{cpp}
pair p { 11, 22 };
println("{}", p);     // (11, 22)
println("{:n}", p);   // 11, 22
println("{:m}", p);   // 11: 22
\end{cpp}

最后,这里有一些输出嵌套vector的例子:

\begin{cpp}
vector<vector<int>> vv { {11, 22}, {33, 44, 55} };
println("{}", vv);          // [[11, 22], [33, 44, 55]]
println("{:n}", vv);        // [11, 22], [33, 44, 55]
println("{:n:n}", vv);      // 11, 22, 33, 44, 55
println("{:n:n:*^4}", vv);  // *11*, *22*, *33*, *44*, *55*
\end{cpp}

\mySubsubsection{2.2.8}{支持自定义类型}

可以扩展格式化库以添加对自定义类型的支持。这涉及到编写std::formatter类模板的特化,其中包含两个成员函数模板:parse()和format()。我知道,在这里各位读者可能还没有理解这个例子中的所有语法,因为它使用了以下技术:

\begin{itemize}
\item
constexpr函数,第9章内容

\item
模板特化、成员函数模板和简化函数模板语法,第12章内容

\item
异常,第14章内容

\item
迭代器,第17章内容
\end{itemize}

为了完整起见,并让各位体验一下,来看看如何实现自定义格式化器,可以回到这个示例。

假设有下面的类来存储一个键值对:

\begin{cpp}
class KeyValue
{
    public:
        KeyValue(string_view key, int value) : m_key { key }, m_value { value } {}

        const string& getKey() const { return m_key; }
        int getValue() const { return m_value; }
    private:
        string m_key;
        int m_value { 0 };
};
\end{cpp}

KeyValue对象的自定义格式化程序,可以通过编写以下类模板特化来实现。这个KeyValue格式化器支持:

\begin{itemize}
\item
自定义格式说明符:\{:k\}只输出键,\{:v\}只输出值,\{:b\}和\{\}同时输出键和值。

\item
嵌套格式说明符:为键和/或值指定可选格式。语法为:\{:b:KeyFormat:ValueFormat\}。

\begin{cpp}
template <>
class std::formatter<KeyValue>
{
    public:
        constexpr auto parse(auto& context)
        {
            string keyFormat, valueFormat;
            size_t numberOfParsedColons { 0 };
            auto iter { begin(context) };
            for (; iter != end(context); ++iter) {
                if (*iter == '}') { break; }

                if (numberOfParsedColons == 0) { // Parsing output type
                    switch (*iter) {
                        case 'k': case 'K': // {:k format specifier
                            m_outputType = OutputType::KeyOnly; break;
                        case 'v': case 'V': // {:v format specifier
                            m_outputType = OutputType::ValueOnly; break;
                        case 'b': case 'B': // {:b format specifier
                            m_outputType = OutputType::KeyAndValue; break;
                        case ':':
                            ++numberOfParsedColons; break;
                        default:
                            throw format_error { "Invalid KeyValue format." };
                    }
                } else if (numberOfParsedColons == 1) { // Parsing key format
                    if (*iter == ':') { ++numberOfParsedColons; }
                    else { keyFormat += *iter; }
                } else if (numberOfParsedColons == 2) { // Parsing value format
                    valueFormat += *iter;
                }
            }
            // Validate key format specifier.
            if (!keyFormat.empty()) {
                format_parse_context keyFormatterContext { keyFormat };
                m_keyFormatter.parse(keyFormatterContext);
            }
            // Validate value format specifier.
            if (!valueFormat.empty()) {
                format_parse_context valueFormatterContext { valueFormat };
                m_valueFormatter.parse(valueFormatterContext);
            }
            if (iter != end(context) && *iter != '}') {
                throw format_error { "Invalid KeyValue format." };
            }
            return iter;
        }

        auto format(const KeyValue& kv, auto& ctx) const
        {
            switch (m_outputType) {
                using enum OutputType;
                case KeyOnly:
                    ctx.advance_to(m_keyFormatter.format(kv.getKey(), ctx));
                    break;
                case ValueOnly:
                    ctx.advance_to(m_valueFormatter.format(kv.getValue(), ctx));
                    break;
                default:
                    ctx.advance_to(m_keyFormatter.format(kv.getKey(), ctx));
                    ctx.advance_to(format_to(ctx.out(), " - "));
                    ctx.advance_to(m_valueFormatter.format(kv.getValue(), ctx));
                    break;
                }
                return ctx.out();
            }
        private:
            enum class OutputType { KeyOnly, ValueOnly, KeyAndValue };
            OutputType m_outputType { OutputType::KeyAndValue };
            formatter<string> m_keyFormatter;
            formatter<int> m_valueFormatter;
};
\end{cpp}
\end{itemize}

parse()成员函数负责解析给定的字符范围[begin(context), end(context))中的格式说明符,将解析的结果存储在formatter类的数据成员中,并返回一个指向解析格式说明符字符串末尾之后字符的迭代器。其中两个数据成员是m\_keyFormatter类型为formatter<string>和m\_valueFormatter类型为formatter<int>,分别用于处理格式说明符的KeyFormat和ValueFormat部分。

format()成员函数根据parse()解析的格式规范格式化第一个参数给定的值,将结果写入ctx.out(),并返回一个指向输出末尾的迭代器。该函数使用std::format\_to(),与std::format()类似,不同之处在于它接受一个输出迭代器,指示输出应该写入的位置。

测试KeyValue格式化器:

\begin{cpp}
const size_t len { 34 }; // Label field length
KeyValue kv { "Key 1", 255 };
println("{:>{}} {}", "Default:", len, kv);
println("{:>{}} {:k}", "Key only:", len, kv);
println("{:>{}} {:v}", "Value only:", len, kv);
println("{:>{}} {:b}", "Key and value with default format:", len, kv);
println("{:>{}} {:k:*^11}", "Key only with special format:", len, kv);
println("{:>{}} {:v::#06X}", "Value only with special format:", len, kv);
println("{:>{}} {::*^11:#06X}", "Key and value with special format:", len, kv);
try {
    auto formatted { vformat("{:cd}", make_format_args(kv)) };
    println("{}", formatted);
} catch (const format_error& caught_exception) {
    println("{}", caught_exception.what());
}
\end{cpp}

输出为:

\begin{shell}
                          Default: Key 1 - 255
                         Key only: Key 1
                       Value only: 255
Key and value with default format: Key 1 - 255
     Key only with special format: ***Key 1***
   Value only with special format: 0X00FF
Key and value with special format: ***Key 1*** - 0X00FF
Invalid KeyValue format.
\end{shell}

作为练习,可以在键和值之间添加对不同分隔符的支持。使用自定义格式化器,则具有无限可能,并且一切都是类型安全的!



















