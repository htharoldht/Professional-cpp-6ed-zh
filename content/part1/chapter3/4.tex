
C++编译器有一些命名规则:

\begin{itemize}
\item
名称可以包含大写和小写字母、数字和下划线。

\item
字母不仅限于英文字母,可以是任何语言的字母,例如日语、阿拉伯语等。

\item
名称不能以数字开头(例如,9to5)。

\item
包含双下划线的名称会(例如 my\_\_name)保留给标准库使用,不应使用。

\item
以下划线后跟大写字母开头的名称(例如 \_Name)始终保留给标准库使用,不应使用。

\item
全局命名空间中以下划线开头的名称(例如 \_name)保留给标准库,不应使用。
\end{itemize}

除了这些规则,名称仅仅是为了帮助开发者们使用程序中的各个元素,但开发者经常使用不具体或不恰当的名称。

\mySubsubsection{3.4.1}{起一个好名字}

变量、成员函数、函数、参数、类、命名空间的最佳名称应准确描述相应的目的。名称还可以带有其他信息,例如类型或特定用法。当然,真正的考验是其他开发者是否能理解用特定名称试图传达的意思。

除了适用于所在组织的规则之外,没有一成不变的命名规则。不过,有些名称怎么看都不恰当,下表展示了一些极端的命名:

% Please add the following required packages to your document preamble:
% \usepackage{longtable}
% Note: It may be necessary to compile the document several times to get a multi-page table to line up properly
\begin{longtable}{|l|l|}
\hline
\textbf{良好的名称} &
\textbf{糟糕的名称} \\ \hline
\endfirsthead
%
\endhead
%
\begin{tabular}[c]{@{}l@{}}sourceName, destinationName\\ 很好的进行区分\end{tabular} &
\begin{tabular}[c]{@{}l@{}}thing1, thing2\\ 过于宽泛\end{tabular} \\ \hline
\begin{tabular}[c]{@{}l@{}}m\_nameCounter\\ 传递数据成员状态\end{tabular} &
\begin{tabular}[c]{@{}l@{}}m\_NC\\ 太模糊,太简短了\end{tabular} \\ \hline
\begin{tabular}[c]{@{}l@{}}calculateMarigoldOffset()\\ 简单、准确\end{tabular} &
\begin{tabular}[c]{@{}l@{}}doAction()\\ 太笼统,不准确\end{tabular} \\ \hline
\begin{tabular}[c]{@{}l@{}}m\_typeString\\ 轻松悦目\end{tabular} &
\begin{tabular}[c]{@{}l@{}}typeSTR256\\ 只有电脑才会喜欢的名字\end{tabular} \\ \hline
\begin{tabular}[c]{@{}l@{}}g\_settings\\ 表示全局状态\end{tabular} &
\begin{tabular}[c]{@{}l@{}}m\_IHateLarry\\ 不可接受的内部笑话\end{tabular} \\ \hline
\begin{tabular}[c]{@{}l@{}}errorMessage\\ 描述性名称\end{tabular} &
\begin{tabular}[c]{@{}l@{}}string\\ 非描述性名称\end{tabular} \\ \hline
\begin{tabular}[c]{@{}l@{}}sourceFile, destinationFile\\ 非缩写\end{tabular} &
\begin{tabular}[c]{@{}l@{}}srcFile, dstFile\\ 缩写\end{tabular} \\ \hline
\end{longtable}

\mySubsubsection{3.4.2}{命名规则}

选择一个名称并不总是需要大量的思考和创造力。很多情况下,有可以借鉴的标准命名技巧,以下是一些可以使用标准名称的数据类型。

\mySamllsection{计数器}

在编程生涯早期,可能见过使用变量 i 作为计数器的代码。通常习惯上使用 i 和 j 分别作为计数器和内层循环计数器,但在使用嵌套循环时要小心。一个常见的错误是在需要“jth”元素时,却引用了“ith”元素。处理二维矩阵时,使用 row 和 column 作为索引而不是 i 和 j 可能更容易理解。一些开发者更喜欢使用计数器 outerLoopIndex 和 innerLoopIndex,甚至有些开发者对使用 i 和 j 作为循环计数器表示强烈不满。

\mySamllsection{前缀}

许多开发者会以提供变量类型或用途信息的字母开头命名他们的变量,也有同样多甚至更多的开发则不赞成使用类型前缀,这可能会使未来演变的代码更难以维护。若一个成员变量从静态更改为非静态,需要重命名所有该名称的使用。若不重命名,现有的名称将继续传达语义,但提供了错误的语义。

有时这没得选,需要遵循组织的指导。下表展示了一些可能会使用的前缀:

% Please add the following required packages to your document preamble:
% \usepackage{longtable}
% Note: It may be necessary to compile the document several times to get a multi-page table to line up properly
\begin{longtable}{|l|l|l|l|}
\hline
\textbf{前缀} &
\textbf{示例} &
\textbf{前缀含义} &
\textbf{用法} \\ \hline
\endfirsthead
%
\endhead
%
\begin{tabular}[c]{@{}l@{}}m\\ m\_\end{tabular} &
\begin{tabular}[c]{@{}l@{}}mData\\ m\_data\end{tabular} &
“member” &
类中的数据成员 \\ \hline
\begin{tabular}[c]{@{}l@{}}s\\ ms\\ ms\_\end{tabular} &
\begin{tabular}[c]{@{}l@{}}sLookupTable\\ msLookupTable\\ ms\_lookupTable\end{tabular} &
“static” &
静态变量或数据成员 \\ \hline
k &
kMaximumLength &
\begin{tabular}[c]{@{}l@{}}“konstant” \\(德语中的“constant”)\end{tabular} &
\begin{tabular}[c]{@{}l@{}}一个常量值,一些开发者会省略前缀来\\表示常量。
\end{tabular} \\ \hline
\begin{tabular}[c]{@{}l@{}}b\\ is\end{tabular} &
\begin{tabular}[c]{@{}l@{}}bCompleted\\ isCompleted\end{tabular} &
“Boolean” &
指定一个布尔值 \\ \hline
\end{longtable}

\mySamllsection{匈牙利命名法}

匈牙利命名法是一种变量和数据成员命名约定,它在微软Windows开发者中很受欢迎。基本思想是,使用更详细的的前缀来表示变量信息,而不是使用单个字母前缀,如 m。以下代码行展示了匈牙利命名法的使用:

\begin{cpp}
char* pszName; // psz means "pointer to string, zero-terminated"
\end{cpp}

匈牙利命名法这个术语源于这样一个事实:发明者查尔斯·西蒙尼(Charles Simonyi)是匈牙利人。有些人还说,使用匈牙利命名法的程序看起来好像是用外语编写的。由于这个后一个原因,一些开发者不喜欢匈牙利命名法。本书中使用了前缀,但不是匈牙利命名法。充分命名的变量除了前缀之外不需要太多的上下文信息,一个名为 m\_name 的数据成员就足以说明一切。

\begin{myNotic}{NOTE}
好的名称能够传达关于它们的目的,而不会使代码变得难以阅读。
\end{myNotic}

\mySamllsection{getter和setter}

若类包含一个如 m\_status 的数据成员,通常的做法是提供一个名为 getStatus() 的 getter 方法来访问该成员,以及一个可选的名为 setStatus() 的 setter 方法。为了访问一个布尔类型的数据成员,通常使用 is 作为前缀而不是 get,例如 isRunning()。C++ 没有规定这些函数的命名,但组织内可能希望采用这种或类似的命名方案。

\mySamllsection{大写}

代码中命名时,有许多不同的方式来使用大写字母。与编码风格的大多数元素一样,团队需要采用标准化的方法,并且所有成员都遵循这种方式。一种让代码变得混乱的方式是让一些开发者将类名全部使用小写字母,并用下划线表示空格(priority\_queue),而其他开发者使用大写字母,并且每个后续的单词都大写(PriorityQueue)。变量和数据成员总是以小写字母开头,并使用下划线(my\_queue)或大写字母(myQueue)来表示单词分隔。C++函数传统上以大写字母开头,但本书中,我会使用小写字母开头的风格来区分函数名和类名。

\mySamllsection{命名空间常量}

假设正在编写一个带有图形用户界面的程序。该程序有几个菜单,包括文件(File)、编辑(Edit)和帮助(Help)。为了表示每个菜单的ID,可能会决定使用一个常量。对于一个指向帮助菜单ID的常量,一个完全合理的名称是Help。

Help这个名字在添加一个帮助按钮到主窗口之前,都能很好地工作。还需要一个常量来引用按钮的ID,但是Help已经使用了。

解决这个问题的一个方案是将常量放在不同的命名空间中,这在第1章中讨论过。可以创建了两个命名空间:Menu和Button。每个命名空间都有一个Help常量,可以使用它们作为Menu::Help和Button::Help。更推荐的解决方案是使用枚举,也在第1章中介绍过。


