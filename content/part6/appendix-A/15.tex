
标准库的某些方面可能难以处理,很少有面试官会期望你背诵标准库类的详细信息,除非你声称自己是标准库的专家。如果知道面试的工作会大量使用标准库,可能会想在面试前一天写一些标准库代码来刷新记忆。否则,回忆标准库的高级设计和基本用法应该就足够了。

\mySubsubsection{A.15.1}{需要注意的事项}

\begin{itemize}
\item
不同类型的容器,及其与迭代器的关联

\item
使用vector,这是最常用的标准库类

\item
span类,以及为什么使用

\item
什么是mdspan(C++23)

\item
使用关联容器,如map

\item
关联容器与其他关联容器(例如,map、无序关联容器(例如,unordered\_map)以及扁平关联容器适配器(C++23),例如flat\_map)之间的区别

\item
如何与函数指针、函数对象(可调用对象)和Lambda表达式一起工作

\item
什么是透明操作符函数

\item
标准库算法,及其内置算法的目的

\item
Lambda表达式与标准库算法的组合使用

\item
remove-erase模式

\item
许多标准库算法都可以并行执行以提高性能

\item
如何扩展标准库(不必要了解细节)

\item
范围、投影、视图和范围工厂是什么

\item
范围库的表达能力

\item
对标准库的看法
\end{itemize}

\mySubsubsection{A.15.2}{问题的类型}

如果面试官坚持要问详细的标准库问题,真的可以问任何类型的问题。如果对语法不确定,应该在面试中直言不讳:“在现实生活中,我当然会在《Professional C++》中查找那个细节,但我相当确定它像这样工作……”至少这样,面试官会提醒应该原谅细节,只要把基本概念搞对就好。

关于标准库的高级问题通常用来衡量使用标准库的程度,而不需要回忆所有细节。标准库的普通用户可能熟悉关联和非关联容器。一个稍微高级的用户能够定义一个迭代器,描述迭代器如何与容器一起工作,并描述remove-erase模式。其他高级问题可能要求你谈论你对标准库算法的经验,或者是否自定义了标准库。面试官还可能衡量对函数对象和Lambda表达式的了解,以及它们与标准库算法的使用。当谈论Lambda表达式时,如果解释了使用auto关键字来定义泛型Lambda表达式的用法,将获得额外分数。

可能会要求解释使用范围库的好处,可以编写描述想要做什么的代码。













