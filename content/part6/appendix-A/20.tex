
如果完成了与一家公司的整个面试流程,面试官没有问任何关于流程的问题,应该持怀疑态度——这可能说明他们没有制定流程,或者他们根本不在乎流程。另一种可能是,他们不想用流程巨兽吓跑你。

制定一个好的流程是非常重要的,任何规模的项目的版本控制都应该强制执行。

大多数时候,会有机会向公司提出问题。我建议将公司工程流程和版本控制解决方案,作为标准问题之一。

\mySubsubsection{A.20.1}{需要注意的事项}

\begin{itemize}
\item
传统的生命周期模型

\item
不同模型的权衡

\item
极限编程背后的主要原则

\item
Scrum作为敏捷过程的例子

\item
过去使用过的其他流程

\item
什么是版本控制
\end{itemize}

\mySubsubsection{A.20.2}{问题的类型}

问得最多的一个问题可能是描述以前雇主使用的流程,不要泄露机密信息。回答时,应该提到哪些方面做得好,哪些方面失败了,但尽量不要批评特定的方法论。你讨厌的方法论可能是面试官使用的。

几乎每个候选人都将Scrum/敏捷作为一项技能列出来。如果面试官问你关于Scrum的问题,她可能不想让你简单地背诵教科书上的定义——面试官知道可以阅读Scrum书籍的目录,而从Scrum中挑选几个有吸引力的想法。向面试官解释每个想法,并表达看法。尝试与面试官进行对话,根据她给出的线索,沿着她感兴趣的方向进行。

如果问到有关版本控制的问题,很可能是高层次的问题。应该解释为什么应该使用版本控制以及它的好处,还可以解释本地、客户端/服务器和分布式解决方案之间的区别,并可能解释前雇主是如何实施版本控制的。

