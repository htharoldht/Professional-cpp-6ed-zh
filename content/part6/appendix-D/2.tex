类图用于可视化单个类,并可以包括数据成员和成员函数,还用于展示不同类之间的关系。

\mySubsubsection{D.2.1}{类表示}

UML中,一个类表示为一个最多具有三个部分的框,包含以下信息:

\begin{itemize}
\item
类的名字

\item
类的数据成员

\item
类的成员函数
\end{itemize}

图 D.1 显示了一个例子。MyClass有两个数据成员,一个是string类型,另一个是float类型,还有两个成员函数。每个成员前面的加号和减号指定了其可见性。下表列出了最常用的可见性:

\myGraphic{0.4}{content/part6/appendix-D/images/1.png}{图 D.1}

% Please add the following required packages to your document preamble:
% \usepackage{longtable}
% Note: It may be necessary to compile the document several times to get a multi-page table to line up properly
\begin{longtable}{|l|l|}
\hline
\textbf{可见性} & \textbf{意义} \\ \hline
\endfirsthead
%
\endhead
%
+                   & 公有成员    \\ \hline
-                   & 私有成员   \\ \hline
\#                  & 受保护成员 \\ \hline
\end{longtable}

根据类图目标,有时会省略成员的细节,类可表示为一个简单的框,如图 D.2 所示。例如,只对可视化不同类之间的关系感兴趣,就可以使用这种表示方式。

\myGraphic{0.2}{content/part6/appendix-D/images/2.png}{图 D.2}

\mySubsubsection{D.2.2}{关系表示}

UML 2 支持六种类之间的关系:继承、实现/实施、聚合、组合、关联和依赖。以下各节将介绍这些关系。

\mySamllsection{继承}

继承通过一条线来可视化,这条线从派生类(或称为子类)指向基类(或称为父类)。这条线在基类的一端以一个空心三角形结束,表示is-a关系。图 D.3 显示了一个例子。

\myGraphic{0.5}{content/part6/appendix-D/images/3.png}{图 D.3}

\mySamllsection{认识/实现}

实现接口的类基本上是从该接口继承(is-a关系),但为了区分通用继承和接口实现,后者类似继承但使用虚线而不是实线表示,如图 D.4 所示。ListBox类从UIElement派生,并实现了IClickable和IScrollable接口。

\myGraphic{0.6}{content/part6/appendix-D/images/4.png}{图 D.4}

\mySamllsection{聚集}

聚集表示has-a关系使用一条带有空心菱形形状的线表示,在包含另一个类实例或实例的类的一侧。在聚集关系中,还可以选择性地指定关系中每个参与者的关系多重性。多重性的位置,即在连线的哪一侧书写,一开始可能会令人困惑。例如,在图 D.5 中,一个Class可以包含一个或多个Students,每个Student可以跟随零个或多个Classes。聚集关系意味着即使销毁聚合器,聚合的对象或对象也可以继续存活。例如,销毁一个Class,它的Students不会被销毁。

\myGraphic{0.6}{content/part6/appendix-D/images/5.png}{图 D.5}

下表列出了一些可能的多重性示例:

% Please add the following required packages to your document preamble:
% \usepackage{longtable}
% Note: It may be necessary to compile the document several times to get a multi-page table to line up properly
\begin{longtable}{|l|l|}
\hline
\textbf{多重性} & \textbf{意义}       \\ \hline
\endfirsthead
%
\endhead
%
N                     & 正好N个实例    \\ \hline
0..1                  & 零个或一个实例   \\ \hline
0..*                  & 零个或多个实例 \\ \hline
N..*                  & N个或更多实例    \\ \hline
\end{longtable}

\mySamllsection{组合}

组合类似于聚集,并且视觉上几乎相同,只是使用实心菱形代替空心菱形。与聚集相反,如果销毁包含其他类实例的类,则那些包含的实例也会销毁。图 D.6 显示了一个例子。一个Window可以包含零个或多个Buttons,每个Button必须在一个Window中。如果销毁Window,所有它包含的Buttons也会销毁。

\myGraphic{0.5}{content/part6/appendix-D/images/6.png}{图 D.6}

\mySamllsection{关联}

关联是聚集的一种泛化,代表了类之间的二元链接,而聚集是一个单向链接。二元链接可以在两个方向上遍历。图 D.7 显示了一个例子。每本书都知道它的作者是谁,每位作者也知道她写了哪些书。

\myGraphic{0.5}{content/part6/appendix-D/images/7.png}{图 D.7}

\mySamllsection{依赖}

依赖表示一个类依赖于另一个类,表示为一个箭头指向依赖类的虚线。通常,虚线上有一些文字描述依赖关系。回到第33章中的汽车工厂例子,CarFactory依赖于Car,因为工厂创造汽车。这在图 D.8 中可视化为:

\myGraphic{0.5}{content/part6/appendix-D/images/8.png}{图 D.8}

