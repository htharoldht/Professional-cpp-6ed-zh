
与“程序做什么?”不同,面向对象提出另一个问题:“我正在建模哪些现实世界的对象?” OOP 基于这样的观念:应该将程序划分成物理对象的模型,而不是任务。尽管这最初听起来有些抽象,但当从类、组件、属性和行为的角度考虑对象时,就会变得很清晰。

\mySubsubsection{5.2.1}{类}

类有助于区分对象与其定义,以橙子为例。橙子作为美味的水果,生长在树上,其含义与我现在正在吃的那一个不同。

回答“橙子是什么?”时,是在谈论称为“橙子”的类。所有的橙子都是水果,并且所有的橙子都生长在树上,可能所有的橙子都是橙色,所有的橙子都有某种特定的味道,所以类定义的是一类对象。

当描述一个特定的橙子时,是在谈论一个对象。所有的对象都属于一个特定的类。因为桌上的那个橙子是橙子,知道它属于橙子类,因此知道它是长在树上的水果。还可以进一步说它是中等橙色,味道“相当美味”。一个对象是类的实例——具有区分同一类其他实例的特征。

作为一个更具体的例子,再考虑之前的股票选择应用程序。在 OOP 中,“股票报价”是一个类,因为它定义了报价的抽象概念。特定的报价,比如“当前微软股票报价”,是一个对象,因为这是一类特定实例。

从 C 背景出发,将类和对象视为类型和变量的类似物。实际上,第1章展示了类的语法与 C 结构体的语法相似。

\mySubsubsection{5.2.2}{对象}

如果考虑一个复杂的现实世界对象,比如飞机,应该很容易就能看出它是由更小的组件组成的,包括机身、控制、起落架、引擎和其他许多部分。将对象视为其更小组件的能力是面向对象编程的关键,正如将复杂任务分解为更小的过程是程序化编程的基础一样。

组件本质上与类相同,只是更小、更具体。一个好的面向对象的程序可能有一个 Airplane 类,但若完全描述了一架飞机,这个类会非常大。Airplane 类处理许多更小、更可管理的部分,每个这些组件可能还有进一步的子组件。例如,起落架是飞机的一个组件,而轮子是起落架中的一个组件。

\mySubsubsection{5.2.3}{属性}

属性是区分一个对象与其他对象的特征。回到橙子类,回想一下所有橙子都定义为具有某种橙色和特定的味道。这两个特征是属性。所有的橙子都有相同的属性,只是值不同。我的橙子味道“非常美味”,而你的味道可能“非常糟糕”。

这也可以在类层面上考虑属性,所有的橙子都是水果,都生长在树上。这些都是水果类的属性,而具体的橙色则由特定的水果对象决定。类的所有对象共享属性,而对象属性存在于类的所有对象中,但具有不同的值。

在股票选择示例中,股票报价有几个对象属性,包括公司名称、股票代码、当前价格和其他统计数据。

属性描述了一个对象的特征,其回答了“是什么使这个对象与众不同?”。

\mySubsubsection{5.2.4}{行为}

行为回答以下两个问题之一:“这个对象做什么?”或“能对这个对象做什么?”以橙子为例,我们可以对它做些事情,一种情况是吃掉它。就像属性一样,可以在类层面或对象层面上考虑行为。所有的橙子基本上都可以以相同的方式吃掉,但它们在某些其他行为上可能会有所不同,例如滚下斜坡,其中完美圆形的橙子的行为与更扁平的橙子会有所不同。

股票选择示例提供了一些更实用的行为。当我从程序的角度思考时,我确定了我的程序需要分析股票报价作为其功能之一。从面向对象的角度思考,可能会决定股票报价对象可以分析自己,分析成为股票报价对象的一个行为。

面向对象编程中,大部分功能代码被从过程移出,放入类中。通过构建具有某些行为的类并定义如何交互,面向对象编程提供了一种更丰富的机制,将代码附加到它操作的数据上,类的行为在类成员函数中实现。

如第4章所述,C++ 是一种支持多范式的语言,既支持面向对象编程,也支持程序化编程。C++ 不会强迫将所有东西都放入类中,就像 Java 一样。在 C++ 中,当认为面向对象编程有意义时,可以自由地使用类,但与程序化编程结合使用并没有什么不对,而且可以将某些功能保持在独立的函数中。事实上,C++ 标准库的大部分功能都是以独立的函数形式提供的。

\mySubsubsection{5.2.5}{将一切放在一起}

有了这些概念,可以再次审视股票选择程序,以面向对象的方式重新设计。

正如所讨论的,“股票报价”可以作为一个很好的类开始。为了获取报价列表,程序需要一个表示一组股票报价的概念,这通常称为集合。一个更好的设计可能是创建一个表示“股票报价集合”的类,可由代表单个“股票报价”的较小组件组成。

继续考虑属性,集合类将至少有一个属性——实际接收的报价列表。还可能有其他属性,例如最近一次检索的确切日期和时间。至于行为,“股票报价集合”能够与服务器通信以获取报价,并提供一个排序后的报价列表,这些都是“获取报价”和“排序报价”的行为。

股票报价类将具有前面讨论的属性——名称、代码、当前价格等,它还将有一个分析行为。还需要考虑其他行为,例如购买和出售股票。

创建显示组件之间关系图是个好习惯。图 5.1 使用 UML 类图语法,见附录 D,来表示 StockQuoteCollection 包含零个或多个(0…*)StockQuote 对象,并且一个 StockQuote 对象属于一个(1)StockQuoteCollection。

\myGraphic{0.7}{content/part2/chapter5/images/1.png}{图 5.1}

来看一个第二个例子。如前所述,橙子具有颜色、味道等属性,以及可食用、滚动等行为。可以想出更多行为,例如抛掷、剥皮或挤汁。另一个橙子的属性可能是其种子的集合。图 5.2 显示了橙子和种子类的可能 UML 类图,包括一个橙子包含零个或多个(0…*)种子,并且一个种子属于一个(1)橙子的关系。

\myGraphic{0.5}{content/part2/chapter5/images/2.png}{图 5.2}
























