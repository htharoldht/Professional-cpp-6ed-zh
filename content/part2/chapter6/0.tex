\noindent
\textbf{内容概要}

\begin{itemize}
\item
重用思想:为重用而设计

\item
如何设计可重用代码

\item
如何抽象

\item
为重用而组织代码的策略

\item
设计可用界面的六个策略

\item
如何调和通用性与易用性

\item
SOLID原则
\end{itemize}

第4章中讨论在程序中重用库和其他代码是一个重要的设计策略,而这只是重用策略的一半。另一半是设计和编写自己的代码,以便可以在程序中重用。设计良好的库和设计不良的库之间存在显著差异,设计良好的库使用起来令人愉快,而设计不良的库可能会让你厌恶并决定自己编写代码。无论是在编写一个明确为其他开发者设计的库,还是在决定类层次结构,都应该在设计代码时考虑到重用,而永远不知道在后续项目中何时会需要类似的功能。

第4章介绍了重用的设计主题,并解释了如何通过将库和其他代码融入你的设计中应用这一主题,但并没有解释如何设计可重用的代码,而这就是本章的主题,建立在第5章中描述的面向对象设计原则之上。
